%% Generated by Sphinx.
\def\sphinxdocclass{report}
\documentclass[a4paper,11pt,russian,openany,oneside]{sphinxmanual}
\ifdefined\pdfpxdimen
   \let\sphinxpxdimen\pdfpxdimen\else\newdimen\sphinxpxdimen
\fi \sphinxpxdimen=.75bp\relax
\ifdefined\pdfimageresolution
    \pdfimageresolution= \numexpr \dimexpr1in\relax/\sphinxpxdimen\relax
\fi
%% let collapsible pdf bookmarks panel have high depth per default
\PassOptionsToPackage{bookmarksdepth=5}{hyperref}
%% turn off hyperref patch of \index as sphinx.xdy xindy module takes care of
%% suitable \hyperpage mark-up, working around hyperref-xindy incompatibility
\PassOptionsToPackage{hyperindex=false}{hyperref}
%% memoir class requires extra handling
\makeatletter\@ifclassloaded{memoir}
{\ifdefined\memhyperindexfalse\memhyperindexfalse\fi}{}\makeatother

\PassOptionsToPackage{booktabs}{sphinx}
\PassOptionsToPackage{colorrows}{sphinx}

\PassOptionsToPackage{warn}{textcomp}

\catcode`^^^^00a0\active\protected\def^^^^00a0{\leavevmode\nobreak\ }
\usepackage{cmap}
\usepackage{fontspec}
\defaultfontfeatures[\rmfamily,\sffamily,\ttfamily]{}
\usepackage{amsmath,amssymb,amstext}
\usepackage{polyglossia}
\setmainlanguage{russian}



\setmainfont{FreeSerif}[
  Extension      = .otf,
  UprightFont    = *,
  ItalicFont     = *Italic,
  BoldFont       = *Bold,
  BoldItalicFont = *BoldItalic
]
\setsansfont{FreeSans}[
  Extension      = .otf,
  UprightFont    = *,
  ItalicFont     = *Oblique,
  BoldFont       = *Bold,
  BoldItalicFont = *BoldOblique,
]
\setmonofont{FreeMono}[Scale=0.9,
  Extension      = .otf,
  UprightFont    = *,
  ItalicFont     = *Oblique,
  BoldFont       = *Bold,
  BoldItalicFont = *BoldOblique,
]



\usepackage[Sonny]{fncychap}
\ChNameVar{\Large\normalfont\sffamily}
\ChTitleVar{\Large\normalfont\sffamily}
\usepackage[,maxlistdepth=10]{sphinx}

\fvset{fontsize=auto}
\usepackage{geometry}


% Include hyperref last.
\usepackage{hyperref}
% Fix anchor placement for figures with captions.
\usepackage{hypcap}% it must be loaded after hyperref.
% Set up styles of URL: it should be placed after hyperref.
\urlstyle{same}

\addto\captionsrussian{\renewcommand{\contentsname}{Содержание:}}

\usepackage{sphinxmessages}
\setcounter{tocdepth}{1}


        \usepackage{fontspec}
        \setmainfont{DejaVu Serif}
        \setsansfont{DejaVu Sans}
        \setmonofont{DejaVu Sans Mono}
        \usepackage{polyglossia}
        \setdefaultlanguage{russian}
        \setotherlanguage{english}
        
        % Настройки для кода
        \usepackage{minted}
        
        % Улучшенное оглавление
        \usepackage{tocloft}
        \setlength{\cftbeforesecskip}{2pt}
        
        % Настройки заголовков
        \usepackage{titlesec}
        \titleformat{\chapter}[display]
            {\normalfont\huge\bfseries}{\chaptertitlename\ \thechapter}{20pt}{\Huge}
        
        % Отступы и межстрочный интервал
        \usepackage{parskip}
        \setlength{\parskip}{6pt}
        \linespread{1.1}
        
        % Настройки для списков
        \usepackage{enumitem}
        \setlist{nosep}
        
        % Улучшенное оформление ссылок
        \usepackage{hyperref}
        \hypersetup{
            colorlinks=true,
            linkcolor=blue,
            filecolor=magenta,      
            urlcolor=cyan,
            citecolor=green,
        }
    

\title{KLA Project Documentation}
\date{окт. 03, 2025}
\release{1.0}
\author{diana}
\newcommand{\sphinxlogo}{\vbox{}}
\renewcommand{\releasename}{Выпуск}
\makeindex
\begin{document}

\pagestyle{empty}
\sphinxmaketitle
\pagestyle{plain}

        \renewcommand{\contentsname}{Оглавление}
        \setcounter{tocdepth}{3}
        \tableofcontents
    
\pagestyle{normal}
\phantomsection\label{\detokenize{index::doc}}


\sphinxAtStartPar
KLA \sphinxhyphen{} это анализатор раскладок клавиатуры, который позволяет оценить эффективность различных раскладок на основе анализа текстов и словарей.


\chapter{Основные возможности}
\label{\detokenize{index:id1}}\begin{itemize}
\item {} 
\sphinxAtStartPar
Анализ текстовых файлов и словарей

\item {} 
\sphinxAtStartPar
Поддержка больших файлов с потоковой обработкой

\item {} 
\sphinxAtStartPar
Сохранение результатов в базе данных

\item {} 
\sphinxAtStartPar
Создание отчетов и графиков

\item {} 
\sphinxAtStartPar
Сравнение различных раскладок

\item {} 
\sphinxAtStartPar
Интерактивное меню для удобной работы

\end{itemize}


\chapter{Быстрый старт}
\label{\detokenize{index:id2}}\begin{enumerate}
\sphinxsetlistlabels{\arabic}{enumi}{enumii}{}{.}%
\item {} 
\sphinxAtStartPar
Установите зависимости: \sphinxcode{\sphinxupquote{pip install \sphinxhyphen{}r requirements.txt}}

\item {} 
\sphinxAtStartPar
Запустите программу: \sphinxcode{\sphinxupquote{uv run main.py}}

\item {} 
\sphinxAtStartPar
Выберите раскладку для тестирования

\item {} 
\sphinxAtStartPar
Загрузите файл для анализа

\end{enumerate}


\chapter{Содержание документации}
\label{\detokenize{index:id3}}
\sphinxstepscope


\section{Установка и настройка}
\label{\detokenize{installation:id1}}\label{\detokenize{installation::doc}}

\subsection{Системные требования}
\label{\detokenize{installation:id2}}\begin{itemize}
\item {} 
\sphinxAtStartPar
Python 3.13 или выше

\item {} 
\sphinxAtStartPar
SQLite3 (обычно входит в состав Python)

\item {} 
\sphinxAtStartPar
Операционная система: Windows, macOS, Linux

\end{itemize}


\subsection{Установка зависимостей}
\label{\detokenize{installation:id3}}\begin{enumerate}
\sphinxsetlistlabels{\arabic}{enumi}{enumii}{}{.}%
\item {} 
\sphinxAtStartPar
Клонируйте репозиторий или скачайте исходный код

\item {} 
\sphinxAtStartPar
Перейдите в директорию проекта

\item {} 
\sphinxAtStartPar
Установите зависимости:

\end{enumerate}

\begin{sphinxVerbatim}[commandchars=\\\{\}]
pip\PYG{+w}{ }install\PYG{+w}{ }\PYGZhy{}r\PYG{+w}{ }requirements.txt
\end{sphinxVerbatim}

\sphinxAtStartPar
Или с использованием uv:

\begin{sphinxVerbatim}[commandchars=\\\{\}]
uv\PYG{+w}{ }sync
\end{sphinxVerbatim}


\subsection{Зависимости}
\label{\detokenize{installation:id4}}
\sphinxAtStartPar
Проект использует следующие библиотеки: диана
\begin{itemize}
\item {} 
\sphinxAtStartPar
\sphinxstylestrong{matplotlib} (>=3.10.6) \sphinxhyphen{} для создания графиков и диаграмм.

\end{itemize}
\begin{quote}

\sphinxAtStartPar
Достаточно мощная и гибкая библиотека для визуализации данных в Python,
которая считается стандартом для создания статических, анимированных и интерактивных визуализаций.
\begin{quote}

\sphinxAtStartPar
Почему Matplotlib?
\end{quote}
\end{quote}

\sphinxAtStartPar
1. Он обладает всесторонней функциональностью
Matplotlib поддерживает невероятно широкий спектр типов визуализаций:
Линейные графики (line plots)
Точечные диаграммы (scatter plots)
Гистограммы и столбчатые диаграммы
Круговые диаграммы (pie charts)
Контурные графики и градиентные поля
Трехмерные визуализации (через toolkit mplot3d)
2. Matplotlib позволяет создавать графики высокого качества,
а также поддерживает экспорт в различные форматы (PNG, PDF, SVG, EPS, JPEG) с контролем разрешения (DPI) и размера.
3. При необходимости функционал Matplotlib можно расширить большим количеством специально созданных для него пакетов и расширений
Из альтернативных вариантов мы рассмотрели:Plotly
\sphinxhyphen{}Она фокусируется на интерактивных и веб\sphinxhyphen{}ориентированных визуализациях, но Matplotlib,
\begin{quote}

\sphinxAtStartPar
оказался проще в освоении и удобнее при использовании не интерактивных графиков.
\end{quote}
\begin{itemize}
\item {} 
\sphinxAtStartPar
\sphinxstylestrong{pytest} (>=8.4.2) \sphinxhyphen{} для тестирования

\item {} 
\sphinxAtStartPar
\sphinxstylestrong{sphinx} (>=8.2.3) \sphinxhyphen{} для генерации документации

\item {} 
\sphinxAtStartPar
\sphinxstylestrong{sphinx\sphinxhyphen{}rtd\sphinxhyphen{}theme} (>=3.0.2) \sphinxhyphen{} тема для документации

\item {} \begin{description}
\sphinxlineitem{\sphinxstylestrong{tqdm} (>=4.67.1) \sphinxhyphen{} для отображения прогресс\sphinxhyphen{}баров}
\sphinxAtStartPar
Почему TQDM?

\end{description}

\end{itemize}
\begin{enumerate}
\sphinxsetlistlabels{\arabic}{enumi}{enumii}{}{.}%
\item {} 
\sphinxAtStartPar
Простота и удобство использования

\item {} 
\sphinxAtStartPar
Минимальная нагрузка на выполнение кода

\item {} 
\sphinxAtStartPar
tqdm корректно работает в различных средах и графических интерфейсах.

\item {} 
\sphinxAtStartPar
Гибкая настройка отображения и кастомизации

\end{enumerate}

\sphinxAtStartPar
Альтернативные варианты, которые мы рассматривали:
Progressbar2
TQDM  предлагает более простой синтаксис и лучшее быстродействие по сравнению с progressbar2.

\sphinxAtStartPar
Alive\sphinxhyphen{}progres
В отличие от alive\sphinxhyphen{}progress с богатыми анимациями,TQDM предоставляет минималистичный и производительный подход,
\begin{quote}

\sphinxAtStartPar
что лучше подходит для наших задач.
\end{quote}


\subsection{Первый запуск}
\label{\detokenize{installation:id5}}
\sphinxAtStartPar
При первом запуске программа автоматически:
\begin{enumerate}
\sphinxsetlistlabels{\arabic}{enumi}{enumii}{}{.}%
\item {} 
\sphinxAtStartPar
Создаст базу данных SQLite (\sphinxcode{\sphinxupquote{database.db}})

\item {} 
\sphinxAtStartPar
Инициализирует необходимые таблицы

\item {} 
\sphinxAtStartPar
Добавит тестовые данные

\end{enumerate}


\subsection{Структура проекта}
\label{\detokenize{installation:id6}}
\begin{sphinxVerbatim}[commandchars=\\\{\}]
\PYG{n}{kva}\PYG{o}{/}
\PYG{err}{├}\PYG{err}{─}\PYG{err}{─} \PYG{n}{main}\PYG{o}{.}\PYG{n}{py}                    \PYG{c+c1}{\PYGZsh{} Главный файл программы}
\PYG{err}{├}\PYG{err}{─}\PYG{err}{─} \PYG{n}{data\PYGZus{}module}\PYG{o}{/}               \PYG{c+c1}{\PYGZsh{} Модуль экспорта данных}
\PYG{err}{├}\PYG{err}{─}\PYG{err}{─} \PYG{n}{database\PYGZus{}module}\PYG{o}{/}           \PYG{c+c1}{\PYGZsh{} Модуль работы с БД}
\PYG{err}{├}\PYG{err}{─}\PYG{err}{─} \PYG{n}{processing\PYGZus{}module}\PYG{o}{/}         \PYG{c+c1}{\PYGZsh{} Модуль обработки текстов}
\PYG{err}{├}\PYG{err}{─}\PYG{err}{─} \PYG{n}{scan\PYGZus{}module}\PYG{o}{/}              \PYG{c+c1}{\PYGZsh{} Модуль чтения файлов}
\PYG{err}{├}\PYG{err}{─}\PYG{err}{─} \PYG{n}{tests\PYGZus{}module}\PYG{o}{/}             \PYG{c+c1}{\PYGZsh{} Модуль тестов}
\PYG{err}{├}\PYG{err}{─}\PYG{err}{─} \PYG{n}{output\PYGZus{}data}\PYG{o}{/}              \PYG{c+c1}{\PYGZsh{} Выходные данные}
\PYG{err}{├}\PYG{err}{─}\PYG{err}{─} \PYG{n}{example\PYGZus{}layouts}\PYG{o}{/}          \PYG{c+c1}{\PYGZsh{} Примеры раскладок}
\PYG{err}{├}\PYG{err}{─}\PYG{err}{─} \PYG{n}{docs}\PYG{o}{/}                     \PYG{c+c1}{\PYGZsh{} Документация}
\PYG{err}{└}\PYG{err}{─}\PYG{err}{─} \PYG{n}{requirements}\PYG{o}{.}\PYG{n}{txt}          \PYG{c+c1}{\PYGZsh{} Зависимости}
\end{sphinxVerbatim}

\sphinxstepscope


\section{Руководство пользователя}
\label{\detokenize{usage:id1}}\label{\detokenize{usage::doc}}

\subsection{Запуск программы}
\label{\detokenize{usage:id2}}
\sphinxAtStartPar
Для запуска программы выполните:

\begin{sphinxVerbatim}[commandchars=\\\{\}]
python\PYG{+w}{ }main.py
\end{sphinxVerbatim}


\subsection{Главное меню}
\label{\detokenize{usage:id3}}
\sphinxAtStartPar
После запуска откроется главное меню с опциями:
\begin{enumerate}
\sphinxsetlistlabels{\arabic}{enumi}{enumii}{}{.}%
\item {} 
\sphinxAtStartPar
\sphinxstylestrong{Выбрать раскладку для тестирования} \sphinxhyphen{} основная функция программы

\item {} 
\sphinxAtStartPar
\sphinxstylestrong{Выход из программы} \sphinxhyphen{} завершение работы

\end{enumerate}


\subsection{Выбор раскладки}
\label{\detokenize{usage:id4}}
\sphinxAtStartPar
В меню выбора раскладки вы увидите:
\begin{itemize}
\item {} 
\sphinxAtStartPar
Список доступных раскладок в базе данных

\item {} 
\sphinxAtStartPar
Возможность выбрать раскладку для анализа

\item {} 
\sphinxAtStartPar
Переход к меню обработки файлов

\end{itemize}


\subsection{Обработка файлов}
\label{\detokenize{usage:id5}}
\sphinxAtStartPar
После выбора раскладки доступны следующие опции:


\subsubsection{Обработка файла со словами}
\label{\detokenize{usage:id6}}\begin{itemize}
\item {} 
\sphinxAtStartPar
Анализирует файл, где каждая строка содержит отдельное слово

\item {} 
\sphinxAtStartPar
Подходит для словарей и списков слов

\item {} 
\sphinxAtStartPar
Поддерживает большие файлы с потоковой обработкой

\end{itemize}


\subsubsection{Обработка текстового файла}
\label{\detokenize{usage:id7}}\begin{itemize}
\item {} 
\sphinxAtStartPar
Анализирует сплошной текст

\item {} 
\sphinxAtStartPar
Подходит для книг, статей, документов

\item {} 
\sphinxAtStartPar
Автоматически подсчитывает количество слов

\end{itemize}


\subsection{Работа с большими файлами}
\label{\detokenize{usage:id8}}
\sphinxAtStartPar
Программа автоматически определяет размер файла и выбирает оптимальный метод обработки:
\begin{itemize}
\item {} 
\sphinxAtStartPar
\sphinxstylestrong{Файлы до 10 MB} \sphinxhyphen{} загружаются в память полностью

\item {} 
\sphinxAtStartPar
\sphinxstylestrong{Файлы 10\sphinxhyphen{}50 MB} \sphinxhyphen{} обрабатываются потоково с предупреждением

\item {} 
\sphinxAtStartPar
\sphinxstylestrong{Файлы свыше 50 MB} \sphinxhyphen{} требуют подтверждения пользователя

\end{itemize}


\subsection{Результаты анализа}
\label{\detokenize{usage:id9}}
\sphinxAtStartPar
После обработки файла отображается подробная статистика:


\subsubsection{Основные метрики}
\label{\detokenize{usage:id10}}\begin{itemize}
\item {} 
\sphinxAtStartPar
\sphinxstylestrong{Обработано слов} \sphinxhyphen{} количество проанализированных слов

\item {} 
\sphinxAtStartPar
\sphinxstylestrong{Всего символов} \sphinxhyphen{} общее количество символов в файле

\item {} 
\sphinxAtStartPar
\sphinxstylestrong{Обработано символов} \sphinxhyphen{} символы, для которых есть правила в раскладке

\item {} 
\sphinxAtStartPar
\sphinxstylestrong{Общее количество ошибок} \sphinxhyphen{} суммарное количество ошибок

\end{itemize}


\subsubsection{Средние значения}
\label{\detokenize{usage:id11}}\begin{itemize}
\item {} 
\sphinxAtStartPar
\sphinxstylestrong{Среднее ошибок на слово} \sphinxhyphen{} средняя сложность слов

\item {} 
\sphinxAtStartPar
\sphinxstylestrong{Среднее ошибок на символ} \sphinxhyphen{} средняя сложность символов

\item {} 
\sphinxAtStartPar
\sphinxstylestrong{Покрытие раскладкой} \sphinxhyphen{} процент символов, покрытых правилами

\end{itemize}


\subsubsection{Оценка качества}
\label{\detokenize{usage:id12}}
\sphinxAtStartPar
Программа автоматически оценивает качество раскладки:
\begin{itemize}
\item {} 
\sphinxAtStartPar
\sphinxstylestrong{ОТЛИЧНО} \sphinxhyphen{} менее 2 ошибок на слово

\item {} 
\sphinxAtStartPar
\sphinxstylestrong{ХОРОШО} \sphinxhyphen{} 2\sphinxhyphen{}5 ошибок на слово

\item {} 
\sphinxAtStartPar
\sphinxstylestrong{СРЕДНЕ} \sphinxhyphen{} 5\sphinxhyphen{}10 ошибок на слово

\item {} 
\sphinxAtStartPar
\sphinxstylestrong{ПЛОХО} \sphinxhyphen{} более 10 ошибок на слово

\end{itemize}


\subsection{Сохранение результатов}
\label{\detokenize{usage:id13}}
\sphinxAtStartPar
После анализа можно:
\begin{enumerate}
\sphinxsetlistlabels{\arabic}{enumi}{enumii}{}{.}%
\item {} 
\sphinxAtStartPar
\sphinxstylestrong{Сохранить в базу данных} \sphinxhyphen{} для истории и сравнения

\item {} 
\sphinxAtStartPar
\sphinxstylestrong{Экспортировать в CSV} \sphinxhyphen{} для внешнего анализа

\item {} 
\sphinxAtStartPar
\sphinxstylestrong{Создать графики} \sphinxhyphen{} визуализация результатов

\item {} 
\sphinxAtStartPar
\sphinxstylestrong{Экспортировать неизвестные символы} \sphinxhyphen{} для улучшения раскладки

\end{enumerate}


\subsection{История анализов}
\label{\detokenize{usage:id14}}
\sphinxAtStartPar
В разделе «История анализов» можно:
\begin{itemize}
\item {} 
\sphinxAtStartPar
Просмотреть статистику по раскладке

\item {} 
\sphinxAtStartPar
Увидеть последние тесты

\item {} 
\sphinxAtStartPar
Сравнить результаты разных файлов

\end{itemize}


\subsection{Создание графиков}
\label{\detokenize{usage:id15}}
\sphinxAtStartPar
Доступны два типа графиков:
\begin{enumerate}
\sphinxsetlistlabels{\arabic}{enumi}{enumii}{}{.}%
\item {} 
\sphinxAtStartPar
\sphinxstylestrong{График истории} \sphinxhyphen{} показывает изменение результатов во времени

\item {} 
\sphinxAtStartPar
\sphinxstylestrong{Сравнение раскладок} \sphinxhyphen{} сравнивает эффективность разных раскладок

\end{enumerate}


\subsection{Загрузка раскладки из файла}
\label{\detokenize{usage:id16}}
\sphinxAtStartPar
Можно загрузить собственную раскладку:
\begin{enumerate}
\sphinxsetlistlabels{\arabic}{enumi}{enumii}{}{.}%
\item {} 
\sphinxAtStartPar
Подготовьте файл с раскладкой в нужном формате

\item {} 
\sphinxAtStartPar
Выберите «Загрузить раскладку из файла»

\item {} 
\sphinxAtStartPar
Укажите путь к файлу

\item {} 
\sphinxAtStartPar
Программа проверит корректность и предложит сохранить в БД

\end{enumerate}


\subsection{Советы по использованию}
\label{\detokenize{usage:id17}}\begin{itemize}
\item {} 
\sphinxAtStartPar
Используйте файлы в кодировке UTF\sphinxhyphen{}8 для лучшей совместимости

\item {} 
\sphinxAtStartPar
Для больших файлов убедитесь в наличии достаточного места на диске

\item {} 
\sphinxAtStartPar
Регулярно сохраняйте результаты в базу данных для анализа прогресса

\item {} 
\sphinxAtStartPar
Создавайте графики для визуального сравнения раскладок

\end{itemize}

\sphinxstepscope


\section{Модули проекта KVA}
\label{\detokenize{modules:kva}}\label{\detokenize{modules::doc}}
\sphinxstepscope


\subsection{Главный модуль (main)}
\label{\detokenize{main:main}}\label{\detokenize{main::doc}}

\subsubsection{Классы}
\label{\detokenize{main:id1}}

\paragraph{MenuAction}
\label{\detokenize{main:menuaction}}

\paragraph{MenuSystem}
\label{\detokenize{main:menusystem}}

\subsubsection{Функции}
\label{\detokenize{main:id2}}
\sphinxstepscope


\subsection{Модуль данных (data\_module)}
\label{\detokenize{data_module:data-module}}\label{\detokenize{data_module::doc}}
\sphinxAtStartPar
Модуль для экспорта и создания отчетов по результатам анализа.
\index{module@\spxentry{module}!data\_module@\spxentry{data\_module}}\index{data\_module@\spxentry{data\_module}!module@\spxentry{module}}

\subsubsection{Подмодули}
\label{\detokenize{data_module:module-data_module}}\label{\detokenize{data_module:id1}}

\paragraph{make\_export\_file}
\label{\detokenize{data_module:module-data_module.make_export_file}}\label{\detokenize{data_module:make-export-file}}\index{module@\spxentry{module}!data\_module.make\_export\_file@\spxentry{data\_module.make\_export\_file}}\index{data\_module.make\_export\_file@\spxentry{data\_module.make\_export\_file}!module@\spxentry{module}}
\sphinxAtStartPar
Модуль для экспорта результатов анализа в CSV файлы
\index{create\_csv\_report() (в модуле data\_module.make\_export\_file)@\spxentry{create\_csv\_report()}\spxextra{в модуле data\_module.make\_export\_file}}

\begin{savenotes}\begin{fulllineitems}
\phantomsection\label{\detokenize{data_module:data_module.make_export_file.create_csv_report}}
\pysigstartsignatures
\pysiglinewithargsret
{\sphinxcode{\sphinxupquote{data\_module.make\_export\_file.}}\sphinxbfcode{\sphinxupquote{create\_csv\_report}}}
{\sphinxparam{\DUrole{n}{result}}\sphinxparamcomma \sphinxparam{\DUrole{n}{file\_path}}\sphinxparamcomma \sphinxparam{\DUrole{n}{layout\_name}}\sphinxparamcomma \sphinxparam{\DUrole{n}{output\_dir}\DUrole{o}{=}\DUrole{default_value}{'reports'}}}
{}
\pysigstopsignatures
\sphinxAtStartPar
Создает CSV отчет с результатами анализа
\begin{quote}\begin{description}
\sphinxlineitem{Параметры}\begin{itemize}
\item {} 
\sphinxAtStartPar
\sphinxstyleliteralstrong{\sphinxupquote{result}} (\sphinxstyleliteralemphasis{\sphinxupquote{Dict}}\sphinxstyleliteralemphasis{\sphinxupquote{{[}}}\sphinxstyleliteralemphasis{\sphinxupquote{str}}\sphinxstyleliteralemphasis{\sphinxupquote{, }}\sphinxstyleliteralemphasis{\sphinxupquote{Any}}\sphinxstyleliteralemphasis{\sphinxupquote{{]}}}) – Результаты анализа

\item {} 
\sphinxAtStartPar
\sphinxstyleliteralstrong{\sphinxupquote{file\_path}} (\sphinxstyleliteralemphasis{\sphinxupquote{str}}) – Путь к анализируемому файлу

\item {} 
\sphinxAtStartPar
\sphinxstyleliteralstrong{\sphinxupquote{layout\_name}} (\sphinxstyleliteralemphasis{\sphinxupquote{str}}) – Название раскладки

\item {} 
\sphinxAtStartPar
\sphinxstyleliteralstrong{\sphinxupquote{output\_dir}} (\sphinxstyleliteralemphasis{\sphinxupquote{str}}) – Директория для сохранения отчетов

\end{itemize}

\sphinxlineitem{Результат}
\sphinxAtStartPar
Путь к созданному CSV файлу

\sphinxlineitem{Тип результата}
\sphinxAtStartPar
str

\end{description}\end{quote}

\end{fulllineitems}\end{savenotes}

\index{create\_detailed\_csv\_report() (в модуле data\_module.make\_export\_file)@\spxentry{create\_detailed\_csv\_report()}\spxextra{в модуле data\_module.make\_export\_file}}

\begin{savenotes}\begin{fulllineitems}
\phantomsection\label{\detokenize{data_module:data_module.make_export_file.create_detailed_csv_report}}
\pysigstartsignatures
\pysiglinewithargsret
{\sphinxcode{\sphinxupquote{data\_module.make\_export\_file.}}\sphinxbfcode{\sphinxupquote{create\_detailed\_csv\_report}}}
{\sphinxparam{\DUrole{n}{results\_list}}\sphinxparamcomma \sphinxparam{\DUrole{n}{output\_dir}\DUrole{o}{=}\DUrole{default_value}{'reports'}}}
{}
\pysigstopsignatures
\sphinxAtStartPar
Создает детальный CSV отчет для сравнения нескольких анализов
\begin{quote}\begin{description}
\sphinxlineitem{Параметры}\begin{itemize}
\item {} 
\sphinxAtStartPar
\sphinxstyleliteralstrong{\sphinxupquote{results\_list}} (\sphinxstyleliteralemphasis{\sphinxupquote{List}}\sphinxstyleliteralemphasis{\sphinxupquote{{[}}}\sphinxstyleliteralemphasis{\sphinxupquote{Dict}}\sphinxstyleliteralemphasis{\sphinxupquote{{[}}}\sphinxstyleliteralemphasis{\sphinxupquote{str}}\sphinxstyleliteralemphasis{\sphinxupquote{, }}\sphinxstyleliteralemphasis{\sphinxupquote{Any}}\sphinxstyleliteralemphasis{\sphinxupquote{{]}}}\sphinxstyleliteralemphasis{\sphinxupquote{{]}}}) – Список результатов анализа

\item {} 
\sphinxAtStartPar
\sphinxstyleliteralstrong{\sphinxupquote{output\_dir}} (\sphinxstyleliteralemphasis{\sphinxupquote{str}}) – Директория для сохранения отчетов

\end{itemize}

\sphinxlineitem{Результат}
\sphinxAtStartPar
Путь к созданному CSV файлу

\sphinxlineitem{Тип результата}
\sphinxAtStartPar
str

\end{description}\end{quote}

\end{fulllineitems}\end{savenotes}

\index{\_get\_quality\_assessment() (в модуле data\_module.make\_export\_file)@\spxentry{\_get\_quality\_assessment()}\spxextra{в модуле data\_module.make\_export\_file}}

\begin{savenotes}\begin{fulllineitems}
\phantomsection\label{\detokenize{data_module:data_module.make_export_file._get_quality_assessment}}
\pysigstartsignatures
\pysiglinewithargsret
{\sphinxcode{\sphinxupquote{data\_module.make\_export\_file.}}\sphinxbfcode{\sphinxupquote{\_get\_quality\_assessment}}}
{\sphinxparam{\DUrole{n}{avg\_errors\_per\_word}}}
{}
\pysigstopsignatures
\sphinxAtStartPar
Возвращает текстовую оценку качества
\begin{quote}\begin{description}
\sphinxlineitem{Параметры}
\sphinxAtStartPar
\sphinxstyleliteralstrong{\sphinxupquote{avg\_errors\_per\_word}} (\sphinxstyleliteralemphasis{\sphinxupquote{float}})

\sphinxlineitem{Тип результата}
\sphinxAtStartPar
str

\end{description}\end{quote}

\end{fulllineitems}\end{savenotes}

\index{export\_unknown\_characters\_csv() (в модуле data\_module.make\_export\_file)@\spxentry{export\_unknown\_characters\_csv()}\spxextra{в модуле data\_module.make\_export\_file}}

\begin{savenotes}\begin{fulllineitems}
\phantomsection\label{\detokenize{data_module:data_module.make_export_file.export_unknown_characters_csv}}
\pysigstartsignatures
\pysiglinewithargsret
{\sphinxcode{\sphinxupquote{data\_module.make\_export\_file.}}\sphinxbfcode{\sphinxupquote{export\_unknown\_characters\_csv}}}
{\sphinxparam{\DUrole{n}{unknown\_chars}}\sphinxparamcomma \sphinxparam{\DUrole{n}{layout\_name}}\sphinxparamcomma \sphinxparam{\DUrole{n}{output\_dir}\DUrole{o}{=}\DUrole{default_value}{'reports'}}}
{}
\pysigstopsignatures
\sphinxAtStartPar
Экспортирует список неизвестных символов в отдельный CSV файл
\begin{quote}\begin{description}
\sphinxlineitem{Параметры}\begin{itemize}
\item {} 
\sphinxAtStartPar
\sphinxstyleliteralstrong{\sphinxupquote{unknown\_chars}} (\sphinxstyleliteralemphasis{\sphinxupquote{set}}) – Множество неизвестных символов

\item {} 
\sphinxAtStartPar
\sphinxstyleliteralstrong{\sphinxupquote{layout\_name}} (\sphinxstyleliteralemphasis{\sphinxupquote{str}}) – Название раскладки

\item {} 
\sphinxAtStartPar
\sphinxstyleliteralstrong{\sphinxupquote{output\_dir}} (\sphinxstyleliteralemphasis{\sphinxupquote{str}}) – Директория для сохранения

\end{itemize}

\sphinxlineitem{Результат}
\sphinxAtStartPar
Путь к созданному файлу

\sphinxlineitem{Тип результата}
\sphinxAtStartPar
str

\end{description}\end{quote}

\end{fulllineitems}\end{savenotes}



\paragraph{make\_export\_plot}
\label{\detokenize{data_module:module-data_module.make_export_plot}}\label{\detokenize{data_module:make-export-plot}}\index{module@\spxentry{module}!data\_module.make\_export\_plot@\spxentry{data\_module.make\_export\_plot}}\index{data\_module.make\_export\_plot@\spxentry{data\_module.make\_export\_plot}!module@\spxentry{module}}
\sphinxAtStartPar
Модуль для создания графиков и визуализации результатов анализа
\index{create\_analysis\_charts() (в модуле data\_module.make\_export\_plot)@\spxentry{create\_analysis\_charts()}\spxextra{в модуле data\_module.make\_export\_plot}}

\begin{savenotes}\begin{fulllineitems}
\phantomsection\label{\detokenize{data_module:data_module.make_export_plot.create_analysis_charts}}
\pysigstartsignatures
\pysiglinewithargsret
{\sphinxcode{\sphinxupquote{data\_module.make\_export\_plot.}}\sphinxbfcode{\sphinxupquote{create\_analysis\_charts}}}
{\sphinxparam{\DUrole{n}{result}}\sphinxparamcomma \sphinxparam{\DUrole{n}{layout\_name}}\sphinxparamcomma \sphinxparam{\DUrole{n}{file\_path}}\sphinxparamcomma \sphinxparam{\DUrole{n}{output\_dir}\DUrole{o}{=}\DUrole{default_value}{'reports'}}}
{}
\pysigstopsignatures
\sphinxAtStartPar
Создает набор графиков для анализа результатов
\begin{quote}\begin{description}
\sphinxlineitem{Параметры}\begin{itemize}
\item {} 
\sphinxAtStartPar
\sphinxstyleliteralstrong{\sphinxupquote{result}} (\sphinxstyleliteralemphasis{\sphinxupquote{Dict}}\sphinxstyleliteralemphasis{\sphinxupquote{{[}}}\sphinxstyleliteralemphasis{\sphinxupquote{str}}\sphinxstyleliteralemphasis{\sphinxupquote{, }}\sphinxstyleliteralemphasis{\sphinxupquote{Any}}\sphinxstyleliteralemphasis{\sphinxupquote{{]}}}) – Результаты анализа

\item {} 
\sphinxAtStartPar
\sphinxstyleliteralstrong{\sphinxupquote{layout\_name}} (\sphinxstyleliteralemphasis{\sphinxupquote{str}}) – Название раскладки

\item {} 
\sphinxAtStartPar
\sphinxstyleliteralstrong{\sphinxupquote{file\_path}} (\sphinxstyleliteralemphasis{\sphinxupquote{str}}) – Путь к анализируемому файлу

\item {} 
\sphinxAtStartPar
\sphinxstyleliteralstrong{\sphinxupquote{output\_dir}} (\sphinxstyleliteralemphasis{\sphinxupquote{str}}) – Директория для сохранения графиков

\end{itemize}

\sphinxlineitem{Результат}
\sphinxAtStartPar
Список путей к созданным графикам

\sphinxlineitem{Тип результата}
\sphinxAtStartPar
List{[}str{]}

\end{description}\end{quote}

\end{fulllineitems}\end{savenotes}

\index{\_create\_coverage\_pie\_chart() (в модуле data\_module.make\_export\_plot)@\spxentry{\_create\_coverage\_pie\_chart()}\spxextra{в модуле data\_module.make\_export\_plot}}

\begin{savenotes}\begin{fulllineitems}
\phantomsection\label{\detokenize{data_module:data_module.make_export_plot._create_coverage_pie_chart}}
\pysigstartsignatures
\pysiglinewithargsret
{\sphinxcode{\sphinxupquote{data\_module.make\_export\_plot.}}\sphinxbfcode{\sphinxupquote{\_create\_coverage\_pie\_chart}}}
{\sphinxparam{\DUrole{n}{result}}\sphinxparamcomma \sphinxparam{\DUrole{n}{layout\_name}}\sphinxparamcomma \sphinxparam{\DUrole{n}{timestamp}}\sphinxparamcomma \sphinxparam{\DUrole{n}{output\_dir}}}
{}
\pysigstopsignatures
\sphinxAtStartPar
Создает круговую диаграмму покрытия символов
\begin{quote}\begin{description}
\sphinxlineitem{Параметры}\begin{itemize}
\item {} 
\sphinxAtStartPar
\sphinxstyleliteralstrong{\sphinxupquote{result}} (\sphinxstyleliteralemphasis{\sphinxupquote{Dict}}\sphinxstyleliteralemphasis{\sphinxupquote{{[}}}\sphinxstyleliteralemphasis{\sphinxupquote{str}}\sphinxstyleliteralemphasis{\sphinxupquote{, }}\sphinxstyleliteralemphasis{\sphinxupquote{Any}}\sphinxstyleliteralemphasis{\sphinxupquote{{]}}})

\item {} 
\sphinxAtStartPar
\sphinxstyleliteralstrong{\sphinxupquote{layout\_name}} (\sphinxstyleliteralemphasis{\sphinxupquote{str}})

\item {} 
\sphinxAtStartPar
\sphinxstyleliteralstrong{\sphinxupquote{timestamp}} (\sphinxstyleliteralemphasis{\sphinxupquote{str}})

\item {} 
\sphinxAtStartPar
\sphinxstyleliteralstrong{\sphinxupquote{output\_dir}} (\sphinxstyleliteralemphasis{\sphinxupquote{str}})

\end{itemize}

\sphinxlineitem{Тип результата}
\sphinxAtStartPar
str

\end{description}\end{quote}

\end{fulllineitems}\end{savenotes}

\index{\_create\_error\_distribution\_chart() (в модуле data\_module.make\_export\_plot)@\spxentry{\_create\_error\_distribution\_chart()}\spxextra{в модуле data\_module.make\_export\_plot}}

\begin{savenotes}\begin{fulllineitems}
\phantomsection\label{\detokenize{data_module:data_module.make_export_plot._create_error_distribution_chart}}
\pysigstartsignatures
\pysiglinewithargsret
{\sphinxcode{\sphinxupquote{data\_module.make\_export\_plot.}}\sphinxbfcode{\sphinxupquote{\_create\_error\_distribution\_chart}}}
{\sphinxparam{\DUrole{n}{result}}\sphinxparamcomma \sphinxparam{\DUrole{n}{layout\_name}}\sphinxparamcomma \sphinxparam{\DUrole{n}{timestamp}}\sphinxparamcomma \sphinxparam{\DUrole{n}{output\_dir}}}
{}
\pysigstopsignatures
\sphinxAtStartPar
Создает гистограмму распределения ошибок
\begin{quote}\begin{description}
\sphinxlineitem{Параметры}\begin{itemize}
\item {} 
\sphinxAtStartPar
\sphinxstyleliteralstrong{\sphinxupquote{result}} (\sphinxstyleliteralemphasis{\sphinxupquote{Dict}}\sphinxstyleliteralemphasis{\sphinxupquote{{[}}}\sphinxstyleliteralemphasis{\sphinxupquote{str}}\sphinxstyleliteralemphasis{\sphinxupquote{, }}\sphinxstyleliteralemphasis{\sphinxupquote{Any}}\sphinxstyleliteralemphasis{\sphinxupquote{{]}}})

\item {} 
\sphinxAtStartPar
\sphinxstyleliteralstrong{\sphinxupquote{layout\_name}} (\sphinxstyleliteralemphasis{\sphinxupquote{str}})

\item {} 
\sphinxAtStartPar
\sphinxstyleliteralstrong{\sphinxupquote{timestamp}} (\sphinxstyleliteralemphasis{\sphinxupquote{str}})

\item {} 
\sphinxAtStartPar
\sphinxstyleliteralstrong{\sphinxupquote{output\_dir}} (\sphinxstyleliteralemphasis{\sphinxupquote{str}})

\end{itemize}

\sphinxlineitem{Тип результата}
\sphinxAtStartPar
str

\end{description}\end{quote}

\end{fulllineitems}\end{savenotes}

\index{\_create\_metrics\_comparison\_chart() (в модуле data\_module.make\_export\_plot)@\spxentry{\_create\_metrics\_comparison\_chart()}\spxextra{в модуле data\_module.make\_export\_plot}}

\begin{savenotes}\begin{fulllineitems}
\phantomsection\label{\detokenize{data_module:data_module.make_export_plot._create_metrics_comparison_chart}}
\pysigstartsignatures
\pysiglinewithargsret
{\sphinxcode{\sphinxupquote{data\_module.make\_export\_plot.}}\sphinxbfcode{\sphinxupquote{\_create\_metrics\_comparison\_chart}}}
{\sphinxparam{\DUrole{n}{result}}\sphinxparamcomma \sphinxparam{\DUrole{n}{layout\_name}}\sphinxparamcomma \sphinxparam{\DUrole{n}{timestamp}}\sphinxparamcomma \sphinxparam{\DUrole{n}{output\_dir}}}
{}
\pysigstopsignatures
\sphinxAtStartPar
Создает сравнительную диаграмму метрик
\begin{quote}\begin{description}
\sphinxlineitem{Параметры}\begin{itemize}
\item {} 
\sphinxAtStartPar
\sphinxstyleliteralstrong{\sphinxupquote{result}} (\sphinxstyleliteralemphasis{\sphinxupquote{Dict}}\sphinxstyleliteralemphasis{\sphinxupquote{{[}}}\sphinxstyleliteralemphasis{\sphinxupquote{str}}\sphinxstyleliteralemphasis{\sphinxupquote{, }}\sphinxstyleliteralemphasis{\sphinxupquote{Any}}\sphinxstyleliteralemphasis{\sphinxupquote{{]}}})

\item {} 
\sphinxAtStartPar
\sphinxstyleliteralstrong{\sphinxupquote{layout\_name}} (\sphinxstyleliteralemphasis{\sphinxupquote{str}})

\item {} 
\sphinxAtStartPar
\sphinxstyleliteralstrong{\sphinxupquote{timestamp}} (\sphinxstyleliteralemphasis{\sphinxupquote{str}})

\item {} 
\sphinxAtStartPar
\sphinxstyleliteralstrong{\sphinxupquote{output\_dir}} (\sphinxstyleliteralemphasis{\sphinxupquote{str}})

\end{itemize}

\sphinxlineitem{Тип результата}
\sphinxAtStartPar
str

\end{description}\end{quote}

\end{fulllineitems}\end{savenotes}

\index{create\_history\_comparison\_chart() (в модуле data\_module.make\_export\_plot)@\spxentry{create\_history\_comparison\_chart()}\spxextra{в модуле data\_module.make\_export\_plot}}

\begin{savenotes}\begin{fulllineitems}
\phantomsection\label{\detokenize{data_module:data_module.make_export_plot.create_history_comparison_chart}}
\pysigstartsignatures
\pysiglinewithargsret
{\sphinxcode{\sphinxupquote{data\_module.make\_export\_plot.}}\sphinxbfcode{\sphinxupquote{create\_history\_comparison\_chart}}}
{\sphinxparam{\DUrole{n}{layout\_name}}\sphinxparamcomma \sphinxparam{\DUrole{n}{output\_dir}\DUrole{o}{=}\DUrole{default_value}{'reports'}}}
{}
\pysigstopsignatures
\sphinxAtStartPar
Создает график сравнения истории анализов для раскладки
\begin{quote}\begin{description}
\sphinxlineitem{Параметры}\begin{itemize}
\item {} 
\sphinxAtStartPar
\sphinxstyleliteralstrong{\sphinxupquote{layout\_name}} (\sphinxstyleliteralemphasis{\sphinxupquote{str}}) – Название раскладки

\item {} 
\sphinxAtStartPar
\sphinxstyleliteralstrong{\sphinxupquote{output\_dir}} (\sphinxstyleliteralemphasis{\sphinxupquote{str}}) – Директория для сохранения

\end{itemize}

\sphinxlineitem{Результат}
\sphinxAtStartPar
Путь к созданному графику

\sphinxlineitem{Тип результата}
\sphinxAtStartPar
str

\end{description}\end{quote}

\end{fulllineitems}\end{savenotes}

\index{create\_layouts\_comparison\_chart() (в модуле data\_module.make\_export\_plot)@\spxentry{create\_layouts\_comparison\_chart()}\spxextra{в модуле data\_module.make\_export\_plot}}

\begin{savenotes}\begin{fulllineitems}
\phantomsection\label{\detokenize{data_module:data_module.make_export_plot.create_layouts_comparison_chart}}
\pysigstartsignatures
\pysiglinewithargsret
{\sphinxcode{\sphinxupquote{data\_module.make\_export\_plot.}}\sphinxbfcode{\sphinxupquote{create\_layouts\_comparison\_chart}}}
{\sphinxparam{\DUrole{n}{output\_dir}\DUrole{o}{=}\DUrole{default_value}{'reports'}}}
{}
\pysigstopsignatures
\sphinxAtStartPar
Создает сравнительный график для всех раскладок
\begin{quote}\begin{description}
\sphinxlineitem{Параметры}
\sphinxAtStartPar
\sphinxstyleliteralstrong{\sphinxupquote{output\_dir}} (\sphinxstyleliteralemphasis{\sphinxupquote{str}}) – Директория для сохранения

\sphinxlineitem{Результат}
\sphinxAtStartPar
Путь к созданному графику

\sphinxlineitem{Тип результата}
\sphinxAtStartPar
str

\end{description}\end{quote}

\end{fulllineitems}\end{savenotes}


\sphinxstepscope


\subsection{Модуль базы данных (database\_module)}
\label{\detokenize{database_module:database-module}}\label{\detokenize{database_module::doc}}
\sphinxAtStartPar
Модуль для работы с базой данных SQLite, хранения раскладок и результатов анализа.
\index{module@\spxentry{module}!database\_module@\spxentry{database\_module}}\index{database\_module@\spxentry{database\_module}!module@\spxentry{module}}

\subsubsection{Подмодули}
\label{\detokenize{database_module:module-database_module}}\label{\detokenize{database_module:id1}}

\paragraph{database}
\label{\detokenize{database_module:module-database_module.database}}\label{\detokenize{database_module:database}}\index{module@\spxentry{module}!database\_module.database@\spxentry{database\_module.database}}\index{database\_module.database@\spxentry{database\_module.database}!module@\spxentry{module}}\index{take\_lk\_from\_db() (в модуле database\_module.database)@\spxentry{take\_lk\_from\_db()}\spxextra{в модуле database\_module.database}}

\begin{savenotes}\begin{fulllineitems}
\phantomsection\label{\detokenize{database_module:database_module.database.take_lk_from_db}}
\pysigstartsignatures
\pysiglinewithargsret
{\sphinxcode{\sphinxupquote{database\_module.database.}}\sphinxbfcode{\sphinxupquote{take\_lk\_from\_db}}}
{\sphinxparam{\DUrole{n}{name}}}
{}
\pysigstopsignatures
\sphinxAtStartPar
Возвращает словарь правил раскладки, по ее имени
Если такой раскладки нет \sphinxhyphen{} вернет None
\begin{quote}\begin{description}
\sphinxlineitem{Параметры}
\sphinxAtStartPar
\sphinxstyleliteralstrong{\sphinxupquote{name}} (\sphinxstyleliteralemphasis{\sphinxupquote{str}})

\sphinxlineitem{Тип результата}
\sphinxAtStartPar
dict | None

\end{description}\end{quote}

\end{fulllineitems}\end{savenotes}

\index{take\_all\_data\_from\_lk() (в модуле database\_module.database)@\spxentry{take\_all\_data\_from\_lk()}\spxextra{в модуле database\_module.database}}

\begin{savenotes}\begin{fulllineitems}
\phantomsection\label{\detokenize{database_module:database_module.database.take_all_data_from_lk}}
\pysigstartsignatures
\pysiglinewithargsret
{\sphinxcode{\sphinxupquote{database\_module.database.}}\sphinxbfcode{\sphinxupquote{take\_all\_data\_from\_lk}}}
{}
{}
\pysigstopsignatures
\sphinxAtStartPar
Возвращает все содержимое из таблицы lk(с раскладками), включая тестовые
\begin{quote}\begin{description}
\sphinxlineitem{Тип результата}
\sphinxAtStartPar
list

\end{description}\end{quote}

\end{fulllineitems}\end{savenotes}

\index{take\_lk\_names\_from\_lk() (в модуле database\_module.database)@\spxentry{take\_lk\_names\_from\_lk()}\spxextra{в модуле database\_module.database}}

\begin{savenotes}\begin{fulllineitems}
\phantomsection\label{\detokenize{database_module:database_module.database.take_lk_names_from_lk}}
\pysigstartsignatures
\pysiglinewithargsret
{\sphinxcode{\sphinxupquote{database\_module.database.}}\sphinxbfcode{\sphinxupquote{take\_lk\_names\_from\_lk}}}
{}
{}
\pysigstopsignatures
\sphinxAtStartPar
Врозвращает список всех имеющихся в бд раскладок,
кроме тех, в названии которых есть слово test.
\begin{quote}\begin{description}
\sphinxlineitem{Тип результата}
\sphinxAtStartPar
list

\end{description}\end{quote}

\end{fulllineitems}\end{savenotes}

\index{save\_analysis\_result() (в модуле database\_module.database)@\spxentry{save\_analysis\_result()}\spxextra{в модуле database\_module.database}}

\begin{savenotes}\begin{fulllineitems}
\phantomsection\label{\detokenize{database_module:database_module.database.save_analysis_result}}
\pysigstartsignatures
\pysiglinewithargsret
{\sphinxcode{\sphinxupquote{database\_module.database.}}\sphinxbfcode{\sphinxupquote{save\_analysis\_result}}}
{\sphinxparam{\DUrole{n}{layout\_name}}\sphinxparamcomma \sphinxparam{\DUrole{n}{result}}\sphinxparamcomma \sphinxparam{\DUrole{n}{file\_path}}\sphinxparamcomma \sphinxparam{\DUrole{n}{analysis\_type}\DUrole{o}{=}\DUrole{default_value}{'words'}}}
{}
\pysigstopsignatures
\sphinxAtStartPar
Сохраняет результаты анализа в таблицу data
\begin{quote}\begin{description}
\sphinxlineitem{Параметры}\begin{itemize}
\item {} 
\sphinxAtStartPar
\sphinxstyleliteralstrong{\sphinxupquote{layout\_name}} (\sphinxstyleliteralemphasis{\sphinxupquote{str}}) – Название раскладки

\item {} 
\sphinxAtStartPar
\sphinxstyleliteralstrong{\sphinxupquote{result}} (\sphinxstyleliteralemphasis{\sphinxupquote{dict}}) – Результаты анализа

\item {} 
\sphinxAtStartPar
\sphinxstyleliteralstrong{\sphinxupquote{file\_path}} (\sphinxstyleliteralemphasis{\sphinxupquote{str}}) – Путь к анализируемому файлу

\item {} 
\sphinxAtStartPar
\sphinxstyleliteralstrong{\sphinxupquote{analysis\_type}} (\sphinxstyleliteralemphasis{\sphinxupquote{str}}) – Тип анализа („words“ или „text“)

\end{itemize}

\sphinxlineitem{Результат}
\sphinxAtStartPar
ID созданной записи

\sphinxlineitem{Тип результата}
\sphinxAtStartPar
int

\end{description}\end{quote}

\end{fulllineitems}\end{savenotes}

\index{get\_analysis\_history() (в модуле database\_module.database)@\spxentry{get\_analysis\_history()}\spxextra{в модуле database\_module.database}}

\begin{savenotes}\begin{fulllineitems}
\phantomsection\label{\detokenize{database_module:database_module.database.get_analysis_history}}
\pysigstartsignatures
\pysiglinewithargsret
{\sphinxcode{\sphinxupquote{database\_module.database.}}\sphinxbfcode{\sphinxupquote{get\_analysis\_history}}}
{\sphinxparam{\DUrole{n}{layout\_name}\DUrole{o}{=}\DUrole{default_value}{None}}\sphinxparamcomma \sphinxparam{\DUrole{n}{limit}\DUrole{o}{=}\DUrole{default_value}{50}}}
{}
\pysigstopsignatures
\sphinxAtStartPar
Получает историю анализов из базы данных
\begin{quote}\begin{description}
\sphinxlineitem{Параметры}\begin{itemize}
\item {} 
\sphinxAtStartPar
\sphinxstyleliteralstrong{\sphinxupquote{layout\_name}} (\sphinxstyleliteralemphasis{\sphinxupquote{str}}\sphinxstyleliteralemphasis{\sphinxupquote{ | }}\sphinxstyleliteralemphasis{\sphinxupquote{None}}) – Название раскладки (если None, то все раскладки)

\item {} 
\sphinxAtStartPar
\sphinxstyleliteralstrong{\sphinxupquote{limit}} (\sphinxstyleliteralemphasis{\sphinxupquote{int}}) – Максимальное количество записей

\end{itemize}

\sphinxlineitem{Результат}
\sphinxAtStartPar
Список записей с результатами

\sphinxlineitem{Тип результата}
\sphinxAtStartPar
list

\end{description}\end{quote}

\end{fulllineitems}\end{savenotes}

\index{get\_analysis\_statistics() (в модуле database\_module.database)@\spxentry{get\_analysis\_statistics()}\spxextra{в модуле database\_module.database}}

\begin{savenotes}\begin{fulllineitems}
\phantomsection\label{\detokenize{database_module:database_module.database.get_analysis_statistics}}
\pysigstartsignatures
\pysiglinewithargsret
{\sphinxcode{\sphinxupquote{database\_module.database.}}\sphinxbfcode{\sphinxupquote{get\_analysis\_statistics}}}
{\sphinxparam{\DUrole{n}{layout\_name}}}
{}
\pysigstopsignatures
\sphinxAtStartPar
Получает статистику анализов для раскладки
\begin{quote}\begin{description}
\sphinxlineitem{Параметры}
\sphinxAtStartPar
\sphinxstyleliteralstrong{\sphinxupquote{layout\_name}} (\sphinxstyleliteralemphasis{\sphinxupquote{str}}) – Название раскладки

\sphinxlineitem{Результат}
\sphinxAtStartPar
Статистика анализов

\sphinxlineitem{Тип результата}
\sphinxAtStartPar
dict

\end{description}\end{quote}

\end{fulllineitems}\end{savenotes}

\index{delete\_analysis\_result() (в модуле database\_module.database)@\spxentry{delete\_analysis\_result()}\spxextra{в модуле database\_module.database}}

\begin{savenotes}\begin{fulllineitems}
\phantomsection\label{\detokenize{database_module:database_module.database.delete_analysis_result}}
\pysigstartsignatures
\pysiglinewithargsret
{\sphinxcode{\sphinxupquote{database\_module.database.}}\sphinxbfcode{\sphinxupquote{delete\_analysis\_result}}}
{\sphinxparam{\DUrole{n}{record\_id}}}
{}
\pysigstopsignatures
\sphinxAtStartPar
Удаляет результат анализа по ID
\begin{quote}\begin{description}
\sphinxlineitem{Параметры}
\sphinxAtStartPar
\sphinxstyleliteralstrong{\sphinxupquote{record\_id}} (\sphinxstyleliteralemphasis{\sphinxupquote{int}}) – ID записи для удаления

\sphinxlineitem{Результат}
\sphinxAtStartPar
True если запись была удалена

\sphinxlineitem{Тип результата}
\sphinxAtStartPar
bool

\end{description}\end{quote}

\end{fulllineitems}\end{savenotes}



\subparagraph{Функции работы с раскладками}
\label{\detokenize{database_module:id2}}\index{take\_lk\_from\_db() (в модуле database\_module.database)@\spxentry{take\_lk\_from\_db()}\spxextra{в модуле database\_module.database}}

\begin{savenotes}\begin{fulllineitems}
\phantomsection\label{\detokenize{database_module:id0}}
\pysigstartsignatures
\pysiglinewithargsret
{\sphinxcode{\sphinxupquote{database\_module.database.}}\sphinxbfcode{\sphinxupquote{take\_lk\_from\_db}}}
{\sphinxparam{\DUrole{n}{name}}}
{}
\pysigstopsignatures
\sphinxAtStartPar
Возвращает словарь правил раскладки, по ее имени
Если такой раскладки нет \sphinxhyphen{} вернет None
\begin{quote}\begin{description}
\sphinxlineitem{Параметры}
\sphinxAtStartPar
\sphinxstyleliteralstrong{\sphinxupquote{name}} (\sphinxstyleliteralemphasis{\sphinxupquote{str}})

\sphinxlineitem{Тип результата}
\sphinxAtStartPar
dict | None

\end{description}\end{quote}

\end{fulllineitems}\end{savenotes}

\index{take\_all\_data\_from\_lk() (в модуле database\_module.database)@\spxentry{take\_all\_data\_from\_lk()}\spxextra{в модуле database\_module.database}}

\begin{savenotes}\begin{fulllineitems}
\phantomsection\label{\detokenize{database_module:id3}}
\pysigstartsignatures
\pysiglinewithargsret
{\sphinxcode{\sphinxupquote{database\_module.database.}}\sphinxbfcode{\sphinxupquote{take\_all\_data\_from\_lk}}}
{}
{}
\pysigstopsignatures
\sphinxAtStartPar
Возвращает все содержимое из таблицы lk(с раскладками), включая тестовые
\begin{quote}\begin{description}
\sphinxlineitem{Тип результата}
\sphinxAtStartPar
list

\end{description}\end{quote}

\end{fulllineitems}\end{savenotes}

\index{take\_lk\_names\_from\_lk() (в модуле database\_module.database)@\spxentry{take\_lk\_names\_from\_lk()}\spxextra{в модуле database\_module.database}}

\begin{savenotes}\begin{fulllineitems}
\phantomsection\label{\detokenize{database_module:id4}}
\pysigstartsignatures
\pysiglinewithargsret
{\sphinxcode{\sphinxupquote{database\_module.database.}}\sphinxbfcode{\sphinxupquote{take\_lk\_names\_from\_lk}}}
{}
{}
\pysigstopsignatures
\sphinxAtStartPar
Врозвращает список всех имеющихся в бд раскладок,
кроме тех, в названии которых есть слово test.
\begin{quote}\begin{description}
\sphinxlineitem{Тип результата}
\sphinxAtStartPar
list

\end{description}\end{quote}

\end{fulllineitems}\end{savenotes}



\subparagraph{Функции работы с результатами анализа}
\label{\detokenize{database_module:id5}}\index{save\_analysis\_result() (в модуле database\_module.database)@\spxentry{save\_analysis\_result()}\spxextra{в модуле database\_module.database}}

\begin{savenotes}\begin{fulllineitems}
\phantomsection\label{\detokenize{database_module:id6}}
\pysigstartsignatures
\pysiglinewithargsret
{\sphinxcode{\sphinxupquote{database\_module.database.}}\sphinxbfcode{\sphinxupquote{save\_analysis\_result}}}
{\sphinxparam{\DUrole{n}{layout\_name}}\sphinxparamcomma \sphinxparam{\DUrole{n}{result}}\sphinxparamcomma \sphinxparam{\DUrole{n}{file\_path}}\sphinxparamcomma \sphinxparam{\DUrole{n}{analysis\_type}\DUrole{o}{=}\DUrole{default_value}{'words'}}}
{}
\pysigstopsignatures
\sphinxAtStartPar
Сохраняет результаты анализа в таблицу data
\begin{quote}\begin{description}
\sphinxlineitem{Параметры}\begin{itemize}
\item {} 
\sphinxAtStartPar
\sphinxstyleliteralstrong{\sphinxupquote{layout\_name}} (\sphinxstyleliteralemphasis{\sphinxupquote{str}}) – Название раскладки

\item {} 
\sphinxAtStartPar
\sphinxstyleliteralstrong{\sphinxupquote{result}} (\sphinxstyleliteralemphasis{\sphinxupquote{dict}}) – Результаты анализа

\item {} 
\sphinxAtStartPar
\sphinxstyleliteralstrong{\sphinxupquote{file\_path}} (\sphinxstyleliteralemphasis{\sphinxupquote{str}}) – Путь к анализируемому файлу

\item {} 
\sphinxAtStartPar
\sphinxstyleliteralstrong{\sphinxupquote{analysis\_type}} (\sphinxstyleliteralemphasis{\sphinxupquote{str}}) – Тип анализа („words“ или „text“)

\end{itemize}

\sphinxlineitem{Результат}
\sphinxAtStartPar
ID созданной записи

\sphinxlineitem{Тип результата}
\sphinxAtStartPar
int

\end{description}\end{quote}

\end{fulllineitems}\end{savenotes}

\index{get\_analysis\_history() (в модуле database\_module.database)@\spxentry{get\_analysis\_history()}\spxextra{в модуле database\_module.database}}

\begin{savenotes}\begin{fulllineitems}
\phantomsection\label{\detokenize{database_module:id7}}
\pysigstartsignatures
\pysiglinewithargsret
{\sphinxcode{\sphinxupquote{database\_module.database.}}\sphinxbfcode{\sphinxupquote{get\_analysis\_history}}}
{\sphinxparam{\DUrole{n}{layout\_name}\DUrole{o}{=}\DUrole{default_value}{None}}\sphinxparamcomma \sphinxparam{\DUrole{n}{limit}\DUrole{o}{=}\DUrole{default_value}{50}}}
{}
\pysigstopsignatures
\sphinxAtStartPar
Получает историю анализов из базы данных
\begin{quote}\begin{description}
\sphinxlineitem{Параметры}\begin{itemize}
\item {} 
\sphinxAtStartPar
\sphinxstyleliteralstrong{\sphinxupquote{layout\_name}} (\sphinxstyleliteralemphasis{\sphinxupquote{str}}\sphinxstyleliteralemphasis{\sphinxupquote{ | }}\sphinxstyleliteralemphasis{\sphinxupquote{None}}) – Название раскладки (если None, то все раскладки)

\item {} 
\sphinxAtStartPar
\sphinxstyleliteralstrong{\sphinxupquote{limit}} (\sphinxstyleliteralemphasis{\sphinxupquote{int}}) – Максимальное количество записей

\end{itemize}

\sphinxlineitem{Результат}
\sphinxAtStartPar
Список записей с результатами

\sphinxlineitem{Тип результата}
\sphinxAtStartPar
list

\end{description}\end{quote}

\end{fulllineitems}\end{savenotes}

\index{get\_analysis\_statistics() (в модуле database\_module.database)@\spxentry{get\_analysis\_statistics()}\spxextra{в модуле database\_module.database}}

\begin{savenotes}\begin{fulllineitems}
\phantomsection\label{\detokenize{database_module:id8}}
\pysigstartsignatures
\pysiglinewithargsret
{\sphinxcode{\sphinxupquote{database\_module.database.}}\sphinxbfcode{\sphinxupquote{get\_analysis\_statistics}}}
{\sphinxparam{\DUrole{n}{layout\_name}}}
{}
\pysigstopsignatures
\sphinxAtStartPar
Получает статистику анализов для раскладки
\begin{quote}\begin{description}
\sphinxlineitem{Параметры}
\sphinxAtStartPar
\sphinxstyleliteralstrong{\sphinxupquote{layout\_name}} (\sphinxstyleliteralemphasis{\sphinxupquote{str}}) – Название раскладки

\sphinxlineitem{Результат}
\sphinxAtStartPar
Статистика анализов

\sphinxlineitem{Тип результата}
\sphinxAtStartPar
dict

\end{description}\end{quote}

\end{fulllineitems}\end{savenotes}

\index{delete\_analysis\_result() (в модуле database\_module.database)@\spxentry{delete\_analysis\_result()}\spxextra{в модуле database\_module.database}}

\begin{savenotes}\begin{fulllineitems}
\phantomsection\label{\detokenize{database_module:id9}}
\pysigstartsignatures
\pysiglinewithargsret
{\sphinxcode{\sphinxupquote{database\_module.database.}}\sphinxbfcode{\sphinxupquote{delete\_analysis\_result}}}
{\sphinxparam{\DUrole{n}{record\_id}}}
{}
\pysigstopsignatures
\sphinxAtStartPar
Удаляет результат анализа по ID
\begin{quote}\begin{description}
\sphinxlineitem{Параметры}
\sphinxAtStartPar
\sphinxstyleliteralstrong{\sphinxupquote{record\_id}} (\sphinxstyleliteralemphasis{\sphinxupquote{int}}) – ID записи для удаления

\sphinxlineitem{Результат}
\sphinxAtStartPar
True если запись была удалена

\sphinxlineitem{Тип результата}
\sphinxAtStartPar
bool

\end{description}\end{quote}

\end{fulllineitems}\end{savenotes}



\paragraph{db\_init}
\label{\detokenize{database_module:db-init}}
\sphinxstepscope


\subsection{Модуль обработки (processing\_module)}
\label{\detokenize{processing_module:processing-module}}\label{\detokenize{processing_module::doc}}
\sphinxAtStartPar
Модуль для анализа текстов и подсчета ошибок на основе правил раскладки.
\index{module@\spxentry{module}!processing\_module@\spxentry{processing\_module}}\index{processing\_module@\spxentry{processing\_module}!module@\spxentry{module}}

\subsubsection{Подмодули}
\label{\detokenize{processing_module:module-processing_module}}\label{\detokenize{processing_module:id1}}

\paragraph{calculate\_data}
\label{\detokenize{processing_module:module-processing_module.calculate_data}}\label{\detokenize{processing_module:calculate-data}}\index{module@\spxentry{module}!processing\_module.calculate\_data@\spxentry{processing\_module.calculate\_data}}\index{processing\_module.calculate\_data@\spxentry{processing\_module.calculate\_data}!module@\spxentry{module}}\index{make\_processing() (в модуле processing\_module.calculate\_data)@\spxentry{make\_processing()}\spxextra{в модуле processing\_module.calculate\_data}}

\begin{savenotes}\begin{fulllineitems}
\phantomsection\label{\detokenize{processing_module:processing_module.calculate_data.make_processing}}
\pysigstartsignatures
\pysiglinewithargsret
{\sphinxcode{\sphinxupquote{processing\_module.calculate\_data.}}\sphinxbfcode{\sphinxupquote{make\_processing}}}
{\sphinxparam{\DUrole{n}{wordlist}}\sphinxparamcomma \sphinxparam{\DUrole{n}{rules}}}
{}
\pysigstopsignatures
\sphinxAtStartPar
Считает количество ошибок по словарю правил и списку слов.
ВНИМАНИЕ: Используйте только для небольших списков!
\begin{quote}\begin{description}
\sphinxlineitem{Результат}
\sphinxAtStartPar
Словарь с результатами анализа

\sphinxlineitem{Тип результата}
\sphinxAtStartPar
dict

\sphinxlineitem{Параметры}\begin{itemize}
\item {} 
\sphinxAtStartPar
\sphinxstyleliteralstrong{\sphinxupquote{wordlist}} (\sphinxstyleliteralemphasis{\sphinxupquote{list}})

\item {} 
\sphinxAtStartPar
\sphinxstyleliteralstrong{\sphinxupquote{rules}} (\sphinxstyleliteralemphasis{\sphinxupquote{dict}})

\end{itemize}

\end{description}\end{quote}

\end{fulllineitems}\end{savenotes}

\index{make\_processing\_stream() (в модуле processing\_module.calculate\_data)@\spxentry{make\_processing\_stream()}\spxextra{в модуле processing\_module.calculate\_data}}

\begin{savenotes}\begin{fulllineitems}
\phantomsection\label{\detokenize{processing_module:processing_module.calculate_data.make_processing_stream}}
\pysigstartsignatures
\pysiglinewithargsret
{\sphinxcode{\sphinxupquote{processing\_module.calculate\_data.}}\sphinxbfcode{\sphinxupquote{make\_processing\_stream}}}
{\sphinxparam{\DUrole{n}{wordlist\_generator}}\sphinxparamcomma \sphinxparam{\DUrole{n}{rules}}\sphinxparamcomma \sphinxparam{\DUrole{n}{total\_words}\DUrole{o}{=}\DUrole{default_value}{None}}}
{}
\pysigstopsignatures
\sphinxAtStartPar
Обрабатывает большие файлы батчами с прогресс\sphinxhyphen{}баром.
\begin{quote}\begin{description}
\sphinxlineitem{Параметры}\begin{itemize}
\item {} 
\sphinxAtStartPar
\sphinxstyleliteralstrong{\sphinxupquote{wordlist\_generator}} (\sphinxstyleliteralemphasis{\sphinxupquote{Generator}}\sphinxstyleliteralemphasis{\sphinxupquote{{[}}}\sphinxstyleliteralemphasis{\sphinxupquote{List}}\sphinxstyleliteralemphasis{\sphinxupquote{{[}}}\sphinxstyleliteralemphasis{\sphinxupquote{str}}\sphinxstyleliteralemphasis{\sphinxupquote{{]}}}\sphinxstyleliteralemphasis{\sphinxupquote{, }}\sphinxstyleliteralemphasis{\sphinxupquote{None}}\sphinxstyleliteralemphasis{\sphinxupquote{, }}\sphinxstyleliteralemphasis{\sphinxupquote{None}}\sphinxstyleliteralemphasis{\sphinxupquote{{]}}}) – Генератор батчей слов

\item {} 
\sphinxAtStartPar
\sphinxstyleliteralstrong{\sphinxupquote{rules}} (\sphinxstyleliteralemphasis{\sphinxupquote{Dict}}\sphinxstyleliteralemphasis{\sphinxupquote{{[}}}\sphinxstyleliteralemphasis{\sphinxupquote{str}}\sphinxstyleliteralemphasis{\sphinxupquote{, }}\sphinxstyleliteralemphasis{\sphinxupquote{int}}\sphinxstyleliteralemphasis{\sphinxupquote{{]}}}) – Словарь правил для подсчета ошибок

\item {} 
\sphinxAtStartPar
\sphinxstyleliteralstrong{\sphinxupquote{total\_words}} (\sphinxstyleliteralemphasis{\sphinxupquote{int}}\sphinxstyleliteralemphasis{\sphinxupquote{ | }}\sphinxstyleliteralemphasis{\sphinxupquote{None}}) – Общее количество слов для прогресс\sphinxhyphen{}бара (опционально)

\end{itemize}

\sphinxlineitem{Результат}
\sphinxAtStartPar
Словарь с результатами анализа

\sphinxlineitem{Тип результата}
\sphinxAtStartPar
dict

\end{description}\end{quote}

\end{fulllineitems}\end{savenotes}

\index{validate\_rules() (в модуле processing\_module.calculate\_data)@\spxentry{validate\_rules()}\spxextra{в модуле processing\_module.calculate\_data}}

\begin{savenotes}\begin{fulllineitems}
\phantomsection\label{\detokenize{processing_module:processing_module.calculate_data.validate_rules}}
\pysigstartsignatures
\pysiglinewithargsret
{\sphinxcode{\sphinxupquote{processing\_module.calculate\_data.}}\sphinxbfcode{\sphinxupquote{validate\_rules}}}
{\sphinxparam{\DUrole{n}{rules}}}
{}
\pysigstopsignatures
\sphinxAtStartPar
Проверяет корректность словаря правил.
\begin{quote}\begin{description}
\sphinxlineitem{Параметры}
\sphinxAtStartPar
\sphinxstyleliteralstrong{\sphinxupquote{rules}} (\sphinxstyleliteralemphasis{\sphinxupquote{Dict}}\sphinxstyleliteralemphasis{\sphinxupquote{{[}}}\sphinxstyleliteralemphasis{\sphinxupquote{str}}\sphinxstyleliteralemphasis{\sphinxupquote{, }}\sphinxstyleliteralemphasis{\sphinxupquote{int}}\sphinxstyleliteralemphasis{\sphinxupquote{ | }}\sphinxstyleliteralemphasis{\sphinxupquote{float}}\sphinxstyleliteralemphasis{\sphinxupquote{{]}}}) – Словарь правил для проверки

\sphinxlineitem{Результат}
\sphinxAtStartPar
True если правила корректны

\sphinxlineitem{Исключение}
\sphinxAtStartPar
\sphinxstyleliteralstrong{\sphinxupquote{ValueError}} – Если правила некорректны

\sphinxlineitem{Тип результата}
\sphinxAtStartPar
bool

\end{description}\end{quote}

\end{fulllineitems}\end{savenotes}

\index{make\_text\_processing() (в модуле processing\_module.calculate\_data)@\spxentry{make\_text\_processing()}\spxextra{в модуле processing\_module.calculate\_data}}

\begin{savenotes}\begin{fulllineitems}
\phantomsection\label{\detokenize{processing_module:processing_module.calculate_data.make_text_processing}}
\pysigstartsignatures
\pysiglinewithargsret
{\sphinxcode{\sphinxupquote{processing\_module.calculate\_data.}}\sphinxbfcode{\sphinxupquote{make\_text\_processing}}}
{\sphinxparam{\DUrole{n}{text}}\sphinxparamcomma \sphinxparam{\DUrole{n}{rules}}}
{}
\pysigstopsignatures
\sphinxAtStartPar
Считает количество ошибок по словарю правил для сплошного текста.
ВНИМАНИЕ: Используйте только для небольших текстов!
\begin{quote}\begin{description}
\sphinxlineitem{Результат}
\sphinxAtStartPar
Словарь с результатами анализа

\sphinxlineitem{Тип результата}
\sphinxAtStartPar
dict

\sphinxlineitem{Параметры}\begin{itemize}
\item {} 
\sphinxAtStartPar
\sphinxstyleliteralstrong{\sphinxupquote{text}} (\sphinxstyleliteralemphasis{\sphinxupquote{str}})

\item {} 
\sphinxAtStartPar
\sphinxstyleliteralstrong{\sphinxupquote{rules}} (\sphinxstyleliteralemphasis{\sphinxupquote{dict}})

\end{itemize}

\end{description}\end{quote}

\end{fulllineitems}\end{savenotes}

\index{make\_text\_processing\_stream() (в модуле processing\_module.calculate\_data)@\spxentry{make\_text\_processing\_stream()}\spxextra{в модуле processing\_module.calculate\_data}}

\begin{savenotes}\begin{fulllineitems}
\phantomsection\label{\detokenize{processing_module:processing_module.calculate_data.make_text_processing_stream}}
\pysigstartsignatures
\pysiglinewithargsret
{\sphinxcode{\sphinxupquote{processing\_module.calculate\_data.}}\sphinxbfcode{\sphinxupquote{make\_text\_processing\_stream}}}
{\sphinxparam{\DUrole{n}{text\_generator}}\sphinxparamcomma \sphinxparam{\DUrole{n}{rules}}\sphinxparamcomma \sphinxparam{\DUrole{n}{total\_chars}\DUrole{o}{=}\DUrole{default_value}{None}}}
{}
\pysigstopsignatures
\sphinxAtStartPar
Обрабатывает большие текстовые файлы чанками с прогресс\sphinxhyphen{}баром.
\begin{quote}\begin{description}
\sphinxlineitem{Параметры}\begin{itemize}
\item {} 
\sphinxAtStartPar
\sphinxstyleliteralstrong{\sphinxupquote{text\_generator}} (\sphinxstyleliteralemphasis{\sphinxupquote{Generator}}\sphinxstyleliteralemphasis{\sphinxupquote{{[}}}\sphinxstyleliteralemphasis{\sphinxupquote{str}}\sphinxstyleliteralemphasis{\sphinxupquote{, }}\sphinxstyleliteralemphasis{\sphinxupquote{None}}\sphinxstyleliteralemphasis{\sphinxupquote{, }}\sphinxstyleliteralemphasis{\sphinxupquote{None}}\sphinxstyleliteralemphasis{\sphinxupquote{{]}}}) – Генератор чанков текста

\item {} 
\sphinxAtStartPar
\sphinxstyleliteralstrong{\sphinxupquote{rules}} (\sphinxstyleliteralemphasis{\sphinxupquote{Dict}}\sphinxstyleliteralemphasis{\sphinxupquote{{[}}}\sphinxstyleliteralemphasis{\sphinxupquote{str}}\sphinxstyleliteralemphasis{\sphinxupquote{, }}\sphinxstyleliteralemphasis{\sphinxupquote{int}}\sphinxstyleliteralemphasis{\sphinxupquote{{]}}}) – Словарь правил для подсчета ошибок

\item {} 
\sphinxAtStartPar
\sphinxstyleliteralstrong{\sphinxupquote{total\_chars}} (\sphinxstyleliteralemphasis{\sphinxupquote{int}}\sphinxstyleliteralemphasis{\sphinxupquote{ | }}\sphinxstyleliteralemphasis{\sphinxupquote{None}}) – Общее количество символов для прогресс\sphinxhyphen{}бара (опционально)

\end{itemize}

\sphinxlineitem{Результат}
\sphinxAtStartPar
Словарь с результатами анализа

\sphinxlineitem{Тип результата}
\sphinxAtStartPar
dict

\end{description}\end{quote}

\end{fulllineitems}\end{savenotes}



\subparagraph{Функции обработки слов}
\label{\detokenize{processing_module:id2}}\index{make\_processing() (в модуле processing\_module.calculate\_data)@\spxentry{make\_processing()}\spxextra{в модуле processing\_module.calculate\_data}}

\begin{savenotes}\begin{fulllineitems}
\phantomsection\label{\detokenize{processing_module:id0}}
\pysigstartsignatures
\pysiglinewithargsret
{\sphinxcode{\sphinxupquote{processing\_module.calculate\_data.}}\sphinxbfcode{\sphinxupquote{make\_processing}}}
{\sphinxparam{\DUrole{n}{wordlist}}\sphinxparamcomma \sphinxparam{\DUrole{n}{rules}}}
{}
\pysigstopsignatures
\sphinxAtStartPar
Считает количество ошибок по словарю правил и списку слов.
ВНИМАНИЕ: Используйте только для небольших списков!
\begin{quote}\begin{description}
\sphinxlineitem{Результат}
\sphinxAtStartPar
Словарь с результатами анализа

\sphinxlineitem{Тип результата}
\sphinxAtStartPar
dict

\sphinxlineitem{Параметры}\begin{itemize}
\item {} 
\sphinxAtStartPar
\sphinxstyleliteralstrong{\sphinxupquote{wordlist}} (\sphinxstyleliteralemphasis{\sphinxupquote{list}})

\item {} 
\sphinxAtStartPar
\sphinxstyleliteralstrong{\sphinxupquote{rules}} (\sphinxstyleliteralemphasis{\sphinxupquote{dict}})

\end{itemize}

\end{description}\end{quote}

\end{fulllineitems}\end{savenotes}

\index{make\_processing\_stream() (в модуле processing\_module.calculate\_data)@\spxentry{make\_processing\_stream()}\spxextra{в модуле processing\_module.calculate\_data}}

\begin{savenotes}\begin{fulllineitems}
\phantomsection\label{\detokenize{processing_module:id3}}
\pysigstartsignatures
\pysiglinewithargsret
{\sphinxcode{\sphinxupquote{processing\_module.calculate\_data.}}\sphinxbfcode{\sphinxupquote{make\_processing\_stream}}}
{\sphinxparam{\DUrole{n}{wordlist\_generator}}\sphinxparamcomma \sphinxparam{\DUrole{n}{rules}}\sphinxparamcomma \sphinxparam{\DUrole{n}{total\_words}\DUrole{o}{=}\DUrole{default_value}{None}}}
{}
\pysigstopsignatures
\sphinxAtStartPar
Обрабатывает большие файлы батчами с прогресс\sphinxhyphen{}баром.
\begin{quote}\begin{description}
\sphinxlineitem{Параметры}\begin{itemize}
\item {} 
\sphinxAtStartPar
\sphinxstyleliteralstrong{\sphinxupquote{wordlist\_generator}} (\sphinxstyleliteralemphasis{\sphinxupquote{Generator}}\sphinxstyleliteralemphasis{\sphinxupquote{{[}}}\sphinxstyleliteralemphasis{\sphinxupquote{List}}\sphinxstyleliteralemphasis{\sphinxupquote{{[}}}\sphinxstyleliteralemphasis{\sphinxupquote{str}}\sphinxstyleliteralemphasis{\sphinxupquote{{]}}}\sphinxstyleliteralemphasis{\sphinxupquote{, }}\sphinxstyleliteralemphasis{\sphinxupquote{None}}\sphinxstyleliteralemphasis{\sphinxupquote{, }}\sphinxstyleliteralemphasis{\sphinxupquote{None}}\sphinxstyleliteralemphasis{\sphinxupquote{{]}}}) – Генератор батчей слов

\item {} 
\sphinxAtStartPar
\sphinxstyleliteralstrong{\sphinxupquote{rules}} (\sphinxstyleliteralemphasis{\sphinxupquote{Dict}}\sphinxstyleliteralemphasis{\sphinxupquote{{[}}}\sphinxstyleliteralemphasis{\sphinxupquote{str}}\sphinxstyleliteralemphasis{\sphinxupquote{, }}\sphinxstyleliteralemphasis{\sphinxupquote{int}}\sphinxstyleliteralemphasis{\sphinxupquote{{]}}}) – Словарь правил для подсчета ошибок

\item {} 
\sphinxAtStartPar
\sphinxstyleliteralstrong{\sphinxupquote{total\_words}} (\sphinxstyleliteralemphasis{\sphinxupquote{int}}\sphinxstyleliteralemphasis{\sphinxupquote{ | }}\sphinxstyleliteralemphasis{\sphinxupquote{None}}) – Общее количество слов для прогресс\sphinxhyphen{}бара (опционально)

\end{itemize}

\sphinxlineitem{Результат}
\sphinxAtStartPar
Словарь с результатами анализа

\sphinxlineitem{Тип результата}
\sphinxAtStartPar
dict

\end{description}\end{quote}

\end{fulllineitems}\end{savenotes}



\subparagraph{Функции обработки текста}
\label{\detokenize{processing_module:id4}}\index{make\_text\_processing() (в модуле processing\_module.calculate\_data)@\spxentry{make\_text\_processing()}\spxextra{в модуле processing\_module.calculate\_data}}

\begin{savenotes}\begin{fulllineitems}
\phantomsection\label{\detokenize{processing_module:id5}}
\pysigstartsignatures
\pysiglinewithargsret
{\sphinxcode{\sphinxupquote{processing\_module.calculate\_data.}}\sphinxbfcode{\sphinxupquote{make\_text\_processing}}}
{\sphinxparam{\DUrole{n}{text}}\sphinxparamcomma \sphinxparam{\DUrole{n}{rules}}}
{}
\pysigstopsignatures
\sphinxAtStartPar
Считает количество ошибок по словарю правил для сплошного текста.
ВНИМАНИЕ: Используйте только для небольших текстов!
\begin{quote}\begin{description}
\sphinxlineitem{Результат}
\sphinxAtStartPar
Словарь с результатами анализа

\sphinxlineitem{Тип результата}
\sphinxAtStartPar
dict

\sphinxlineitem{Параметры}\begin{itemize}
\item {} 
\sphinxAtStartPar
\sphinxstyleliteralstrong{\sphinxupquote{text}} (\sphinxstyleliteralemphasis{\sphinxupquote{str}})

\item {} 
\sphinxAtStartPar
\sphinxstyleliteralstrong{\sphinxupquote{rules}} (\sphinxstyleliteralemphasis{\sphinxupquote{dict}})

\end{itemize}

\end{description}\end{quote}

\end{fulllineitems}\end{savenotes}

\index{make\_text\_processing\_stream() (в модуле processing\_module.calculate\_data)@\spxentry{make\_text\_processing\_stream()}\spxextra{в модуле processing\_module.calculate\_data}}

\begin{savenotes}\begin{fulllineitems}
\phantomsection\label{\detokenize{processing_module:id6}}
\pysigstartsignatures
\pysiglinewithargsret
{\sphinxcode{\sphinxupquote{processing\_module.calculate\_data.}}\sphinxbfcode{\sphinxupquote{make\_text\_processing\_stream}}}
{\sphinxparam{\DUrole{n}{text\_generator}}\sphinxparamcomma \sphinxparam{\DUrole{n}{rules}}\sphinxparamcomma \sphinxparam{\DUrole{n}{total\_chars}\DUrole{o}{=}\DUrole{default_value}{None}}}
{}
\pysigstopsignatures
\sphinxAtStartPar
Обрабатывает большие текстовые файлы чанками с прогресс\sphinxhyphen{}баром.
\begin{quote}\begin{description}
\sphinxlineitem{Параметры}\begin{itemize}
\item {} 
\sphinxAtStartPar
\sphinxstyleliteralstrong{\sphinxupquote{text\_generator}} (\sphinxstyleliteralemphasis{\sphinxupquote{Generator}}\sphinxstyleliteralemphasis{\sphinxupquote{{[}}}\sphinxstyleliteralemphasis{\sphinxupquote{str}}\sphinxstyleliteralemphasis{\sphinxupquote{, }}\sphinxstyleliteralemphasis{\sphinxupquote{None}}\sphinxstyleliteralemphasis{\sphinxupquote{, }}\sphinxstyleliteralemphasis{\sphinxupquote{None}}\sphinxstyleliteralemphasis{\sphinxupquote{{]}}}) – Генератор чанков текста

\item {} 
\sphinxAtStartPar
\sphinxstyleliteralstrong{\sphinxupquote{rules}} (\sphinxstyleliteralemphasis{\sphinxupquote{Dict}}\sphinxstyleliteralemphasis{\sphinxupquote{{[}}}\sphinxstyleliteralemphasis{\sphinxupquote{str}}\sphinxstyleliteralemphasis{\sphinxupquote{, }}\sphinxstyleliteralemphasis{\sphinxupquote{int}}\sphinxstyleliteralemphasis{\sphinxupquote{{]}}}) – Словарь правил для подсчета ошибок

\item {} 
\sphinxAtStartPar
\sphinxstyleliteralstrong{\sphinxupquote{total\_chars}} (\sphinxstyleliteralemphasis{\sphinxupquote{int}}\sphinxstyleliteralemphasis{\sphinxupquote{ | }}\sphinxstyleliteralemphasis{\sphinxupquote{None}}) – Общее количество символов для прогресс\sphinxhyphen{}бара (опционально)

\end{itemize}

\sphinxlineitem{Результат}
\sphinxAtStartPar
Словарь с результатами анализа

\sphinxlineitem{Тип результата}
\sphinxAtStartPar
dict

\end{description}\end{quote}

\end{fulllineitems}\end{savenotes}



\subparagraph{Вспомогательные функции}
\label{\detokenize{processing_module:id7}}\index{validate\_rules() (в модуле processing\_module.calculate\_data)@\spxentry{validate\_rules()}\spxextra{в модуле processing\_module.calculate\_data}}

\begin{savenotes}\begin{fulllineitems}
\phantomsection\label{\detokenize{processing_module:id8}}
\pysigstartsignatures
\pysiglinewithargsret
{\sphinxcode{\sphinxupquote{processing\_module.calculate\_data.}}\sphinxbfcode{\sphinxupquote{validate\_rules}}}
{\sphinxparam{\DUrole{n}{rules}}}
{}
\pysigstopsignatures
\sphinxAtStartPar
Проверяет корректность словаря правил.
\begin{quote}\begin{description}
\sphinxlineitem{Параметры}
\sphinxAtStartPar
\sphinxstyleliteralstrong{\sphinxupquote{rules}} (\sphinxstyleliteralemphasis{\sphinxupquote{Dict}}\sphinxstyleliteralemphasis{\sphinxupquote{{[}}}\sphinxstyleliteralemphasis{\sphinxupquote{str}}\sphinxstyleliteralemphasis{\sphinxupquote{, }}\sphinxstyleliteralemphasis{\sphinxupquote{int}}\sphinxstyleliteralemphasis{\sphinxupquote{ | }}\sphinxstyleliteralemphasis{\sphinxupquote{float}}\sphinxstyleliteralemphasis{\sphinxupquote{{]}}}) – Словарь правил для проверки

\sphinxlineitem{Результат}
\sphinxAtStartPar
True если правила корректны

\sphinxlineitem{Исключение}
\sphinxAtStartPar
\sphinxstyleliteralstrong{\sphinxupquote{ValueError}} – Если правила некорректны

\sphinxlineitem{Тип результата}
\sphinxAtStartPar
bool

\end{description}\end{quote}

\end{fulllineitems}\end{savenotes}


\sphinxstepscope


\subsection{Модуль сканирования (scan\_module)}
\label{\detokenize{scan_module:scan-module}}\label{\detokenize{scan_module::doc}}
\sphinxAtStartPar
Модуль для чтения файлов и раскладок клавиатуры.
\index{module@\spxentry{module}!scan\_module@\spxentry{scan\_module}}\index{scan\_module@\spxentry{scan\_module}!module@\spxentry{module}}

\subsubsection{Подмодули}
\label{\detokenize{scan_module:module-scan_module}}\label{\detokenize{scan_module:id1}}

\paragraph{read\_files}
\label{\detokenize{scan_module:module-scan_module.read_files}}\label{\detokenize{scan_module:read-files}}\index{module@\spxentry{module}!scan\_module.read\_files@\spxentry{scan\_module.read\_files}}\index{scan\_module.read\_files@\spxentry{scan\_module.read\_files}!module@\spxentry{module}}\index{get\_file\_size\_mb() (в модуле scan\_module.read\_files)@\spxentry{get\_file\_size\_mb()}\spxextra{в модуле scan\_module.read\_files}}

\begin{savenotes}\begin{fulllineitems}
\phantomsection\label{\detokenize{scan_module:scan_module.read_files.get_file_size_mb}}
\pysigstartsignatures
\pysiglinewithargsret
{\sphinxcode{\sphinxupquote{scan\_module.read\_files.}}\sphinxbfcode{\sphinxupquote{get\_file\_size\_mb}}}
{\sphinxparam{\DUrole{n}{filename}}}
{}
\pysigstopsignatures
\sphinxAtStartPar
Возвращает размер файла в мегабайтах
\begin{quote}\begin{description}
\sphinxlineitem{Параметры}
\sphinxAtStartPar
\sphinxstyleliteralstrong{\sphinxupquote{filename}} (\sphinxstyleliteralemphasis{\sphinxupquote{str}})

\sphinxlineitem{Тип результата}
\sphinxAtStartPar
float

\end{description}\end{quote}

\end{fulllineitems}\end{savenotes}

\index{get\_words\_from\_file() (в модуле scan\_module.read\_files)@\spxentry{get\_words\_from\_file()}\spxextra{в модуле scan\_module.read\_files}}

\begin{savenotes}\begin{fulllineitems}
\phantomsection\label{\detokenize{scan_module:scan_module.read_files.get_words_from_file}}
\pysigstartsignatures
\pysiglinewithargsret
{\sphinxcode{\sphinxupquote{scan\_module.read\_files.}}\sphinxbfcode{\sphinxupquote{get\_words\_from\_file}}}
{\sphinxparam{\DUrole{n}{filename}}}
{}
\pysigstopsignatures
\sphinxAtStartPar
Считывает построчно из файла слова и преобразует
их в список для удобной обработки.
ВНИМАНИЕ: Используйте только для небольших файлов!
\begin{quote}\begin{description}
\sphinxlineitem{Параметры}
\sphinxAtStartPar
\sphinxstyleliteralstrong{\sphinxupquote{filename}} (\sphinxstyleliteralemphasis{\sphinxupquote{str}})

\sphinxlineitem{Тип результата}
\sphinxAtStartPar
list

\end{description}\end{quote}

\end{fulllineitems}\end{savenotes}

\index{get\_words\_from\_file\_stream() (в модуле scan\_module.read\_files)@\spxentry{get\_words\_from\_file\_stream()}\spxextra{в модуле scan\_module.read\_files}}

\begin{savenotes}\begin{fulllineitems}
\phantomsection\label{\detokenize{scan_module:scan_module.read_files.get_words_from_file_stream}}
\pysigstartsignatures
\pysiglinewithargsret
{\sphinxcode{\sphinxupquote{scan\_module.read\_files.}}\sphinxbfcode{\sphinxupquote{get\_words\_from\_file\_stream}}}
{\sphinxparam{\DUrole{n}{filename}}\sphinxparamcomma \sphinxparam{\DUrole{n}{batch\_size}\DUrole{o}{=}\DUrole{default_value}{1000}}}
{}
\pysigstopsignatures
\sphinxAtStartPar
Генератор для потоковой обработки больших файлов.
Возвращает батчи слов для эффективной обработки.
\begin{quote}\begin{description}
\sphinxlineitem{Параметры}\begin{itemize}
\item {} 
\sphinxAtStartPar
\sphinxstyleliteralstrong{\sphinxupquote{filename}} (\sphinxstyleliteralemphasis{\sphinxupquote{str}})

\item {} 
\sphinxAtStartPar
\sphinxstyleliteralstrong{\sphinxupquote{batch\_size}} (\sphinxstyleliteralemphasis{\sphinxupquote{int}})

\end{itemize}

\sphinxlineitem{Тип результата}
\sphinxAtStartPar
\sphinxstyleemphasis{Generator}{[}\sphinxstyleemphasis{List}{[}str{]}, None, None{]}

\end{description}\end{quote}

\end{fulllineitems}\end{savenotes}

\index{count\_lines\_in\_file() (в модуле scan\_module.read\_files)@\spxentry{count\_lines\_in\_file()}\spxextra{в модуле scan\_module.read\_files}}

\begin{savenotes}\begin{fulllineitems}
\phantomsection\label{\detokenize{scan_module:scan_module.read_files.count_lines_in_file}}
\pysigstartsignatures
\pysiglinewithargsret
{\sphinxcode{\sphinxupquote{scan\_module.read\_files.}}\sphinxbfcode{\sphinxupquote{count\_lines\_in\_file}}}
{\sphinxparam{\DUrole{n}{filename}}}
{}
\pysigstopsignatures
\sphinxAtStartPar
Подсчитывает количество непустых строк в файле
\begin{quote}\begin{description}
\sphinxlineitem{Параметры}
\sphinxAtStartPar
\sphinxstyleliteralstrong{\sphinxupquote{filename}} (\sphinxstyleliteralemphasis{\sphinxupquote{str}})

\sphinxlineitem{Тип результата}
\sphinxAtStartPar
int

\end{description}\end{quote}

\end{fulllineitems}\end{savenotes}

\index{get\_text\_from\_file() (в модуле scan\_module.read\_files)@\spxentry{get\_text\_from\_file()}\spxextra{в модуле scan\_module.read\_files}}

\begin{savenotes}\begin{fulllineitems}
\phantomsection\label{\detokenize{scan_module:scan_module.read_files.get_text_from_file}}
\pysigstartsignatures
\pysiglinewithargsret
{\sphinxcode{\sphinxupquote{scan\_module.read\_files.}}\sphinxbfcode{\sphinxupquote{get\_text\_from\_file}}}
{\sphinxparam{\DUrole{n}{filename}}}
{}
\pysigstopsignatures
\sphinxAtStartPar
Считывает весь текст из файла как единую строку.
ВНИМАНИЕ: Используйте только для небольших файлов!
\begin{quote}\begin{description}
\sphinxlineitem{Параметры}
\sphinxAtStartPar
\sphinxstyleliteralstrong{\sphinxupquote{filename}} (\sphinxstyleliteralemphasis{\sphinxupquote{str}})

\sphinxlineitem{Тип результата}
\sphinxAtStartPar
str

\end{description}\end{quote}

\end{fulllineitems}\end{savenotes}

\index{get\_text\_from\_file\_stream() (в модуле scan\_module.read\_files)@\spxentry{get\_text\_from\_file\_stream()}\spxextra{в модуле scan\_module.read\_files}}

\begin{savenotes}\begin{fulllineitems}
\phantomsection\label{\detokenize{scan_module:scan_module.read_files.get_text_from_file_stream}}
\pysigstartsignatures
\pysiglinewithargsret
{\sphinxcode{\sphinxupquote{scan\_module.read\_files.}}\sphinxbfcode{\sphinxupquote{get\_text\_from\_file\_stream}}}
{\sphinxparam{\DUrole{n}{filename}}\sphinxparamcomma \sphinxparam{\DUrole{n}{chunk\_size}\DUrole{o}{=}\DUrole{default_value}{8192}}}
{}
\pysigstopsignatures
\sphinxAtStartPar
Генератор для потокового чтения больших текстовых файлов.
Возвращает чанки текста для эффективной обработки.
\begin{quote}\begin{description}
\sphinxlineitem{Параметры}\begin{itemize}
\item {} 
\sphinxAtStartPar
\sphinxstyleliteralstrong{\sphinxupquote{filename}} (\sphinxstyleliteralemphasis{\sphinxupquote{str}})

\item {} 
\sphinxAtStartPar
\sphinxstyleliteralstrong{\sphinxupquote{chunk\_size}} (\sphinxstyleliteralemphasis{\sphinxupquote{int}})

\end{itemize}

\sphinxlineitem{Тип результата}
\sphinxAtStartPar
\sphinxstyleemphasis{Generator}{[}str, None, None{]}

\end{description}\end{quote}

\end{fulllineitems}\end{savenotes}

\index{count\_characters\_in\_file() (в модуле scan\_module.read\_files)@\spxentry{count\_characters\_in\_file()}\spxextra{в модуле scan\_module.read\_files}}

\begin{savenotes}\begin{fulllineitems}
\phantomsection\label{\detokenize{scan_module:scan_module.read_files.count_characters_in_file}}
\pysigstartsignatures
\pysiglinewithargsret
{\sphinxcode{\sphinxupquote{scan\_module.read\_files.}}\sphinxbfcode{\sphinxupquote{count\_characters\_in\_file}}}
{\sphinxparam{\DUrole{n}{filename}}}
{}
\pysigstopsignatures
\sphinxAtStartPar
Подсчитывает количество символов в файле
\begin{quote}\begin{description}
\sphinxlineitem{Параметры}
\sphinxAtStartPar
\sphinxstyleliteralstrong{\sphinxupquote{filename}} (\sphinxstyleliteralemphasis{\sphinxupquote{str}})

\sphinxlineitem{Тип результата}
\sphinxAtStartPar
int

\end{description}\end{quote}

\end{fulllineitems}\end{savenotes}



\subparagraph{Функции чтения слов}
\label{\detokenize{scan_module:id2}}\index{get\_words\_from\_file() (в модуле scan\_module.read\_files)@\spxentry{get\_words\_from\_file()}\spxextra{в модуле scan\_module.read\_files}}

\begin{savenotes}\begin{fulllineitems}
\phantomsection\label{\detokenize{scan_module:id0}}
\pysigstartsignatures
\pysiglinewithargsret
{\sphinxcode{\sphinxupquote{scan\_module.read\_files.}}\sphinxbfcode{\sphinxupquote{get\_words\_from\_file}}}
{\sphinxparam{\DUrole{n}{filename}}}
{}
\pysigstopsignatures
\sphinxAtStartPar
Считывает построчно из файла слова и преобразует
их в список для удобной обработки.
ВНИМАНИЕ: Используйте только для небольших файлов!
\begin{quote}\begin{description}
\sphinxlineitem{Параметры}
\sphinxAtStartPar
\sphinxstyleliteralstrong{\sphinxupquote{filename}} (\sphinxstyleliteralemphasis{\sphinxupquote{str}})

\sphinxlineitem{Тип результата}
\sphinxAtStartPar
list

\end{description}\end{quote}

\end{fulllineitems}\end{savenotes}

\index{get\_words\_from\_file\_stream() (в модуле scan\_module.read\_files)@\spxentry{get\_words\_from\_file\_stream()}\spxextra{в модуле scan\_module.read\_files}}

\begin{savenotes}\begin{fulllineitems}
\phantomsection\label{\detokenize{scan_module:id3}}
\pysigstartsignatures
\pysiglinewithargsret
{\sphinxcode{\sphinxupquote{scan\_module.read\_files.}}\sphinxbfcode{\sphinxupquote{get\_words\_from\_file\_stream}}}
{\sphinxparam{\DUrole{n}{filename}}\sphinxparamcomma \sphinxparam{\DUrole{n}{batch\_size}\DUrole{o}{=}\DUrole{default_value}{1000}}}
{}
\pysigstopsignatures
\sphinxAtStartPar
Генератор для потоковой обработки больших файлов.
Возвращает батчи слов для эффективной обработки.
\begin{quote}\begin{description}
\sphinxlineitem{Параметры}\begin{itemize}
\item {} 
\sphinxAtStartPar
\sphinxstyleliteralstrong{\sphinxupquote{filename}} (\sphinxstyleliteralemphasis{\sphinxupquote{str}})

\item {} 
\sphinxAtStartPar
\sphinxstyleliteralstrong{\sphinxupquote{batch\_size}} (\sphinxstyleliteralemphasis{\sphinxupquote{int}})

\end{itemize}

\sphinxlineitem{Тип результата}
\sphinxAtStartPar
\sphinxstyleemphasis{Generator}{[}\sphinxstyleemphasis{List}{[}str{]}, None, None{]}

\end{description}\end{quote}

\end{fulllineitems}\end{savenotes}



\subparagraph{Функции чтения текста}
\label{\detokenize{scan_module:id4}}\index{get\_text\_from\_file() (в модуле scan\_module.read\_files)@\spxentry{get\_text\_from\_file()}\spxextra{в модуле scan\_module.read\_files}}

\begin{savenotes}\begin{fulllineitems}
\phantomsection\label{\detokenize{scan_module:id5}}
\pysigstartsignatures
\pysiglinewithargsret
{\sphinxcode{\sphinxupquote{scan\_module.read\_files.}}\sphinxbfcode{\sphinxupquote{get\_text\_from\_file}}}
{\sphinxparam{\DUrole{n}{filename}}}
{}
\pysigstopsignatures
\sphinxAtStartPar
Считывает весь текст из файла как единую строку.
ВНИМАНИЕ: Используйте только для небольших файлов!
\begin{quote}\begin{description}
\sphinxlineitem{Параметры}
\sphinxAtStartPar
\sphinxstyleliteralstrong{\sphinxupquote{filename}} (\sphinxstyleliteralemphasis{\sphinxupquote{str}})

\sphinxlineitem{Тип результата}
\sphinxAtStartPar
str

\end{description}\end{quote}

\end{fulllineitems}\end{savenotes}

\index{get\_text\_from\_file\_stream() (в модуле scan\_module.read\_files)@\spxentry{get\_text\_from\_file\_stream()}\spxextra{в модуле scan\_module.read\_files}}

\begin{savenotes}\begin{fulllineitems}
\phantomsection\label{\detokenize{scan_module:id6}}
\pysigstartsignatures
\pysiglinewithargsret
{\sphinxcode{\sphinxupquote{scan\_module.read\_files.}}\sphinxbfcode{\sphinxupquote{get\_text\_from\_file\_stream}}}
{\sphinxparam{\DUrole{n}{filename}}\sphinxparamcomma \sphinxparam{\DUrole{n}{chunk\_size}\DUrole{o}{=}\DUrole{default_value}{8192}}}
{}
\pysigstopsignatures
\sphinxAtStartPar
Генератор для потокового чтения больших текстовых файлов.
Возвращает чанки текста для эффективной обработки.
\begin{quote}\begin{description}
\sphinxlineitem{Параметры}\begin{itemize}
\item {} 
\sphinxAtStartPar
\sphinxstyleliteralstrong{\sphinxupquote{filename}} (\sphinxstyleliteralemphasis{\sphinxupquote{str}})

\item {} 
\sphinxAtStartPar
\sphinxstyleliteralstrong{\sphinxupquote{chunk\_size}} (\sphinxstyleliteralemphasis{\sphinxupquote{int}})

\end{itemize}

\sphinxlineitem{Тип результата}
\sphinxAtStartPar
\sphinxstyleemphasis{Generator}{[}str, None, None{]}

\end{description}\end{quote}

\end{fulllineitems}\end{savenotes}



\subparagraph{Вспомогательные функции}
\label{\detokenize{scan_module:id7}}\index{get\_file\_size\_mb() (в модуле scan\_module.read\_files)@\spxentry{get\_file\_size\_mb()}\spxextra{в модуле scan\_module.read\_files}}

\begin{savenotes}\begin{fulllineitems}
\phantomsection\label{\detokenize{scan_module:id8}}
\pysigstartsignatures
\pysiglinewithargsret
{\sphinxcode{\sphinxupquote{scan\_module.read\_files.}}\sphinxbfcode{\sphinxupquote{get\_file\_size\_mb}}}
{\sphinxparam{\DUrole{n}{filename}}}
{}
\pysigstopsignatures
\sphinxAtStartPar
Возвращает размер файла в мегабайтах
\begin{quote}\begin{description}
\sphinxlineitem{Параметры}
\sphinxAtStartPar
\sphinxstyleliteralstrong{\sphinxupquote{filename}} (\sphinxstyleliteralemphasis{\sphinxupquote{str}})

\sphinxlineitem{Тип результата}
\sphinxAtStartPar
float

\end{description}\end{quote}

\end{fulllineitems}\end{savenotes}

\index{count\_lines\_in\_file() (в модуле scan\_module.read\_files)@\spxentry{count\_lines\_in\_file()}\spxextra{в модуле scan\_module.read\_files}}

\begin{savenotes}\begin{fulllineitems}
\phantomsection\label{\detokenize{scan_module:id9}}
\pysigstartsignatures
\pysiglinewithargsret
{\sphinxcode{\sphinxupquote{scan\_module.read\_files.}}\sphinxbfcode{\sphinxupquote{count\_lines\_in\_file}}}
{\sphinxparam{\DUrole{n}{filename}}}
{}
\pysigstopsignatures
\sphinxAtStartPar
Подсчитывает количество непустых строк в файле
\begin{quote}\begin{description}
\sphinxlineitem{Параметры}
\sphinxAtStartPar
\sphinxstyleliteralstrong{\sphinxupquote{filename}} (\sphinxstyleliteralemphasis{\sphinxupquote{str}})

\sphinxlineitem{Тип результата}
\sphinxAtStartPar
int

\end{description}\end{quote}

\end{fulllineitems}\end{savenotes}

\index{count\_characters\_in\_file() (в модуле scan\_module.read\_files)@\spxentry{count\_characters\_in\_file()}\spxextra{в модуле scan\_module.read\_files}}

\begin{savenotes}\begin{fulllineitems}
\phantomsection\label{\detokenize{scan_module:id10}}
\pysigstartsignatures
\pysiglinewithargsret
{\sphinxcode{\sphinxupquote{scan\_module.read\_files.}}\sphinxbfcode{\sphinxupquote{count\_characters\_in\_file}}}
{\sphinxparam{\DUrole{n}{filename}}}
{}
\pysigstopsignatures
\sphinxAtStartPar
Подсчитывает количество символов в файле
\begin{quote}\begin{description}
\sphinxlineitem{Параметры}
\sphinxAtStartPar
\sphinxstyleliteralstrong{\sphinxupquote{filename}} (\sphinxstyleliteralemphasis{\sphinxupquote{str}})

\sphinxlineitem{Тип результата}
\sphinxAtStartPar
int

\end{description}\end{quote}

\end{fulllineitems}\end{savenotes}



\paragraph{read\_layout}
\label{\detokenize{scan_module:module-scan_module.read_layout}}\label{\detokenize{scan_module:read-layout}}\index{module@\spxentry{module}!scan\_module.read\_layout@\spxentry{scan\_module.read\_layout}}\index{scan\_module.read\_layout@\spxentry{scan\_module.read\_layout}!module@\spxentry{module}}\index{read\_kl() (в модуле scan\_module.read\_layout)@\spxentry{read\_kl()}\spxextra{в модуле scan\_module.read\_layout}}

\begin{savenotes}\begin{fulllineitems}
\phantomsection\label{\detokenize{scan_module:scan_module.read_layout.read_kl}}
\pysigstartsignatures
\pysiglinewithargsret
{\sphinxcode{\sphinxupquote{scan\_module.read\_layout.}}\sphinxbfcode{\sphinxupquote{read\_kl}}}
{\sphinxparam{\DUrole{n}{filename}}}
{}
\pysigstopsignatures
\sphinxAtStartPar
Универсальная функция для чтения раскладок из различных форматов
Поддерживает: JSON, CSV, TXT (key:value), XML
\begin{quote}\begin{description}
\sphinxlineitem{Параметры}
\sphinxAtStartPar
\sphinxstyleliteralstrong{\sphinxupquote{filename}} (\sphinxstyleliteralemphasis{\sphinxupquote{str}})

\sphinxlineitem{Тип результата}
\sphinxAtStartPar
dict | None

\end{description}\end{quote}

\end{fulllineitems}\end{savenotes}

\index{\_read\_json\_layout() (в модуле scan\_module.read\_layout)@\spxentry{\_read\_json\_layout()}\spxextra{в модуле scan\_module.read\_layout}}

\begin{savenotes}\begin{fulllineitems}
\phantomsection\label{\detokenize{scan_module:scan_module.read_layout._read_json_layout}}
\pysigstartsignatures
\pysiglinewithargsret
{\sphinxcode{\sphinxupquote{scan\_module.read\_layout.}}\sphinxbfcode{\sphinxupquote{\_read\_json\_layout}}}
{\sphinxparam{\DUrole{n}{filename}}}
{}
\pysigstopsignatures
\sphinxAtStartPar
Читает раскладку из JSON файла
Ожидаемый формат: \{«a»: 1, «b»: 2, …\} или \{«layout»: \{«a»: 1, «b»: 2\}\}
\begin{quote}\begin{description}
\sphinxlineitem{Параметры}
\sphinxAtStartPar
\sphinxstyleliteralstrong{\sphinxupquote{filename}} (\sphinxstyleliteralemphasis{\sphinxupquote{str}})

\sphinxlineitem{Тип результата}
\sphinxAtStartPar
dict

\end{description}\end{quote}

\end{fulllineitems}\end{savenotes}

\index{\_read\_csv\_layout() (в модуле scan\_module.read\_layout)@\spxentry{\_read\_csv\_layout()}\spxextra{в модуле scan\_module.read\_layout}}

\begin{savenotes}\begin{fulllineitems}
\phantomsection\label{\detokenize{scan_module:scan_module.read_layout._read_csv_layout}}
\pysigstartsignatures
\pysiglinewithargsret
{\sphinxcode{\sphinxupquote{scan\_module.read\_layout.}}\sphinxbfcode{\sphinxupquote{\_read\_csv\_layout}}}
{\sphinxparam{\DUrole{n}{filename}}}
{}
\pysigstopsignatures
\sphinxAtStartPar
Читает раскладку из CSV файла
Ожидаемые форматы:
\sphinxhyphen{} letter,error
\sphinxhyphen{} key,value
\sphinxhyphen{} symbol,weight
\begin{quote}\begin{description}
\sphinxlineitem{Параметры}
\sphinxAtStartPar
\sphinxstyleliteralstrong{\sphinxupquote{filename}} (\sphinxstyleliteralemphasis{\sphinxupquote{str}})

\sphinxlineitem{Тип результата}
\sphinxAtStartPar
dict

\end{description}\end{quote}

\end{fulllineitems}\end{savenotes}

\index{\_read\_text\_layout() (в модуле scan\_module.read\_layout)@\spxentry{\_read\_text\_layout()}\spxextra{в модуле scan\_module.read\_layout}}

\begin{savenotes}\begin{fulllineitems}
\phantomsection\label{\detokenize{scan_module:scan_module.read_layout._read_text_layout}}
\pysigstartsignatures
\pysiglinewithargsret
{\sphinxcode{\sphinxupquote{scan\_module.read\_layout.}}\sphinxbfcode{\sphinxupquote{\_read\_text\_layout}}}
{\sphinxparam{\DUrole{n}{filename}}}
{}
\pysigstopsignatures
\sphinxAtStartPar
Читает раскладку из текстового файла
Поддерживаемые форматы:
\sphinxhyphen{} key:value
\sphinxhyphen{} key=value
\sphinxhyphen{} key value
\sphinxhyphen{} key       value
\begin{quote}\begin{description}
\sphinxlineitem{Параметры}
\sphinxAtStartPar
\sphinxstyleliteralstrong{\sphinxupquote{filename}} (\sphinxstyleliteralemphasis{\sphinxupquote{str}})

\sphinxlineitem{Тип результата}
\sphinxAtStartPar
dict

\end{description}\end{quote}

\end{fulllineitems}\end{savenotes}

\index{\_read\_xml\_layout() (в модуле scan\_module.read\_layout)@\spxentry{\_read\_xml\_layout()}\spxextra{в модуле scan\_module.read\_layout}}

\begin{savenotes}\begin{fulllineitems}
\phantomsection\label{\detokenize{scan_module:scan_module.read_layout._read_xml_layout}}
\pysigstartsignatures
\pysiglinewithargsret
{\sphinxcode{\sphinxupquote{scan\_module.read\_layout.}}\sphinxbfcode{\sphinxupquote{\_read\_xml\_layout}}}
{\sphinxparam{\DUrole{n}{filename}}}
{}
\pysigstopsignatures
\sphinxAtStartPar
Читает раскладку из XML файла
Ожидаемый формат:
<layout>
\begin{quote}

\sphinxAtStartPar
<key symbol=»a» error=»1»/>
<key symbol=»b» error=»2»/>
\end{quote}

\sphinxAtStartPar
</layout>
\begin{quote}\begin{description}
\sphinxlineitem{Параметры}
\sphinxAtStartPar
\sphinxstyleliteralstrong{\sphinxupquote{filename}} (\sphinxstyleliteralemphasis{\sphinxupquote{str}})

\sphinxlineitem{Тип результата}
\sphinxAtStartPar
dict

\end{description}\end{quote}

\end{fulllineitems}\end{savenotes}

\index{\_auto\_detect\_and\_read() (в модуле scan\_module.read\_layout)@\spxentry{\_auto\_detect\_and\_read()}\spxextra{в модуле scan\_module.read\_layout}}

\begin{savenotes}\begin{fulllineitems}
\phantomsection\label{\detokenize{scan_module:scan_module.read_layout._auto_detect_and_read}}
\pysigstartsignatures
\pysiglinewithargsret
{\sphinxcode{\sphinxupquote{scan\_module.read\_layout.}}\sphinxbfcode{\sphinxupquote{\_auto\_detect\_and\_read}}}
{\sphinxparam{\DUrole{n}{filename}}}
{}
\pysigstopsignatures
\sphinxAtStartPar
Автоматически определяет формат файла и читает раскладку
\begin{quote}\begin{description}
\sphinxlineitem{Параметры}
\sphinxAtStartPar
\sphinxstyleliteralstrong{\sphinxupquote{filename}} (\sphinxstyleliteralemphasis{\sphinxupquote{str}})

\sphinxlineitem{Тип результата}
\sphinxAtStartPar
dict

\end{description}\end{quote}

\end{fulllineitems}\end{savenotes}

\index{\_extract\_layout\_from\_dict() (в модуле scan\_module.read\_layout)@\spxentry{\_extract\_layout\_from\_dict()}\spxextra{в модуле scan\_module.read\_layout}}

\begin{savenotes}\begin{fulllineitems}
\phantomsection\label{\detokenize{scan_module:scan_module.read_layout._extract_layout_from_dict}}
\pysigstartsignatures
\pysiglinewithargsret
{\sphinxcode{\sphinxupquote{scan\_module.read\_layout.}}\sphinxbfcode{\sphinxupquote{\_extract\_layout\_from\_dict}}}
{\sphinxparam{\DUrole{n}{data}}}
{}
\pysigstopsignatures
\sphinxAtStartPar
Извлекает раскладку из словаря различной структуры
\begin{quote}\begin{description}
\sphinxlineitem{Параметры}
\sphinxAtStartPar
\sphinxstyleliteralstrong{\sphinxupquote{data}} (\sphinxstyleliteralemphasis{\sphinxupquote{Any}})

\sphinxlineitem{Тип результата}
\sphinxAtStartPar
dict

\end{description}\end{quote}

\end{fulllineitems}\end{savenotes}

\index{\_is\_numeric() (в модуле scan\_module.read\_layout)@\spxentry{\_is\_numeric()}\spxextra{в модуле scan\_module.read\_layout}}

\begin{savenotes}\begin{fulllineitems}
\phantomsection\label{\detokenize{scan_module:scan_module.read_layout._is_numeric}}
\pysigstartsignatures
\pysiglinewithargsret
{\sphinxcode{\sphinxupquote{scan\_module.read\_layout.}}\sphinxbfcode{\sphinxupquote{\_is\_numeric}}}
{\sphinxparam{\DUrole{n}{value}}}
{}
\pysigstopsignatures
\sphinxAtStartPar
Проверяет, является ли строка числом
\begin{quote}\begin{description}
\sphinxlineitem{Параметры}
\sphinxAtStartPar
\sphinxstyleliteralstrong{\sphinxupquote{value}} (\sphinxstyleliteralemphasis{\sphinxupquote{str}})

\sphinxlineitem{Тип результата}
\sphinxAtStartPar
bool

\end{description}\end{quote}

\end{fulllineitems}\end{savenotes}

\index{save\_layout\_to\_file() (в модуле scan\_module.read\_layout)@\spxentry{save\_layout\_to\_file()}\spxextra{в модуле scan\_module.read\_layout}}

\begin{savenotes}\begin{fulllineitems}
\phantomsection\label{\detokenize{scan_module:scan_module.read_layout.save_layout_to_file}}
\pysigstartsignatures
\pysiglinewithargsret
{\sphinxcode{\sphinxupquote{scan\_module.read\_layout.}}\sphinxbfcode{\sphinxupquote{save\_layout\_to\_file}}}
{\sphinxparam{\DUrole{n}{layout}}\sphinxparamcomma \sphinxparam{\DUrole{n}{filename}}\sphinxparamcomma \sphinxparam{\DUrole{n}{format\_type}\DUrole{o}{=}\DUrole{default_value}{'json'}}}
{}
\pysigstopsignatures
\sphinxAtStartPar
Сохраняет раскладку в файл в указанном формате
\begin{quote}\begin{description}
\sphinxlineitem{Параметры}\begin{itemize}
\item {} 
\sphinxAtStartPar
\sphinxstyleliteralstrong{\sphinxupquote{layout}} (\sphinxstyleliteralemphasis{\sphinxupquote{dict}}) – Словарь раскладки

\item {} 
\sphinxAtStartPar
\sphinxstyleliteralstrong{\sphinxupquote{filename}} (\sphinxstyleliteralemphasis{\sphinxupquote{str}}) – Путь к файлу для сохранения

\item {} 
\sphinxAtStartPar
\sphinxstyleliteralstrong{\sphinxupquote{format\_type}} (\sphinxstyleliteralemphasis{\sphinxupquote{str}}) – Формат файла („json“, „csv“, „txt“, „xml“)

\end{itemize}

\sphinxlineitem{Результат}
\sphinxAtStartPar
True если сохранение прошло успешно

\sphinxlineitem{Тип результата}
\sphinxAtStartPar
bool

\end{description}\end{quote}

\end{fulllineitems}\end{savenotes}

\index{validate\_layout() (в модуле scan\_module.read\_layout)@\spxentry{validate\_layout()}\spxextra{в модуле scan\_module.read\_layout}}

\begin{savenotes}\begin{fulllineitems}
\phantomsection\label{\detokenize{scan_module:scan_module.read_layout.validate_layout}}
\pysigstartsignatures
\pysiglinewithargsret
{\sphinxcode{\sphinxupquote{scan\_module.read\_layout.}}\sphinxbfcode{\sphinxupquote{validate\_layout}}}
{\sphinxparam{\DUrole{n}{layout}}}
{}
\pysigstopsignatures
\sphinxAtStartPar
Валидирует раскладку на корректность
\begin{quote}\begin{description}
\sphinxlineitem{Параметры}
\sphinxAtStartPar
\sphinxstyleliteralstrong{\sphinxupquote{layout}} (\sphinxstyleliteralemphasis{\sphinxupquote{dict}}) – Словарь раскладки для проверки

\sphinxlineitem{Результат}
\sphinxAtStartPar
(is\_valid, list\_of\_errors)

\sphinxlineitem{Тип результата}
\sphinxAtStartPar
tuple

\end{description}\end{quote}

\end{fulllineitems}\end{savenotes}


\sphinxstepscope


\subsection{Модуль тестов (tests\_module)}
\label{\detokenize{tests_module:tests-module}}\label{\detokenize{tests_module::doc}}
\sphinxAtStartPar
Обзор тестовых модулей для проверки функциональности проекта.


\subsection{tests\_data\_module}
\label{\detokenize{tests_module:tests-data-module}}
\sphinxAtStartPar
Тесты для модуля анализа данных и экспорта результатов.


\subsubsection{Тестируемые функции модуля make\_export\_file}
\label{\detokenize{tests_module:make-export-file}}

\paragraph{export\_unknown\_characters\_csv()}
\label{\detokenize{tests_module:export-unknown-characters-csv}}
\sphinxAtStartPar
Функция для экспорта неизвестных символов в CSV файл.


\subparagraph{Тесты:}
\label{\detokenize{tests_module:id1}}

\paragraph{\_get\_quality\_assessment()}
\label{\detokenize{tests_module:get-quality-assessment}}
\sphinxAtStartPar
Внутренняя функция для оценки качества текста.


\subparagraph{Тесты:}
\label{\detokenize{tests_module:id2}}

\paragraph{create\_csv\_report()}
\label{\detokenize{tests_module:create-csv-report}}
\sphinxAtStartPar
Функция для создания CSV отчета.


\subparagraph{Тесты:}
\label{\detokenize{tests_module:id3}}

\paragraph{create\_detailed\_csv\_report()}
\label{\detokenize{tests_module:create-detailed-csv-report}}
\sphinxAtStartPar
Функция для создания детального CSV отчета по нескольким результатам.


\subparagraph{Тесты:}
\label{\detokenize{tests_module:id4}}

\subsubsection{Тестируемые функции модуля make\_export\_plot}
\label{\detokenize{tests_module:make-export-plot}}

\paragraph{create\_analysis\_charts()}
\label{\detokenize{tests_module:create-analysis-charts}}
\sphinxAtStartPar
Функция для создания набора графиков анализа.


\subparagraph{Тесты:}
\label{\detokenize{tests_module:id5}}

\paragraph{\_create\_coverage\_pie\_chart()}
\label{\detokenize{tests_module:create-coverage-pie-chart}}
\sphinxAtStartPar
Внутренняя функция для создания круговой диаграммы покрытия.


\subparagraph{Тесты:}
\label{\detokenize{tests_module:id6}}

\paragraph{\_create\_error\_distribution\_chart()}
\label{\detokenize{tests_module:create-error-distribution-chart}}
\sphinxAtStartPar
Внутренняя функция для создания гистограммы распределения ошибок.


\subparagraph{Тесты:}
\label{\detokenize{tests_module:id7}}

\paragraph{create\_history\_comparison\_chart()}
\label{\detokenize{tests_module:create-history-comparison-chart}}
\sphinxAtStartPar
Функция для создания графика истории ошибок.


\subparagraph{Тесты:}
\label{\detokenize{tests_module:id8}}

\paragraph{create\_layouts\_comparison\_chart()}
\label{\detokenize{tests_module:create-layouts-comparison-chart}}
\sphinxAtStartPar
Функция для создания сравнительного графика раскладок.


\subparagraph{Тесты:}
\label{\detokenize{tests_module:id9}}

\paragraph{\_create\_metrics\_comparison\_chart()}
\label{\detokenize{tests_module:create-metrics-comparison-chart}}
\sphinxAtStartPar
Внутренняя функция для создания радарной диаграммы метрик.


\subparagraph{Тесты:}
\label{\detokenize{tests_module:id10}}

\subsection{tests\_database}
\label{\detokenize{tests_module:tests-database}}
\sphinxAtStartPar
Тесты для модуля работы с базой данных.


\subsubsection{Тестируемые функции модуля database}
\label{\detokenize{tests_module:database}}

\paragraph{get\_analysis\_history()}
\label{\detokenize{tests_module:get-analysis-history}}
\sphinxAtStartPar
Функция для получения истории анализа.


\subparagraph{Тесты:}
\label{\detokenize{tests_module:id11}}

\paragraph{take\_lk\_names\_from\_lk()}
\label{\detokenize{tests_module:take-lk-names-from-lk}}
\sphinxAtStartPar
Функция для получения имен раскладок из базы данных.


\subparagraph{Тесты:}
\label{\detokenize{tests_module:id12}}

\paragraph{take\_lk\_from\_db()}
\label{\detokenize{tests_module:take-lk-from-db}}
\sphinxAtStartPar
Функция для получения раскладки из базы данных.


\subparagraph{Тесты:}
\label{\detokenize{tests_module:id13}}

\paragraph{save\_analysis\_result()}
\label{\detokenize{tests_module:save-analysis-result}}
\sphinxAtStartPar
Функция для сохранения результатов анализа в базу данных.


\subparagraph{Тесты:}
\label{\detokenize{tests_module:id14}}

\paragraph{take\_all\_data\_from\_lk()}
\label{\detokenize{tests_module:take-all-data-from-lk}}
\sphinxAtStartPar
Функция для получения всех данных из таблицы раскладок.


\subparagraph{Тесты:}
\label{\detokenize{tests_module:id15}}

\paragraph{get\_analysis\_statistics()}
\label{\detokenize{tests_module:get-analysis-statistics}}
\sphinxAtStartPar
Функция для получения статистики анализа.


\subparagraph{Тесты:}
\label{\detokenize{tests_module:id16}}

\paragraph{delete\_analysis\_result()}
\label{\detokenize{tests_module:delete-analysis-result}}
\sphinxAtStartPar
Функция для удаления результатов анализа из базы данных.


\subparagraph{Тесты:}
\label{\detokenize{tests_module:id17}}

\subsubsection{Тестируемые функции модуля db\_init}
\label{\detokenize{tests_module:db-init}}

\paragraph{init\_tables()}
\label{\detokenize{tests_module:init-tables}}
\sphinxAtStartPar
Функция для инициализации таблиц в базе данных.


\subparagraph{Тесты:}
\label{\detokenize{tests_module:id18}}

\paragraph{make\_mok\_data()}
\label{\detokenize{tests_module:make-mok-data}}
\sphinxAtStartPar
Функция для создания тестовых данных в базе данных.


\subparagraph{Тесты:}
\label{\detokenize{tests_module:id19}}

\subsection{tests\_processing}
\label{\detokenize{tests_module:tests-processing}}
\sphinxAtStartPar
Тесты для модуля обработки данных.


\subsubsection{Тестируемые функции модуля calculate\_data}
\label{\detokenize{tests_module:calculate-data}}

\paragraph{make\_processing()}
\label{\detokenize{tests_module:make-processing}}
\sphinxAtStartPar
Функция для обработки списка слов.


\subparagraph{Тесты:}
\label{\detokenize{tests_module:id20}}

\paragraph{make\_processing\_stream()}
\label{\detokenize{tests_module:make-processing-stream}}
\sphinxAtStartPar
Функция для потоковой обработки слов.


\subparagraph{Тесты:}
\label{\detokenize{tests_module:id21}}

\paragraph{make\_text\_processing()}
\label{\detokenize{tests_module:make-text-processing}}
\sphinxAtStartPar
Функция для обработки текста.


\subparagraph{Тесты:}
\label{\detokenize{tests_module:id22}}

\paragraph{make\_text\_processing\_stream()}
\label{\detokenize{tests_module:make-text-processing-stream}}
\sphinxAtStartPar
Функция для потоковой обработки текста.


\subparagraph{Тесты:}
\label{\detokenize{tests_module:id23}}

\paragraph{validate\_rules()}
\label{\detokenize{tests_module:validate-rules}}
\sphinxAtStartPar
Функция для валидации правил раскладки.


\subparagraph{Тесты:}
\label{\detokenize{tests_module:id24}}

\subsection{tests\_scan}
\label{\detokenize{tests_module:tests-scan}}
\sphinxAtStartPar
Тесты для модуля сканирования и чтения файлов.


\subsubsection{Тестируемые функции модуля read\_layout}
\label{\detokenize{tests_module:read-layout}}

\paragraph{read\_kl()}
\label{\detokenize{tests_module:read-kl}}
\sphinxAtStartPar
Функция для чтения раскладки из файла.


\subparagraph{Тесты:}
\label{\detokenize{tests_module:id25}}

\paragraph{\_is\_numeric()}
\label{\detokenize{tests_module:is-numeric}}
\sphinxAtStartPar
Внутренняя функция для проверки числовых строк.


\subparagraph{Тесты:}
\label{\detokenize{tests_module:id26}}

\paragraph{\_extract\_layout\_from\_dict()}
\label{\detokenize{tests_module:extract-layout-from-dict}}
\sphinxAtStartPar
Внутренняя функция для извлечения раскладки из словаря.


\subparagraph{Тесты:}
\label{\detokenize{tests_module:id27}}

\paragraph{\_read\_json\_layout()}
\label{\detokenize{tests_module:read-json-layout}}
\sphinxAtStartPar
Внутренняя функция для чтения JSON раскладки.


\subparagraph{Тесты:}
\label{\detokenize{tests_module:id28}}

\paragraph{\_read\_xml\_layout()}
\label{\detokenize{tests_module:read-xml-layout}}
\sphinxAtStartPar
Внутренняя функция для чтения XML раскладки.


\subparagraph{Тесты:}
\label{\detokenize{tests_module:id29}}

\paragraph{\_auto\_detect\_and\_read()}
\label{\detokenize{tests_module:auto-detect-and-read}}
\sphinxAtStartPar
Внутренняя функция для автоматического определения и чтения формата.


\subparagraph{Тесты:}
\label{\detokenize{tests_module:id30}}

\paragraph{\_read\_text\_layout()}
\label{\detokenize{tests_module:read-text-layout}}
\sphinxAtStartPar
Внутренняя функция для чтения текстовой раскладки.


\subparagraph{Тесты:}
\label{\detokenize{tests_module:id31}}

\paragraph{save\_layout\_to\_file()}
\label{\detokenize{tests_module:save-layout-to-file}}
\sphinxAtStartPar
Функция для сохранения раскладки в файл.


\subparagraph{Тесты:}
\label{\detokenize{tests_module:id32}}

\paragraph{validate\_layout()}
\label{\detokenize{tests_module:validate-layout}}
\sphinxAtStartPar
Функция для валидации раскладки.


\subparagraph{Тесты:}
\label{\detokenize{tests_module:id33}}

\paragraph{\_read\_csv\_layout()}
\label{\detokenize{tests_module:read-csv-layout}}
\sphinxAtStartPar
Внутренняя функция для чтения CSV раскладки.


\subparagraph{Тесты:}
\label{\detokenize{tests_module:id34}}

\subsubsection{Тестируемые функции модуля read\_files}
\label{\detokenize{tests_module:read-files}}

\paragraph{get\_words\_from\_file()}
\label{\detokenize{tests_module:get-words-from-file}}
\sphinxAtStartPar
Функция для чтения слов из файла.


\subparagraph{Тесты:}
\label{\detokenize{tests_module:id35}}

\paragraph{get\_words\_from\_file\_stream()}
\label{\detokenize{tests_module:get-words-from-file-stream}}
\sphinxAtStartPar
Функция для потокового чтения слов из файла.


\subparagraph{Тесты:}
\label{\detokenize{tests_module:id36}}

\paragraph{get\_file\_size\_mb()}
\label{\detokenize{tests_module:get-file-size-mb}}
\sphinxAtStartPar
Функция для получения размера файла в мегабайтах.


\subparagraph{Тесты:}
\label{\detokenize{tests_module:id37}}

\paragraph{count\_lines\_in\_file()}
\label{\detokenize{tests_module:count-lines-in-file}}
\sphinxAtStartPar
Функция для подсчета строк в файле.


\subparagraph{Тесты:}
\label{\detokenize{tests_module:id38}}

\paragraph{count\_characters\_in\_file()}
\label{\detokenize{tests_module:count-characters-in-file}}
\sphinxAtStartPar
Функция для подсчета символов в файле.


\subparagraph{Тесты:}
\label{\detokenize{tests_module:id39}}

\paragraph{get\_text\_from\_file()}
\label{\detokenize{tests_module:get-text-from-file}}
\sphinxAtStartPar
Функция для чтения текста из файла.


\subparagraph{Тесты:}
\label{\detokenize{tests_module:id40}}

\paragraph{get\_text\_from\_file\_stream()}
\label{\detokenize{tests_module:get-text-from-file-stream}}
\sphinxAtStartPar
Функция для потокового чтения текста из файла.


\subparagraph{Тесты:}
\label{\detokenize{tests_module:id41}}

\subsection{test\_imports}
\label{\detokenize{tests_module:test-imports}}
\sphinxAtStartPar
Тесты для проверки корректности импортов всех модулей проекта.


\subsubsection{test\_imports.py}
\label{\detokenize{tests_module:module-tests_module.test_imports}}\label{\detokenize{tests_module:test-imports-py}}\index{module@\spxentry{module}!tests\_module.test\_imports@\spxentry{tests\_module.test\_imports}}\index{tests\_module.test\_imports@\spxentry{tests\_module.test\_imports}!module@\spxentry{module}}
\sphinxAtStartPar
Тестовый скрипт для проверки импортов


\subsection{Индексирование}
\label{\detokenize{tests_module:id42}}\begin{itemize}
\item {} 
\sphinxAtStartPar
\DUrole{xref}{\DUrole{std}{\DUrole{std-ref}{genindex}}}

\item {} 
\sphinxAtStartPar
\DUrole{xref}{\DUrole{std}{\DUrole{std-ref}{modindex}}}

\item {} 
\sphinxAtStartPar
\DUrole{xref}{\DUrole{std}{\DUrole{std-ref}{search}}}

\end{itemize}

\sphinxstepscope


\section{API Reference}
\label{\detokenize{api:api-reference}}\label{\detokenize{api::doc}}
\sphinxAtStartPar
Полная документация API всех модулей проекта.


\begin{savenotes}\sphinxattablestart
\sphinxthistablewithglobalstyle
\sphinxthistablewithnovlinesstyle
\centering
\begin{tabulary}{\linewidth}[t]{\X{1}{2}\X{1}{2}}
\sphinxtoprule
\sphinxtableatstartofbodyhook
\sphinxAtStartPar
{\hyperref[\detokenize{data_module:module-data_module}]{\sphinxcrossref{\sphinxcode{\sphinxupquote{data\_module}}}}} (\autopageref*{\detokenize{data_module:module-data_module}})
&
\sphinxAtStartPar

\\
\sphinxhline
\sphinxAtStartPar
{\hyperref[\detokenize{database_module:module-database_module}]{\sphinxcrossref{\sphinxcode{\sphinxupquote{database\_module}}}}} (\autopageref*{\detokenize{database_module:module-database_module}})
&
\sphinxAtStartPar

\\
\sphinxhline
\sphinxAtStartPar
{\hyperref[\detokenize{processing_module:module-processing_module}]{\sphinxcrossref{\sphinxcode{\sphinxupquote{processing\_module}}}}} (\autopageref*{\detokenize{processing_module:module-processing_module}})
&
\sphinxAtStartPar

\\
\sphinxhline
\sphinxAtStartPar
{\hyperref[\detokenize{scan_module:module-scan_module}]{\sphinxcrossref{\sphinxcode{\sphinxupquote{scan\_module}}}}} (\autopageref*{\detokenize{scan_module:module-scan_module}})
&
\sphinxAtStartPar

\\
\sphinxhline
\sphinxAtStartPar
{\hyperref[\detokenize{_autosummary/tests_module:module-tests_module}]{\sphinxcrossref{\sphinxcode{\sphinxupquote{tests\_module}}}}} (\autopageref*{\detokenize{_autosummary/tests_module:module-tests_module}})
&
\sphinxAtStartPar

\\
\sphinxbottomrule
\end{tabulary}
\sphinxtableafterendhook\par
\sphinxattableend\end{savenotes}

\sphinxstepscope


\subsection{data\_module}
\label{\detokenize{_autosummary/data_module:module-data_module}}\label{\detokenize{_autosummary/data_module:data-module}}\label{\detokenize{_autosummary/data_module::doc}}\index{module@\spxentry{module}!data\_module@\spxentry{data\_module}}\index{data\_module@\spxentry{data\_module}!module@\spxentry{module}}\subsubsection*{Modules}


\begin{savenotes}\sphinxattablestart
\sphinxthistablewithglobalstyle
\sphinxthistablewithnovlinesstyle
\centering
\begin{tabulary}{\linewidth}[t]{\X{1}{2}\X{1}{2}}
\sphinxtoprule
\sphinxtableatstartofbodyhook
\sphinxAtStartPar
{\hyperref[\detokenize{data_module:module-data_module.make_export_file}]{\sphinxcrossref{\sphinxcode{\sphinxupquote{make\_export\_file}}}}} (\autopageref*{\detokenize{data_module:module-data_module.make_export_file}})
&
\sphinxAtStartPar
Модуль для экспорта результатов анализа в CSV файлы
\\
\sphinxhline
\sphinxAtStartPar
{\hyperref[\detokenize{data_module:module-data_module.make_export_plot}]{\sphinxcrossref{\sphinxcode{\sphinxupquote{make\_export\_plot}}}}} (\autopageref*{\detokenize{data_module:module-data_module.make_export_plot}})
&
\sphinxAtStartPar
Модуль для создания графиков и визуализации результатов анализа
\\
\sphinxbottomrule
\end{tabulary}
\sphinxtableafterendhook\par
\sphinxattableend\end{savenotes}

\sphinxstepscope


\subsubsection{data\_module.make\_export\_file}
\label{\detokenize{_autosummary/data_module.make_export_file:module-data_module.make_export_file}}\label{\detokenize{_autosummary/data_module.make_export_file:data-module-make-export-file}}\label{\detokenize{_autosummary/data_module.make_export_file::doc}}\index{module@\spxentry{module}!data\_module.make\_export\_file@\spxentry{data\_module.make\_export\_file}}\index{data\_module.make\_export\_file@\spxentry{data\_module.make\_export\_file}!module@\spxentry{module}}
\sphinxAtStartPar
Модуль для экспорта результатов анализа в CSV файлы
\subsubsection*{Functions}


\begin{savenotes}\sphinxattablestart
\sphinxthistablewithglobalstyle
\sphinxthistablewithnovlinesstyle
\centering
\begin{tabulary}{\linewidth}[t]{\X{1}{2}\X{1}{2}}
\sphinxtoprule
\sphinxtableatstartofbodyhook
\sphinxAtStartPar
{\hyperref[\detokenize{data_module:data_module.make_export_file.create_csv_report}]{\sphinxcrossref{\sphinxcode{\sphinxupquote{create\_csv\_report}}}}} (\autopageref*{\detokenize{data_module:data_module.make_export_file.create_csv_report}})(result, file\_path, layout\_name)
&
\sphinxAtStartPar
Создает CSV отчет с результатами анализа
\\
\sphinxhline
\sphinxAtStartPar
{\hyperref[\detokenize{data_module:data_module.make_export_file.create_detailed_csv_report}]{\sphinxcrossref{\sphinxcode{\sphinxupquote{create\_detailed\_csv\_report}}}}} (\autopageref*{\detokenize{data_module:data_module.make_export_file.create_detailed_csv_report}})(results\_list{[}, ...{]})
&
\sphinxAtStartPar
Создает детальный CSV отчет для сравнения нескольких анализов
\\
\sphinxhline
\sphinxAtStartPar
{\hyperref[\detokenize{data_module:data_module.make_export_file.export_unknown_characters_csv}]{\sphinxcrossref{\sphinxcode{\sphinxupquote{export\_unknown\_characters\_csv}}}}} (\autopageref*{\detokenize{data_module:data_module.make_export_file.export_unknown_characters_csv}})(unknown\_chars, ...)
&
\sphinxAtStartPar
Экспортирует список неизвестных символов в отдельный CSV файл
\\
\sphinxbottomrule
\end{tabulary}
\sphinxtableafterendhook\par
\sphinxattableend\end{savenotes}
\subsubsection*{Classes}


\begin{savenotes}\sphinxattablestart
\sphinxthistablewithglobalstyle
\sphinxthistablewithnovlinesstyle
\centering
\begin{tabulary}{\linewidth}[t]{\X{1}{2}\X{1}{2}}
\sphinxtoprule
\sphinxtableatstartofbodyhook
\sphinxAtStartPar
\sphinxcode{\sphinxupquote{datetime}}(year, month, day{[}, hour{[}, minute{[}, ...)
&
\sphinxAtStartPar
The year, month and day arguments are required.
\\
\sphinxbottomrule
\end{tabulary}
\sphinxtableafterendhook\par
\sphinxattableend\end{savenotes}
\index{create\_csv\_report() (в модуле data\_module.make\_export\_file)@\spxentry{create\_csv\_report()}\spxextra{в модуле data\_module.make\_export\_file}}

\begin{savenotes}\begin{fulllineitems}
\phantomsection\label{\detokenize{_autosummary/data_module.make_export_file:data_module.make_export_file.create_csv_report}}
\pysigstartsignatures
\pysiglinewithargsret
{\sphinxcode{\sphinxupquote{data\_module.make\_export\_file.}}\sphinxbfcode{\sphinxupquote{create\_csv\_report}}}
{\sphinxparam{\DUrole{n}{result}}\sphinxparamcomma \sphinxparam{\DUrole{n}{file\_path}}\sphinxparamcomma \sphinxparam{\DUrole{n}{layout\_name}}\sphinxparamcomma \sphinxparam{\DUrole{n}{output\_dir}\DUrole{o}{=}\DUrole{default_value}{'reports'}}}
{}
\pysigstopsignatures
\sphinxAtStartPar
Создает CSV отчет с результатами анализа
\begin{quote}\begin{description}
\sphinxlineitem{Параметры}\begin{itemize}
\item {} 
\sphinxAtStartPar
\sphinxstyleliteralstrong{\sphinxupquote{result}} (\sphinxstyleliteralemphasis{\sphinxupquote{Dict}}\sphinxstyleliteralemphasis{\sphinxupquote{{[}}}\sphinxstyleliteralemphasis{\sphinxupquote{str}}\sphinxstyleliteralemphasis{\sphinxupquote{, }}\sphinxstyleliteralemphasis{\sphinxupquote{Any}}\sphinxstyleliteralemphasis{\sphinxupquote{{]}}}) – Результаты анализа

\item {} 
\sphinxAtStartPar
\sphinxstyleliteralstrong{\sphinxupquote{file\_path}} (\sphinxstyleliteralemphasis{\sphinxupquote{str}}) – Путь к анализируемому файлу

\item {} 
\sphinxAtStartPar
\sphinxstyleliteralstrong{\sphinxupquote{layout\_name}} (\sphinxstyleliteralemphasis{\sphinxupquote{str}}) – Название раскладки

\item {} 
\sphinxAtStartPar
\sphinxstyleliteralstrong{\sphinxupquote{output\_dir}} (\sphinxstyleliteralemphasis{\sphinxupquote{str}}) – Директория для сохранения отчетов

\end{itemize}

\sphinxlineitem{Результат}
\sphinxAtStartPar
Путь к созданному CSV файлу

\sphinxlineitem{Тип результата}
\sphinxAtStartPar
str

\end{description}\end{quote}

\end{fulllineitems}\end{savenotes}

\index{create\_detailed\_csv\_report() (в модуле data\_module.make\_export\_file)@\spxentry{create\_detailed\_csv\_report()}\spxextra{в модуле data\_module.make\_export\_file}}

\begin{savenotes}\begin{fulllineitems}
\phantomsection\label{\detokenize{_autosummary/data_module.make_export_file:data_module.make_export_file.create_detailed_csv_report}}
\pysigstartsignatures
\pysiglinewithargsret
{\sphinxcode{\sphinxupquote{data\_module.make\_export\_file.}}\sphinxbfcode{\sphinxupquote{create\_detailed\_csv\_report}}}
{\sphinxparam{\DUrole{n}{results\_list}}\sphinxparamcomma \sphinxparam{\DUrole{n}{output\_dir}\DUrole{o}{=}\DUrole{default_value}{'reports'}}}
{}
\pysigstopsignatures
\sphinxAtStartPar
Создает детальный CSV отчет для сравнения нескольких анализов
\begin{quote}\begin{description}
\sphinxlineitem{Параметры}\begin{itemize}
\item {} 
\sphinxAtStartPar
\sphinxstyleliteralstrong{\sphinxupquote{results\_list}} (\sphinxstyleliteralemphasis{\sphinxupquote{List}}\sphinxstyleliteralemphasis{\sphinxupquote{{[}}}\sphinxstyleliteralemphasis{\sphinxupquote{Dict}}\sphinxstyleliteralemphasis{\sphinxupquote{{[}}}\sphinxstyleliteralemphasis{\sphinxupquote{str}}\sphinxstyleliteralemphasis{\sphinxupquote{, }}\sphinxstyleliteralemphasis{\sphinxupquote{Any}}\sphinxstyleliteralemphasis{\sphinxupquote{{]}}}\sphinxstyleliteralemphasis{\sphinxupquote{{]}}}) – Список результатов анализа

\item {} 
\sphinxAtStartPar
\sphinxstyleliteralstrong{\sphinxupquote{output\_dir}} (\sphinxstyleliteralemphasis{\sphinxupquote{str}}) – Директория для сохранения отчетов

\end{itemize}

\sphinxlineitem{Результат}
\sphinxAtStartPar
Путь к созданному CSV файлу

\sphinxlineitem{Тип результата}
\sphinxAtStartPar
str

\end{description}\end{quote}

\end{fulllineitems}\end{savenotes}

\index{\_get\_quality\_assessment() (в модуле data\_module.make\_export\_file)@\spxentry{\_get\_quality\_assessment()}\spxextra{в модуле data\_module.make\_export\_file}}

\begin{savenotes}\begin{fulllineitems}
\phantomsection\label{\detokenize{_autosummary/data_module.make_export_file:data_module.make_export_file._get_quality_assessment}}
\pysigstartsignatures
\pysiglinewithargsret
{\sphinxcode{\sphinxupquote{data\_module.make\_export\_file.}}\sphinxbfcode{\sphinxupquote{\_get\_quality\_assessment}}}
{\sphinxparam{\DUrole{n}{avg\_errors\_per\_word}}}
{}
\pysigstopsignatures
\sphinxAtStartPar
Возвращает текстовую оценку качества
\begin{quote}\begin{description}
\sphinxlineitem{Параметры}
\sphinxAtStartPar
\sphinxstyleliteralstrong{\sphinxupquote{avg\_errors\_per\_word}} (\sphinxstyleliteralemphasis{\sphinxupquote{float}})

\sphinxlineitem{Тип результата}
\sphinxAtStartPar
str

\end{description}\end{quote}

\end{fulllineitems}\end{savenotes}

\index{export\_unknown\_characters\_csv() (в модуле data\_module.make\_export\_file)@\spxentry{export\_unknown\_characters\_csv()}\spxextra{в модуле data\_module.make\_export\_file}}

\begin{savenotes}\begin{fulllineitems}
\phantomsection\label{\detokenize{_autosummary/data_module.make_export_file:data_module.make_export_file.export_unknown_characters_csv}}
\pysigstartsignatures
\pysiglinewithargsret
{\sphinxcode{\sphinxupquote{data\_module.make\_export\_file.}}\sphinxbfcode{\sphinxupquote{export\_unknown\_characters\_csv}}}
{\sphinxparam{\DUrole{n}{unknown\_chars}}\sphinxparamcomma \sphinxparam{\DUrole{n}{layout\_name}}\sphinxparamcomma \sphinxparam{\DUrole{n}{output\_dir}\DUrole{o}{=}\DUrole{default_value}{'reports'}}}
{}
\pysigstopsignatures
\sphinxAtStartPar
Экспортирует список неизвестных символов в отдельный CSV файл
\begin{quote}\begin{description}
\sphinxlineitem{Параметры}\begin{itemize}
\item {} 
\sphinxAtStartPar
\sphinxstyleliteralstrong{\sphinxupquote{unknown\_chars}} (\sphinxstyleliteralemphasis{\sphinxupquote{set}}) – Множество неизвестных символов

\item {} 
\sphinxAtStartPar
\sphinxstyleliteralstrong{\sphinxupquote{layout\_name}} (\sphinxstyleliteralemphasis{\sphinxupquote{str}}) – Название раскладки

\item {} 
\sphinxAtStartPar
\sphinxstyleliteralstrong{\sphinxupquote{output\_dir}} (\sphinxstyleliteralemphasis{\sphinxupquote{str}}) – Директория для сохранения

\end{itemize}

\sphinxlineitem{Результат}
\sphinxAtStartPar
Путь к созданному файлу

\sphinxlineitem{Тип результата}
\sphinxAtStartPar
str

\end{description}\end{quote}

\end{fulllineitems}\end{savenotes}


\sphinxstepscope


\subsubsection{data\_module.make\_export\_plot}
\label{\detokenize{_autosummary/data_module.make_export_plot:module-data_module.make_export_plot}}\label{\detokenize{_autosummary/data_module.make_export_plot:data-module-make-export-plot}}\label{\detokenize{_autosummary/data_module.make_export_plot::doc}}\index{module@\spxentry{module}!data\_module.make\_export\_plot@\spxentry{data\_module.make\_export\_plot}}\index{data\_module.make\_export\_plot@\spxentry{data\_module.make\_export\_plot}!module@\spxentry{module}}
\sphinxAtStartPar
Модуль для создания графиков и визуализации результатов анализа
\subsubsection*{Functions}


\begin{savenotes}\sphinxattablestart
\sphinxthistablewithglobalstyle
\sphinxthistablewithnovlinesstyle
\centering
\begin{tabulary}{\linewidth}[t]{\X{1}{2}\X{1}{2}}
\sphinxtoprule
\sphinxtableatstartofbodyhook
\sphinxAtStartPar
{\hyperref[\detokenize{data_module:data_module.make_export_plot.create_analysis_charts}]{\sphinxcrossref{\sphinxcode{\sphinxupquote{create\_analysis\_charts}}}}} (\autopageref*{\detokenize{data_module:data_module.make_export_plot.create_analysis_charts}})(result, layout\_name, ...)
&
\sphinxAtStartPar
Создает набор графиков для анализа результатов
\\
\sphinxhline
\sphinxAtStartPar
{\hyperref[\detokenize{data_module:data_module.make_export_plot.create_history_comparison_chart}]{\sphinxcrossref{\sphinxcode{\sphinxupquote{create\_history\_comparison\_chart}}}}} (\autopageref*{\detokenize{data_module:data_module.make_export_plot.create_history_comparison_chart}})(layout\_name)
&
\sphinxAtStartPar
Создает график сравнения истории анализов для раскладки
\\
\sphinxhline
\sphinxAtStartPar
{\hyperref[\detokenize{data_module:data_module.make_export_plot.create_layouts_comparison_chart}]{\sphinxcrossref{\sphinxcode{\sphinxupquote{create\_layouts\_comparison\_chart}}}}} (\autopageref*{\detokenize{data_module:data_module.make_export_plot.create_layouts_comparison_chart}})({[}output\_dir{]})
&
\sphinxAtStartPar
Создает сравнительный график для всех раскладок
\\
\sphinxbottomrule
\end{tabulary}
\sphinxtableafterendhook\par
\sphinxattableend\end{savenotes}
\subsubsection*{Classes}


\begin{savenotes}\sphinxattablestart
\sphinxthistablewithglobalstyle
\sphinxthistablewithnovlinesstyle
\centering
\begin{tabulary}{\linewidth}[t]{\X{1}{2}\X{1}{2}}
\sphinxtoprule
\sphinxtableatstartofbodyhook
\sphinxAtStartPar
\sphinxcode{\sphinxupquote{datetime}}(year, month, day{[}, hour{[}, minute{[}, ...)
&
\sphinxAtStartPar
The year, month and day arguments are required.
\\
\sphinxbottomrule
\end{tabulary}
\sphinxtableafterendhook\par
\sphinxattableend\end{savenotes}
\index{create\_analysis\_charts() (в модуле data\_module.make\_export\_plot)@\spxentry{create\_analysis\_charts()}\spxextra{в модуле data\_module.make\_export\_plot}}

\begin{savenotes}\begin{fulllineitems}
\phantomsection\label{\detokenize{_autosummary/data_module.make_export_plot:data_module.make_export_plot.create_analysis_charts}}
\pysigstartsignatures
\pysiglinewithargsret
{\sphinxcode{\sphinxupquote{data\_module.make\_export\_plot.}}\sphinxbfcode{\sphinxupquote{create\_analysis\_charts}}}
{\sphinxparam{\DUrole{n}{result}}\sphinxparamcomma \sphinxparam{\DUrole{n}{layout\_name}}\sphinxparamcomma \sphinxparam{\DUrole{n}{file\_path}}\sphinxparamcomma \sphinxparam{\DUrole{n}{output\_dir}\DUrole{o}{=}\DUrole{default_value}{'reports'}}}
{}
\pysigstopsignatures
\sphinxAtStartPar
Создает набор графиков для анализа результатов
\begin{quote}\begin{description}
\sphinxlineitem{Параметры}\begin{itemize}
\item {} 
\sphinxAtStartPar
\sphinxstyleliteralstrong{\sphinxupquote{result}} (\sphinxstyleliteralemphasis{\sphinxupquote{Dict}}\sphinxstyleliteralemphasis{\sphinxupquote{{[}}}\sphinxstyleliteralemphasis{\sphinxupquote{str}}\sphinxstyleliteralemphasis{\sphinxupquote{, }}\sphinxstyleliteralemphasis{\sphinxupquote{Any}}\sphinxstyleliteralemphasis{\sphinxupquote{{]}}}) – Результаты анализа

\item {} 
\sphinxAtStartPar
\sphinxstyleliteralstrong{\sphinxupquote{layout\_name}} (\sphinxstyleliteralemphasis{\sphinxupquote{str}}) – Название раскладки

\item {} 
\sphinxAtStartPar
\sphinxstyleliteralstrong{\sphinxupquote{file\_path}} (\sphinxstyleliteralemphasis{\sphinxupquote{str}}) – Путь к анализируемому файлу

\item {} 
\sphinxAtStartPar
\sphinxstyleliteralstrong{\sphinxupquote{output\_dir}} (\sphinxstyleliteralemphasis{\sphinxupquote{str}}) – Директория для сохранения графиков

\end{itemize}

\sphinxlineitem{Результат}
\sphinxAtStartPar
Список путей к созданным графикам

\sphinxlineitem{Тип результата}
\sphinxAtStartPar
List{[}str{]}

\end{description}\end{quote}

\end{fulllineitems}\end{savenotes}

\index{\_create\_coverage\_pie\_chart() (в модуле data\_module.make\_export\_plot)@\spxentry{\_create\_coverage\_pie\_chart()}\spxextra{в модуле data\_module.make\_export\_plot}}

\begin{savenotes}\begin{fulllineitems}
\phantomsection\label{\detokenize{_autosummary/data_module.make_export_plot:data_module.make_export_plot._create_coverage_pie_chart}}
\pysigstartsignatures
\pysiglinewithargsret
{\sphinxcode{\sphinxupquote{data\_module.make\_export\_plot.}}\sphinxbfcode{\sphinxupquote{\_create\_coverage\_pie\_chart}}}
{\sphinxparam{\DUrole{n}{result}}\sphinxparamcomma \sphinxparam{\DUrole{n}{layout\_name}}\sphinxparamcomma \sphinxparam{\DUrole{n}{timestamp}}\sphinxparamcomma \sphinxparam{\DUrole{n}{output\_dir}}}
{}
\pysigstopsignatures
\sphinxAtStartPar
Создает круговую диаграмму покрытия символов
\begin{quote}\begin{description}
\sphinxlineitem{Параметры}\begin{itemize}
\item {} 
\sphinxAtStartPar
\sphinxstyleliteralstrong{\sphinxupquote{result}} (\sphinxstyleliteralemphasis{\sphinxupquote{Dict}}\sphinxstyleliteralemphasis{\sphinxupquote{{[}}}\sphinxstyleliteralemphasis{\sphinxupquote{str}}\sphinxstyleliteralemphasis{\sphinxupquote{, }}\sphinxstyleliteralemphasis{\sphinxupquote{Any}}\sphinxstyleliteralemphasis{\sphinxupquote{{]}}})

\item {} 
\sphinxAtStartPar
\sphinxstyleliteralstrong{\sphinxupquote{layout\_name}} (\sphinxstyleliteralemphasis{\sphinxupquote{str}})

\item {} 
\sphinxAtStartPar
\sphinxstyleliteralstrong{\sphinxupquote{timestamp}} (\sphinxstyleliteralemphasis{\sphinxupquote{str}})

\item {} 
\sphinxAtStartPar
\sphinxstyleliteralstrong{\sphinxupquote{output\_dir}} (\sphinxstyleliteralemphasis{\sphinxupquote{str}})

\end{itemize}

\sphinxlineitem{Тип результата}
\sphinxAtStartPar
str

\end{description}\end{quote}

\end{fulllineitems}\end{savenotes}

\index{\_create\_error\_distribution\_chart() (в модуле data\_module.make\_export\_plot)@\spxentry{\_create\_error\_distribution\_chart()}\spxextra{в модуле data\_module.make\_export\_plot}}

\begin{savenotes}\begin{fulllineitems}
\phantomsection\label{\detokenize{_autosummary/data_module.make_export_plot:data_module.make_export_plot._create_error_distribution_chart}}
\pysigstartsignatures
\pysiglinewithargsret
{\sphinxcode{\sphinxupquote{data\_module.make\_export\_plot.}}\sphinxbfcode{\sphinxupquote{\_create\_error\_distribution\_chart}}}
{\sphinxparam{\DUrole{n}{result}}\sphinxparamcomma \sphinxparam{\DUrole{n}{layout\_name}}\sphinxparamcomma \sphinxparam{\DUrole{n}{timestamp}}\sphinxparamcomma \sphinxparam{\DUrole{n}{output\_dir}}}
{}
\pysigstopsignatures
\sphinxAtStartPar
Создает гистограмму распределения ошибок
\begin{quote}\begin{description}
\sphinxlineitem{Параметры}\begin{itemize}
\item {} 
\sphinxAtStartPar
\sphinxstyleliteralstrong{\sphinxupquote{result}} (\sphinxstyleliteralemphasis{\sphinxupquote{Dict}}\sphinxstyleliteralemphasis{\sphinxupquote{{[}}}\sphinxstyleliteralemphasis{\sphinxupquote{str}}\sphinxstyleliteralemphasis{\sphinxupquote{, }}\sphinxstyleliteralemphasis{\sphinxupquote{Any}}\sphinxstyleliteralemphasis{\sphinxupquote{{]}}})

\item {} 
\sphinxAtStartPar
\sphinxstyleliteralstrong{\sphinxupquote{layout\_name}} (\sphinxstyleliteralemphasis{\sphinxupquote{str}})

\item {} 
\sphinxAtStartPar
\sphinxstyleliteralstrong{\sphinxupquote{timestamp}} (\sphinxstyleliteralemphasis{\sphinxupquote{str}})

\item {} 
\sphinxAtStartPar
\sphinxstyleliteralstrong{\sphinxupquote{output\_dir}} (\sphinxstyleliteralemphasis{\sphinxupquote{str}})

\end{itemize}

\sphinxlineitem{Тип результата}
\sphinxAtStartPar
str

\end{description}\end{quote}

\end{fulllineitems}\end{savenotes}

\index{\_create\_metrics\_comparison\_chart() (в модуле data\_module.make\_export\_plot)@\spxentry{\_create\_metrics\_comparison\_chart()}\spxextra{в модуле data\_module.make\_export\_plot}}

\begin{savenotes}\begin{fulllineitems}
\phantomsection\label{\detokenize{_autosummary/data_module.make_export_plot:data_module.make_export_plot._create_metrics_comparison_chart}}
\pysigstartsignatures
\pysiglinewithargsret
{\sphinxcode{\sphinxupquote{data\_module.make\_export\_plot.}}\sphinxbfcode{\sphinxupquote{\_create\_metrics\_comparison\_chart}}}
{\sphinxparam{\DUrole{n}{result}}\sphinxparamcomma \sphinxparam{\DUrole{n}{layout\_name}}\sphinxparamcomma \sphinxparam{\DUrole{n}{timestamp}}\sphinxparamcomma \sphinxparam{\DUrole{n}{output\_dir}}}
{}
\pysigstopsignatures
\sphinxAtStartPar
Создает сравнительную диаграмму метрик
\begin{quote}\begin{description}
\sphinxlineitem{Параметры}\begin{itemize}
\item {} 
\sphinxAtStartPar
\sphinxstyleliteralstrong{\sphinxupquote{result}} (\sphinxstyleliteralemphasis{\sphinxupquote{Dict}}\sphinxstyleliteralemphasis{\sphinxupquote{{[}}}\sphinxstyleliteralemphasis{\sphinxupquote{str}}\sphinxstyleliteralemphasis{\sphinxupquote{, }}\sphinxstyleliteralemphasis{\sphinxupquote{Any}}\sphinxstyleliteralemphasis{\sphinxupquote{{]}}})

\item {} 
\sphinxAtStartPar
\sphinxstyleliteralstrong{\sphinxupquote{layout\_name}} (\sphinxstyleliteralemphasis{\sphinxupquote{str}})

\item {} 
\sphinxAtStartPar
\sphinxstyleliteralstrong{\sphinxupquote{timestamp}} (\sphinxstyleliteralemphasis{\sphinxupquote{str}})

\item {} 
\sphinxAtStartPar
\sphinxstyleliteralstrong{\sphinxupquote{output\_dir}} (\sphinxstyleliteralemphasis{\sphinxupquote{str}})

\end{itemize}

\sphinxlineitem{Тип результата}
\sphinxAtStartPar
str

\end{description}\end{quote}

\end{fulllineitems}\end{savenotes}

\index{create\_history\_comparison\_chart() (в модуле data\_module.make\_export\_plot)@\spxentry{create\_history\_comparison\_chart()}\spxextra{в модуле data\_module.make\_export\_plot}}

\begin{savenotes}\begin{fulllineitems}
\phantomsection\label{\detokenize{_autosummary/data_module.make_export_plot:data_module.make_export_plot.create_history_comparison_chart}}
\pysigstartsignatures
\pysiglinewithargsret
{\sphinxcode{\sphinxupquote{data\_module.make\_export\_plot.}}\sphinxbfcode{\sphinxupquote{create\_history\_comparison\_chart}}}
{\sphinxparam{\DUrole{n}{layout\_name}}\sphinxparamcomma \sphinxparam{\DUrole{n}{output\_dir}\DUrole{o}{=}\DUrole{default_value}{'reports'}}}
{}
\pysigstopsignatures
\sphinxAtStartPar
Создает график сравнения истории анализов для раскладки
\begin{quote}\begin{description}
\sphinxlineitem{Параметры}\begin{itemize}
\item {} 
\sphinxAtStartPar
\sphinxstyleliteralstrong{\sphinxupquote{layout\_name}} (\sphinxstyleliteralemphasis{\sphinxupquote{str}}) – Название раскладки

\item {} 
\sphinxAtStartPar
\sphinxstyleliteralstrong{\sphinxupquote{output\_dir}} (\sphinxstyleliteralemphasis{\sphinxupquote{str}}) – Директория для сохранения

\end{itemize}

\sphinxlineitem{Результат}
\sphinxAtStartPar
Путь к созданному графику

\sphinxlineitem{Тип результата}
\sphinxAtStartPar
str

\end{description}\end{quote}

\end{fulllineitems}\end{savenotes}

\index{create\_layouts\_comparison\_chart() (в модуле data\_module.make\_export\_plot)@\spxentry{create\_layouts\_comparison\_chart()}\spxextra{в модуле data\_module.make\_export\_plot}}

\begin{savenotes}\begin{fulllineitems}
\phantomsection\label{\detokenize{_autosummary/data_module.make_export_plot:data_module.make_export_plot.create_layouts_comparison_chart}}
\pysigstartsignatures
\pysiglinewithargsret
{\sphinxcode{\sphinxupquote{data\_module.make\_export\_plot.}}\sphinxbfcode{\sphinxupquote{create\_layouts\_comparison\_chart}}}
{\sphinxparam{\DUrole{n}{output\_dir}\DUrole{o}{=}\DUrole{default_value}{'reports'}}}
{}
\pysigstopsignatures
\sphinxAtStartPar
Создает сравнительный график для всех раскладок
\begin{quote}\begin{description}
\sphinxlineitem{Параметры}
\sphinxAtStartPar
\sphinxstyleliteralstrong{\sphinxupquote{output\_dir}} (\sphinxstyleliteralemphasis{\sphinxupquote{str}}) – Директория для сохранения

\sphinxlineitem{Результат}
\sphinxAtStartPar
Путь к созданному графику

\sphinxlineitem{Тип результата}
\sphinxAtStartPar
str

\end{description}\end{quote}

\end{fulllineitems}\end{savenotes}


\sphinxstepscope


\subsection{database\_module}
\label{\detokenize{_autosummary/database_module:module-database_module}}\label{\detokenize{_autosummary/database_module:database-module}}\label{\detokenize{_autosummary/database_module::doc}}\index{module@\spxentry{module}!database\_module@\spxentry{database\_module}}\index{database\_module@\spxentry{database\_module}!module@\spxentry{module}}\subsubsection*{Modules}


\begin{savenotes}\sphinxattablestart
\sphinxthistablewithglobalstyle
\sphinxthistablewithnovlinesstyle
\centering
\begin{tabulary}{\linewidth}[t]{\X{1}{2}\X{1}{2}}
\sphinxtoprule
\sphinxtableatstartofbodyhook
\sphinxAtStartPar
{\hyperref[\detokenize{database_module:module-database_module.database}]{\sphinxcrossref{\sphinxcode{\sphinxupquote{database}}}}} (\autopageref*{\detokenize{database_module:module-database_module.database}})
&
\sphinxAtStartPar

\\
\sphinxbottomrule
\end{tabulary}
\sphinxtableafterendhook\par
\sphinxattableend\end{savenotes}

\sphinxstepscope


\subsubsection{database\_module.database}
\label{\detokenize{_autosummary/database_module.database:module-database_module.database}}\label{\detokenize{_autosummary/database_module.database:database-module-database}}\label{\detokenize{_autosummary/database_module.database::doc}}\index{module@\spxentry{module}!database\_module.database@\spxentry{database\_module.database}}\index{database\_module.database@\spxentry{database\_module.database}!module@\spxentry{module}}\subsubsection*{Functions}


\begin{savenotes}\sphinxattablestart
\sphinxthistablewithglobalstyle
\sphinxthistablewithnovlinesstyle
\centering
\begin{tabulary}{\linewidth}[t]{\X{1}{2}\X{1}{2}}
\sphinxtoprule
\sphinxtableatstartofbodyhook
\sphinxAtStartPar
{\hyperref[\detokenize{database_module:id9}]{\sphinxcrossref{\sphinxcode{\sphinxupquote{delete\_analysis\_result}}}}} (\autopageref*{\detokenize{database_module:id9}})(record\_id)
&
\sphinxAtStartPar
Удаляет результат анализа по ID
\\
\sphinxhline
\sphinxAtStartPar
{\hyperref[\detokenize{database_module:id7}]{\sphinxcrossref{\sphinxcode{\sphinxupquote{get\_analysis\_history}}}}} (\autopageref*{\detokenize{database_module:id7}})({[}layout\_name, limit{]})
&
\sphinxAtStartPar
Получает историю анализов из базы данных
\\
\sphinxhline
\sphinxAtStartPar
{\hyperref[\detokenize{database_module:id8}]{\sphinxcrossref{\sphinxcode{\sphinxupquote{get\_analysis\_statistics}}}}} (\autopageref*{\detokenize{database_module:id8}})(layout\_name)
&
\sphinxAtStartPar
Получает статистику анализов для раскладки
\\
\sphinxhline
\sphinxAtStartPar
{\hyperref[\detokenize{database_module:id6}]{\sphinxcrossref{\sphinxcode{\sphinxupquote{save\_analysis\_result}}}}} (\autopageref*{\detokenize{database_module:id6}})(layout\_name, result, ...)
&
\sphinxAtStartPar
Сохраняет результаты анализа в таблицу data
\\
\sphinxhline
\sphinxAtStartPar
{\hyperref[\detokenize{database_module:id3}]{\sphinxcrossref{\sphinxcode{\sphinxupquote{take\_all\_data\_from\_lk}}}}} (\autopageref*{\detokenize{database_module:id3}})()
&
\sphinxAtStartPar
Возвращает все содержимое из таблицы lk(с раскладками), включая тестовые
\\
\sphinxhline
\sphinxAtStartPar
{\hyperref[\detokenize{database_module:id0}]{\sphinxcrossref{\sphinxcode{\sphinxupquote{take\_lk\_from\_db}}}}} (\autopageref*{\detokenize{database_module:id0}})(name)
&
\sphinxAtStartPar
Возвращает словарь правил раскладки, по ее имени Если такой раскладки нет \sphinxhyphen{} вернет None
\\
\sphinxhline
\sphinxAtStartPar
{\hyperref[\detokenize{database_module:id4}]{\sphinxcrossref{\sphinxcode{\sphinxupquote{take\_lk\_names\_from\_lk}}}}} (\autopageref*{\detokenize{database_module:id4}})()
&
\sphinxAtStartPar
Врозвращает список всех имеющихся в бд раскладок, кроме тех, в названии которых есть слово test.
\\
\sphinxbottomrule
\end{tabulary}
\sphinxtableafterendhook\par
\sphinxattableend\end{savenotes}
\index{take\_lk\_from\_db() (в модуле database\_module.database)@\spxentry{take\_lk\_from\_db()}\spxextra{в модуле database\_module.database}}

\begin{savenotes}\begin{fulllineitems}
\phantomsection\label{\detokenize{_autosummary/database_module.database:database_module.database.take_lk_from_db}}
\pysigstartsignatures
\pysiglinewithargsret
{\sphinxcode{\sphinxupquote{database\_module.database.}}\sphinxbfcode{\sphinxupquote{take\_lk\_from\_db}}}
{\sphinxparam{\DUrole{n}{name}}}
{}
\pysigstopsignatures
\sphinxAtStartPar
Возвращает словарь правил раскладки, по ее имени
Если такой раскладки нет \sphinxhyphen{} вернет None
\begin{quote}\begin{description}
\sphinxlineitem{Параметры}
\sphinxAtStartPar
\sphinxstyleliteralstrong{\sphinxupquote{name}} (\sphinxstyleliteralemphasis{\sphinxupquote{str}})

\sphinxlineitem{Тип результата}
\sphinxAtStartPar
dict | None

\end{description}\end{quote}

\end{fulllineitems}\end{savenotes}

\index{take\_all\_data\_from\_lk() (в модуле database\_module.database)@\spxentry{take\_all\_data\_from\_lk()}\spxextra{в модуле database\_module.database}}

\begin{savenotes}\begin{fulllineitems}
\phantomsection\label{\detokenize{_autosummary/database_module.database:database_module.database.take_all_data_from_lk}}
\pysigstartsignatures
\pysiglinewithargsret
{\sphinxcode{\sphinxupquote{database\_module.database.}}\sphinxbfcode{\sphinxupquote{take\_all\_data\_from\_lk}}}
{}
{}
\pysigstopsignatures
\sphinxAtStartPar
Возвращает все содержимое из таблицы lk(с раскладками), включая тестовые
\begin{quote}\begin{description}
\sphinxlineitem{Тип результата}
\sphinxAtStartPar
list

\end{description}\end{quote}

\end{fulllineitems}\end{savenotes}

\index{take\_lk\_names\_from\_lk() (в модуле database\_module.database)@\spxentry{take\_lk\_names\_from\_lk()}\spxextra{в модуле database\_module.database}}

\begin{savenotes}\begin{fulllineitems}
\phantomsection\label{\detokenize{_autosummary/database_module.database:database_module.database.take_lk_names_from_lk}}
\pysigstartsignatures
\pysiglinewithargsret
{\sphinxcode{\sphinxupquote{database\_module.database.}}\sphinxbfcode{\sphinxupquote{take\_lk\_names\_from\_lk}}}
{}
{}
\pysigstopsignatures
\sphinxAtStartPar
Врозвращает список всех имеющихся в бд раскладок,
кроме тех, в названии которых есть слово test.
\begin{quote}\begin{description}
\sphinxlineitem{Тип результата}
\sphinxAtStartPar
list

\end{description}\end{quote}

\end{fulllineitems}\end{savenotes}

\index{save\_analysis\_result() (в модуле database\_module.database)@\spxentry{save\_analysis\_result()}\spxextra{в модуле database\_module.database}}

\begin{savenotes}\begin{fulllineitems}
\phantomsection\label{\detokenize{_autosummary/database_module.database:database_module.database.save_analysis_result}}
\pysigstartsignatures
\pysiglinewithargsret
{\sphinxcode{\sphinxupquote{database\_module.database.}}\sphinxbfcode{\sphinxupquote{save\_analysis\_result}}}
{\sphinxparam{\DUrole{n}{layout\_name}}\sphinxparamcomma \sphinxparam{\DUrole{n}{result}}\sphinxparamcomma \sphinxparam{\DUrole{n}{file\_path}}\sphinxparamcomma \sphinxparam{\DUrole{n}{analysis\_type}\DUrole{o}{=}\DUrole{default_value}{'words'}}}
{}
\pysigstopsignatures
\sphinxAtStartPar
Сохраняет результаты анализа в таблицу data
\begin{quote}\begin{description}
\sphinxlineitem{Параметры}\begin{itemize}
\item {} 
\sphinxAtStartPar
\sphinxstyleliteralstrong{\sphinxupquote{layout\_name}} (\sphinxstyleliteralemphasis{\sphinxupquote{str}}) – Название раскладки

\item {} 
\sphinxAtStartPar
\sphinxstyleliteralstrong{\sphinxupquote{result}} (\sphinxstyleliteralemphasis{\sphinxupquote{dict}}) – Результаты анализа

\item {} 
\sphinxAtStartPar
\sphinxstyleliteralstrong{\sphinxupquote{file\_path}} (\sphinxstyleliteralemphasis{\sphinxupquote{str}}) – Путь к анализируемому файлу

\item {} 
\sphinxAtStartPar
\sphinxstyleliteralstrong{\sphinxupquote{analysis\_type}} (\sphinxstyleliteralemphasis{\sphinxupquote{str}}) – Тип анализа („words“ или „text“)

\end{itemize}

\sphinxlineitem{Результат}
\sphinxAtStartPar
ID созданной записи

\sphinxlineitem{Тип результата}
\sphinxAtStartPar
int

\end{description}\end{quote}

\end{fulllineitems}\end{savenotes}

\index{get\_analysis\_history() (в модуле database\_module.database)@\spxentry{get\_analysis\_history()}\spxextra{в модуле database\_module.database}}

\begin{savenotes}\begin{fulllineitems}
\phantomsection\label{\detokenize{_autosummary/database_module.database:database_module.database.get_analysis_history}}
\pysigstartsignatures
\pysiglinewithargsret
{\sphinxcode{\sphinxupquote{database\_module.database.}}\sphinxbfcode{\sphinxupquote{get\_analysis\_history}}}
{\sphinxparam{\DUrole{n}{layout\_name}\DUrole{o}{=}\DUrole{default_value}{None}}\sphinxparamcomma \sphinxparam{\DUrole{n}{limit}\DUrole{o}{=}\DUrole{default_value}{50}}}
{}
\pysigstopsignatures
\sphinxAtStartPar
Получает историю анализов из базы данных
\begin{quote}\begin{description}
\sphinxlineitem{Параметры}\begin{itemize}
\item {} 
\sphinxAtStartPar
\sphinxstyleliteralstrong{\sphinxupquote{layout\_name}} (\sphinxstyleliteralemphasis{\sphinxupquote{str}}\sphinxstyleliteralemphasis{\sphinxupquote{ | }}\sphinxstyleliteralemphasis{\sphinxupquote{None}}) – Название раскладки (если None, то все раскладки)

\item {} 
\sphinxAtStartPar
\sphinxstyleliteralstrong{\sphinxupquote{limit}} (\sphinxstyleliteralemphasis{\sphinxupquote{int}}) – Максимальное количество записей

\end{itemize}

\sphinxlineitem{Результат}
\sphinxAtStartPar
Список записей с результатами

\sphinxlineitem{Тип результата}
\sphinxAtStartPar
list

\end{description}\end{quote}

\end{fulllineitems}\end{savenotes}

\index{get\_analysis\_statistics() (в модуле database\_module.database)@\spxentry{get\_analysis\_statistics()}\spxextra{в модуле database\_module.database}}

\begin{savenotes}\begin{fulllineitems}
\phantomsection\label{\detokenize{_autosummary/database_module.database:database_module.database.get_analysis_statistics}}
\pysigstartsignatures
\pysiglinewithargsret
{\sphinxcode{\sphinxupquote{database\_module.database.}}\sphinxbfcode{\sphinxupquote{get\_analysis\_statistics}}}
{\sphinxparam{\DUrole{n}{layout\_name}}}
{}
\pysigstopsignatures
\sphinxAtStartPar
Получает статистику анализов для раскладки
\begin{quote}\begin{description}
\sphinxlineitem{Параметры}
\sphinxAtStartPar
\sphinxstyleliteralstrong{\sphinxupquote{layout\_name}} (\sphinxstyleliteralemphasis{\sphinxupquote{str}}) – Название раскладки

\sphinxlineitem{Результат}
\sphinxAtStartPar
Статистика анализов

\sphinxlineitem{Тип результата}
\sphinxAtStartPar
dict

\end{description}\end{quote}

\end{fulllineitems}\end{savenotes}

\index{delete\_analysis\_result() (в модуле database\_module.database)@\spxentry{delete\_analysis\_result()}\spxextra{в модуле database\_module.database}}

\begin{savenotes}\begin{fulllineitems}
\phantomsection\label{\detokenize{_autosummary/database_module.database:database_module.database.delete_analysis_result}}
\pysigstartsignatures
\pysiglinewithargsret
{\sphinxcode{\sphinxupquote{database\_module.database.}}\sphinxbfcode{\sphinxupquote{delete\_analysis\_result}}}
{\sphinxparam{\DUrole{n}{record\_id}}}
{}
\pysigstopsignatures
\sphinxAtStartPar
Удаляет результат анализа по ID
\begin{quote}\begin{description}
\sphinxlineitem{Параметры}
\sphinxAtStartPar
\sphinxstyleliteralstrong{\sphinxupquote{record\_id}} (\sphinxstyleliteralemphasis{\sphinxupquote{int}}) – ID записи для удаления

\sphinxlineitem{Результат}
\sphinxAtStartPar
True если запись была удалена

\sphinxlineitem{Тип результата}
\sphinxAtStartPar
bool

\end{description}\end{quote}

\end{fulllineitems}\end{savenotes}


\sphinxstepscope


\subsection{processing\_module}
\label{\detokenize{_autosummary/processing_module:module-processing_module}}\label{\detokenize{_autosummary/processing_module:processing-module}}\label{\detokenize{_autosummary/processing_module::doc}}\index{module@\spxentry{module}!processing\_module@\spxentry{processing\_module}}\index{processing\_module@\spxentry{processing\_module}!module@\spxentry{module}}\subsubsection*{Modules}


\begin{savenotes}\sphinxattablestart
\sphinxthistablewithglobalstyle
\sphinxthistablewithnovlinesstyle
\centering
\begin{tabulary}{\linewidth}[t]{\X{1}{2}\X{1}{2}}
\sphinxtoprule
\sphinxtableatstartofbodyhook
\sphinxAtStartPar
{\hyperref[\detokenize{processing_module:module-processing_module.calculate_data}]{\sphinxcrossref{\sphinxcode{\sphinxupquote{calculate\_data}}}}} (\autopageref*{\detokenize{processing_module:module-processing_module.calculate_data}})
&
\sphinxAtStartPar

\\
\sphinxbottomrule
\end{tabulary}
\sphinxtableafterendhook\par
\sphinxattableend\end{savenotes}

\sphinxstepscope


\subsubsection{processing\_module.calculate\_data}
\label{\detokenize{_autosummary/processing_module.calculate_data:module-processing_module.calculate_data}}\label{\detokenize{_autosummary/processing_module.calculate_data:processing-module-calculate-data}}\label{\detokenize{_autosummary/processing_module.calculate_data::doc}}\index{module@\spxentry{module}!processing\_module.calculate\_data@\spxentry{processing\_module.calculate\_data}}\index{processing\_module.calculate\_data@\spxentry{processing\_module.calculate\_data}!module@\spxentry{module}}\subsubsection*{Functions}


\begin{savenotes}\sphinxattablestart
\sphinxthistablewithglobalstyle
\sphinxthistablewithnovlinesstyle
\centering
\begin{tabulary}{\linewidth}[t]{\X{1}{2}\X{1}{2}}
\sphinxtoprule
\sphinxtableatstartofbodyhook
\sphinxAtStartPar
{\hyperref[\detokenize{processing_module:id0}]{\sphinxcrossref{\sphinxcode{\sphinxupquote{make\_processing}}}}} (\autopageref*{\detokenize{processing_module:id0}})(wordlist, rules)
&
\sphinxAtStartPar
Считает количество ошибок по словарю правил и списку слов.
\\
\sphinxhline
\sphinxAtStartPar
{\hyperref[\detokenize{processing_module:id3}]{\sphinxcrossref{\sphinxcode{\sphinxupquote{make\_processing\_stream}}}}} (\autopageref*{\detokenize{processing_module:id3}})(wordlist\_generator, rules)
&
\sphinxAtStartPar
Обрабатывает большие файлы батчами с прогресс\sphinxhyphen{}баром.
\\
\sphinxhline
\sphinxAtStartPar
{\hyperref[\detokenize{processing_module:id5}]{\sphinxcrossref{\sphinxcode{\sphinxupquote{make\_text\_processing}}}}} (\autopageref*{\detokenize{processing_module:id5}})(text, rules)
&
\sphinxAtStartPar
Считает количество ошибок по словарю правил для сплошного текста.
\\
\sphinxhline
\sphinxAtStartPar
{\hyperref[\detokenize{processing_module:id6}]{\sphinxcrossref{\sphinxcode{\sphinxupquote{make\_text\_processing\_stream}}}}} (\autopageref*{\detokenize{processing_module:id6}})(text\_generator, ...)
&
\sphinxAtStartPar
Обрабатывает большие текстовые файлы чанками с прогресс\sphinxhyphen{}баром.
\\
\sphinxhline
\sphinxAtStartPar
{\hyperref[\detokenize{processing_module:id8}]{\sphinxcrossref{\sphinxcode{\sphinxupquote{validate\_rules}}}}} (\autopageref*{\detokenize{processing_module:id8}})(rules)
&
\sphinxAtStartPar
Проверяет корректность словаря правил.
\\
\sphinxbottomrule
\end{tabulary}
\sphinxtableafterendhook\par
\sphinxattableend\end{savenotes}
\subsubsection*{Classes}


\begin{savenotes}\sphinxattablestart
\sphinxthistablewithglobalstyle
\sphinxthistablewithnovlinesstyle
\centering
\begin{tabulary}{\linewidth}[t]{\X{1}{2}\X{1}{2}}
\sphinxtoprule
\sphinxtableatstartofbodyhook
\sphinxAtStartPar
\sphinxcode{\sphinxupquote{tqdm}}(*\_, **\_\_)
&
\sphinxAtStartPar
Decorate an iterable object, returning an iterator which acts exactly like the original iterable, but prints a dynamically updating progressbar every time a value is requested.
\\
\sphinxbottomrule
\end{tabulary}
\sphinxtableafterendhook\par
\sphinxattableend\end{savenotes}
\index{make\_processing() (в модуле processing\_module.calculate\_data)@\spxentry{make\_processing()}\spxextra{в модуле processing\_module.calculate\_data}}

\begin{savenotes}\begin{fulllineitems}
\phantomsection\label{\detokenize{_autosummary/processing_module.calculate_data:processing_module.calculate_data.make_processing}}
\pysigstartsignatures
\pysiglinewithargsret
{\sphinxcode{\sphinxupquote{processing\_module.calculate\_data.}}\sphinxbfcode{\sphinxupquote{make\_processing}}}
{\sphinxparam{\DUrole{n}{wordlist}}\sphinxparamcomma \sphinxparam{\DUrole{n}{rules}}}
{}
\pysigstopsignatures
\sphinxAtStartPar
Считает количество ошибок по словарю правил и списку слов.
ВНИМАНИЕ: Используйте только для небольших списков!
\begin{quote}\begin{description}
\sphinxlineitem{Результат}
\sphinxAtStartPar
Словарь с результатами анализа

\sphinxlineitem{Тип результата}
\sphinxAtStartPar
dict

\sphinxlineitem{Параметры}\begin{itemize}
\item {} 
\sphinxAtStartPar
\sphinxstyleliteralstrong{\sphinxupquote{wordlist}} (\sphinxstyleliteralemphasis{\sphinxupquote{list}})

\item {} 
\sphinxAtStartPar
\sphinxstyleliteralstrong{\sphinxupquote{rules}} (\sphinxstyleliteralemphasis{\sphinxupquote{dict}})

\end{itemize}

\end{description}\end{quote}

\end{fulllineitems}\end{savenotes}

\index{make\_processing\_stream() (в модуле processing\_module.calculate\_data)@\spxentry{make\_processing\_stream()}\spxextra{в модуле processing\_module.calculate\_data}}

\begin{savenotes}\begin{fulllineitems}
\phantomsection\label{\detokenize{_autosummary/processing_module.calculate_data:processing_module.calculate_data.make_processing_stream}}
\pysigstartsignatures
\pysiglinewithargsret
{\sphinxcode{\sphinxupquote{processing\_module.calculate\_data.}}\sphinxbfcode{\sphinxupquote{make\_processing\_stream}}}
{\sphinxparam{\DUrole{n}{wordlist\_generator}}\sphinxparamcomma \sphinxparam{\DUrole{n}{rules}}\sphinxparamcomma \sphinxparam{\DUrole{n}{total\_words}\DUrole{o}{=}\DUrole{default_value}{None}}}
{}
\pysigstopsignatures
\sphinxAtStartPar
Обрабатывает большие файлы батчами с прогресс\sphinxhyphen{}баром.
\begin{quote}\begin{description}
\sphinxlineitem{Параметры}\begin{itemize}
\item {} 
\sphinxAtStartPar
\sphinxstyleliteralstrong{\sphinxupquote{wordlist\_generator}} (\sphinxstyleliteralemphasis{\sphinxupquote{Generator}}\sphinxstyleliteralemphasis{\sphinxupquote{{[}}}\sphinxstyleliteralemphasis{\sphinxupquote{List}}\sphinxstyleliteralemphasis{\sphinxupquote{{[}}}\sphinxstyleliteralemphasis{\sphinxupquote{str}}\sphinxstyleliteralemphasis{\sphinxupquote{{]}}}\sphinxstyleliteralemphasis{\sphinxupquote{, }}\sphinxstyleliteralemphasis{\sphinxupquote{None}}\sphinxstyleliteralemphasis{\sphinxupquote{, }}\sphinxstyleliteralemphasis{\sphinxupquote{None}}\sphinxstyleliteralemphasis{\sphinxupquote{{]}}}) – Генератор батчей слов

\item {} 
\sphinxAtStartPar
\sphinxstyleliteralstrong{\sphinxupquote{rules}} (\sphinxstyleliteralemphasis{\sphinxupquote{Dict}}\sphinxstyleliteralemphasis{\sphinxupquote{{[}}}\sphinxstyleliteralemphasis{\sphinxupquote{str}}\sphinxstyleliteralemphasis{\sphinxupquote{, }}\sphinxstyleliteralemphasis{\sphinxupquote{int}}\sphinxstyleliteralemphasis{\sphinxupquote{{]}}}) – Словарь правил для подсчета ошибок

\item {} 
\sphinxAtStartPar
\sphinxstyleliteralstrong{\sphinxupquote{total\_words}} (\sphinxstyleliteralemphasis{\sphinxupquote{int}}\sphinxstyleliteralemphasis{\sphinxupquote{ | }}\sphinxstyleliteralemphasis{\sphinxupquote{None}}) – Общее количество слов для прогресс\sphinxhyphen{}бара (опционально)

\end{itemize}

\sphinxlineitem{Результат}
\sphinxAtStartPar
Словарь с результатами анализа

\sphinxlineitem{Тип результата}
\sphinxAtStartPar
dict

\end{description}\end{quote}

\end{fulllineitems}\end{savenotes}

\index{validate\_rules() (в модуле processing\_module.calculate\_data)@\spxentry{validate\_rules()}\spxextra{в модуле processing\_module.calculate\_data}}

\begin{savenotes}\begin{fulllineitems}
\phantomsection\label{\detokenize{_autosummary/processing_module.calculate_data:processing_module.calculate_data.validate_rules}}
\pysigstartsignatures
\pysiglinewithargsret
{\sphinxcode{\sphinxupquote{processing\_module.calculate\_data.}}\sphinxbfcode{\sphinxupquote{validate\_rules}}}
{\sphinxparam{\DUrole{n}{rules}}}
{}
\pysigstopsignatures
\sphinxAtStartPar
Проверяет корректность словаря правил.
\begin{quote}\begin{description}
\sphinxlineitem{Параметры}
\sphinxAtStartPar
\sphinxstyleliteralstrong{\sphinxupquote{rules}} (\sphinxstyleliteralemphasis{\sphinxupquote{Dict}}\sphinxstyleliteralemphasis{\sphinxupquote{{[}}}\sphinxstyleliteralemphasis{\sphinxupquote{str}}\sphinxstyleliteralemphasis{\sphinxupquote{, }}\sphinxstyleliteralemphasis{\sphinxupquote{int}}\sphinxstyleliteralemphasis{\sphinxupquote{ | }}\sphinxstyleliteralemphasis{\sphinxupquote{float}}\sphinxstyleliteralemphasis{\sphinxupquote{{]}}}) – Словарь правил для проверки

\sphinxlineitem{Результат}
\sphinxAtStartPar
True если правила корректны

\sphinxlineitem{Исключение}
\sphinxAtStartPar
\sphinxstyleliteralstrong{\sphinxupquote{ValueError}} – Если правила некорректны

\sphinxlineitem{Тип результата}
\sphinxAtStartPar
bool

\end{description}\end{quote}

\end{fulllineitems}\end{savenotes}

\index{make\_text\_processing() (в модуле processing\_module.calculate\_data)@\spxentry{make\_text\_processing()}\spxextra{в модуле processing\_module.calculate\_data}}

\begin{savenotes}\begin{fulllineitems}
\phantomsection\label{\detokenize{_autosummary/processing_module.calculate_data:processing_module.calculate_data.make_text_processing}}
\pysigstartsignatures
\pysiglinewithargsret
{\sphinxcode{\sphinxupquote{processing\_module.calculate\_data.}}\sphinxbfcode{\sphinxupquote{make\_text\_processing}}}
{\sphinxparam{\DUrole{n}{text}}\sphinxparamcomma \sphinxparam{\DUrole{n}{rules}}}
{}
\pysigstopsignatures
\sphinxAtStartPar
Считает количество ошибок по словарю правил для сплошного текста.
ВНИМАНИЕ: Используйте только для небольших текстов!
\begin{quote}\begin{description}
\sphinxlineitem{Результат}
\sphinxAtStartPar
Словарь с результатами анализа

\sphinxlineitem{Тип результата}
\sphinxAtStartPar
dict

\sphinxlineitem{Параметры}\begin{itemize}
\item {} 
\sphinxAtStartPar
\sphinxstyleliteralstrong{\sphinxupquote{text}} (\sphinxstyleliteralemphasis{\sphinxupquote{str}})

\item {} 
\sphinxAtStartPar
\sphinxstyleliteralstrong{\sphinxupquote{rules}} (\sphinxstyleliteralemphasis{\sphinxupquote{dict}})

\end{itemize}

\end{description}\end{quote}

\end{fulllineitems}\end{savenotes}

\index{make\_text\_processing\_stream() (в модуле processing\_module.calculate\_data)@\spxentry{make\_text\_processing\_stream()}\spxextra{в модуле processing\_module.calculate\_data}}

\begin{savenotes}\begin{fulllineitems}
\phantomsection\label{\detokenize{_autosummary/processing_module.calculate_data:processing_module.calculate_data.make_text_processing_stream}}
\pysigstartsignatures
\pysiglinewithargsret
{\sphinxcode{\sphinxupquote{processing\_module.calculate\_data.}}\sphinxbfcode{\sphinxupquote{make\_text\_processing\_stream}}}
{\sphinxparam{\DUrole{n}{text\_generator}}\sphinxparamcomma \sphinxparam{\DUrole{n}{rules}}\sphinxparamcomma \sphinxparam{\DUrole{n}{total\_chars}\DUrole{o}{=}\DUrole{default_value}{None}}}
{}
\pysigstopsignatures
\sphinxAtStartPar
Обрабатывает большие текстовые файлы чанками с прогресс\sphinxhyphen{}баром.
\begin{quote}\begin{description}
\sphinxlineitem{Параметры}\begin{itemize}
\item {} 
\sphinxAtStartPar
\sphinxstyleliteralstrong{\sphinxupquote{text\_generator}} (\sphinxstyleliteralemphasis{\sphinxupquote{Generator}}\sphinxstyleliteralemphasis{\sphinxupquote{{[}}}\sphinxstyleliteralemphasis{\sphinxupquote{str}}\sphinxstyleliteralemphasis{\sphinxupquote{, }}\sphinxstyleliteralemphasis{\sphinxupquote{None}}\sphinxstyleliteralemphasis{\sphinxupquote{, }}\sphinxstyleliteralemphasis{\sphinxupquote{None}}\sphinxstyleliteralemphasis{\sphinxupquote{{]}}}) – Генератор чанков текста

\item {} 
\sphinxAtStartPar
\sphinxstyleliteralstrong{\sphinxupquote{rules}} (\sphinxstyleliteralemphasis{\sphinxupquote{Dict}}\sphinxstyleliteralemphasis{\sphinxupquote{{[}}}\sphinxstyleliteralemphasis{\sphinxupquote{str}}\sphinxstyleliteralemphasis{\sphinxupquote{, }}\sphinxstyleliteralemphasis{\sphinxupquote{int}}\sphinxstyleliteralemphasis{\sphinxupquote{{]}}}) – Словарь правил для подсчета ошибок

\item {} 
\sphinxAtStartPar
\sphinxstyleliteralstrong{\sphinxupquote{total\_chars}} (\sphinxstyleliteralemphasis{\sphinxupquote{int}}\sphinxstyleliteralemphasis{\sphinxupquote{ | }}\sphinxstyleliteralemphasis{\sphinxupquote{None}}) – Общее количество символов для прогресс\sphinxhyphen{}бара (опционально)

\end{itemize}

\sphinxlineitem{Результат}
\sphinxAtStartPar
Словарь с результатами анализа

\sphinxlineitem{Тип результата}
\sphinxAtStartPar
dict

\end{description}\end{quote}

\end{fulllineitems}\end{savenotes}


\sphinxstepscope


\subsection{scan\_module}
\label{\detokenize{_autosummary/scan_module:module-scan_module}}\label{\detokenize{_autosummary/scan_module:scan-module}}\label{\detokenize{_autosummary/scan_module::doc}}\index{module@\spxentry{module}!scan\_module@\spxentry{scan\_module}}\index{scan\_module@\spxentry{scan\_module}!module@\spxentry{module}}\subsubsection*{Modules}


\begin{savenotes}\sphinxattablestart
\sphinxthistablewithglobalstyle
\sphinxthistablewithnovlinesstyle
\centering
\begin{tabulary}{\linewidth}[t]{\X{1}{2}\X{1}{2}}
\sphinxtoprule
\sphinxtableatstartofbodyhook
\sphinxAtStartPar
{\hyperref[\detokenize{scan_module:module-scan_module.read_files}]{\sphinxcrossref{\sphinxcode{\sphinxupquote{read\_files}}}}} (\autopageref*{\detokenize{scan_module:module-scan_module.read_files}})
&
\sphinxAtStartPar

\\
\sphinxhline
\sphinxAtStartPar
{\hyperref[\detokenize{scan_module:module-scan_module.read_layout}]{\sphinxcrossref{\sphinxcode{\sphinxupquote{read\_layout}}}}} (\autopageref*{\detokenize{scan_module:module-scan_module.read_layout}})
&
\sphinxAtStartPar

\\
\sphinxbottomrule
\end{tabulary}
\sphinxtableafterendhook\par
\sphinxattableend\end{savenotes}

\sphinxstepscope


\subsubsection{scan\_module.read\_files}
\label{\detokenize{_autosummary/scan_module.read_files:module-scan_module.read_files}}\label{\detokenize{_autosummary/scan_module.read_files:scan-module-read-files}}\label{\detokenize{_autosummary/scan_module.read_files::doc}}\index{module@\spxentry{module}!scan\_module.read\_files@\spxentry{scan\_module.read\_files}}\index{scan\_module.read\_files@\spxentry{scan\_module.read\_files}!module@\spxentry{module}}\subsubsection*{Functions}


\begin{savenotes}\sphinxattablestart
\sphinxthistablewithglobalstyle
\sphinxthistablewithnovlinesstyle
\centering
\begin{tabulary}{\linewidth}[t]{\X{1}{2}\X{1}{2}}
\sphinxtoprule
\sphinxtableatstartofbodyhook
\sphinxAtStartPar
{\hyperref[\detokenize{scan_module:id10}]{\sphinxcrossref{\sphinxcode{\sphinxupquote{count\_characters\_in\_file}}}}} (\autopageref*{\detokenize{scan_module:id10}})(filename)
&
\sphinxAtStartPar
Подсчитывает количество символов в файле
\\
\sphinxhline
\sphinxAtStartPar
{\hyperref[\detokenize{scan_module:id9}]{\sphinxcrossref{\sphinxcode{\sphinxupquote{count\_lines\_in\_file}}}}} (\autopageref*{\detokenize{scan_module:id9}})(filename)
&
\sphinxAtStartPar
Подсчитывает количество непустых строк в файле
\\
\sphinxhline
\sphinxAtStartPar
{\hyperref[\detokenize{scan_module:id8}]{\sphinxcrossref{\sphinxcode{\sphinxupquote{get\_file\_size\_mb}}}}} (\autopageref*{\detokenize{scan_module:id8}})(filename)
&
\sphinxAtStartPar
Возвращает размер файла в мегабайтах
\\
\sphinxhline
\sphinxAtStartPar
{\hyperref[\detokenize{scan_module:id5}]{\sphinxcrossref{\sphinxcode{\sphinxupquote{get\_text\_from\_file}}}}} (\autopageref*{\detokenize{scan_module:id5}})(filename)
&
\sphinxAtStartPar
Считывает весь текст из файла как единую строку.
\\
\sphinxhline
\sphinxAtStartPar
{\hyperref[\detokenize{scan_module:id6}]{\sphinxcrossref{\sphinxcode{\sphinxupquote{get\_text\_from\_file\_stream}}}}} (\autopageref*{\detokenize{scan_module:id6}})(filename{[}, chunk\_size{]})
&
\sphinxAtStartPar
Генератор для потокового чтения больших текстовых файлов.
\\
\sphinxhline
\sphinxAtStartPar
{\hyperref[\detokenize{scan_module:id0}]{\sphinxcrossref{\sphinxcode{\sphinxupquote{get\_words\_from\_file}}}}} (\autopageref*{\detokenize{scan_module:id0}})(filename)
&
\sphinxAtStartPar
Считывает построчно из файла слова и преобразует их в список для удобной обработки.
\\
\sphinxhline
\sphinxAtStartPar
{\hyperref[\detokenize{scan_module:id3}]{\sphinxcrossref{\sphinxcode{\sphinxupquote{get\_words\_from\_file\_stream}}}}} (\autopageref*{\detokenize{scan_module:id3}})(filename{[}, ...{]})
&
\sphinxAtStartPar
Генератор для потоковой обработки больших файлов.
\\
\sphinxbottomrule
\end{tabulary}
\sphinxtableafterendhook\par
\sphinxattableend\end{savenotes}
\index{get\_file\_size\_mb() (в модуле scan\_module.read\_files)@\spxentry{get\_file\_size\_mb()}\spxextra{в модуле scan\_module.read\_files}}

\begin{savenotes}\begin{fulllineitems}
\phantomsection\label{\detokenize{_autosummary/scan_module.read_files:scan_module.read_files.get_file_size_mb}}
\pysigstartsignatures
\pysiglinewithargsret
{\sphinxcode{\sphinxupquote{scan\_module.read\_files.}}\sphinxbfcode{\sphinxupquote{get\_file\_size\_mb}}}
{\sphinxparam{\DUrole{n}{filename}}}
{}
\pysigstopsignatures
\sphinxAtStartPar
Возвращает размер файла в мегабайтах
\begin{quote}\begin{description}
\sphinxlineitem{Параметры}
\sphinxAtStartPar
\sphinxstyleliteralstrong{\sphinxupquote{filename}} (\sphinxstyleliteralemphasis{\sphinxupquote{str}})

\sphinxlineitem{Тип результата}
\sphinxAtStartPar
float

\end{description}\end{quote}

\end{fulllineitems}\end{savenotes}

\index{get\_words\_from\_file() (в модуле scan\_module.read\_files)@\spxentry{get\_words\_from\_file()}\spxextra{в модуле scan\_module.read\_files}}

\begin{savenotes}\begin{fulllineitems}
\phantomsection\label{\detokenize{_autosummary/scan_module.read_files:scan_module.read_files.get_words_from_file}}
\pysigstartsignatures
\pysiglinewithargsret
{\sphinxcode{\sphinxupquote{scan\_module.read\_files.}}\sphinxbfcode{\sphinxupquote{get\_words\_from\_file}}}
{\sphinxparam{\DUrole{n}{filename}}}
{}
\pysigstopsignatures
\sphinxAtStartPar
Считывает построчно из файла слова и преобразует
их в список для удобной обработки.
ВНИМАНИЕ: Используйте только для небольших файлов!
\begin{quote}\begin{description}
\sphinxlineitem{Параметры}
\sphinxAtStartPar
\sphinxstyleliteralstrong{\sphinxupquote{filename}} (\sphinxstyleliteralemphasis{\sphinxupquote{str}})

\sphinxlineitem{Тип результата}
\sphinxAtStartPar
list

\end{description}\end{quote}

\end{fulllineitems}\end{savenotes}

\index{get\_words\_from\_file\_stream() (в модуле scan\_module.read\_files)@\spxentry{get\_words\_from\_file\_stream()}\spxextra{в модуле scan\_module.read\_files}}

\begin{savenotes}\begin{fulllineitems}
\phantomsection\label{\detokenize{_autosummary/scan_module.read_files:scan_module.read_files.get_words_from_file_stream}}
\pysigstartsignatures
\pysiglinewithargsret
{\sphinxcode{\sphinxupquote{scan\_module.read\_files.}}\sphinxbfcode{\sphinxupquote{get\_words\_from\_file\_stream}}}
{\sphinxparam{\DUrole{n}{filename}}\sphinxparamcomma \sphinxparam{\DUrole{n}{batch\_size}\DUrole{o}{=}\DUrole{default_value}{1000}}}
{}
\pysigstopsignatures
\sphinxAtStartPar
Генератор для потоковой обработки больших файлов.
Возвращает батчи слов для эффективной обработки.
\begin{quote}\begin{description}
\sphinxlineitem{Параметры}\begin{itemize}
\item {} 
\sphinxAtStartPar
\sphinxstyleliteralstrong{\sphinxupquote{filename}} (\sphinxstyleliteralemphasis{\sphinxupquote{str}})

\item {} 
\sphinxAtStartPar
\sphinxstyleliteralstrong{\sphinxupquote{batch\_size}} (\sphinxstyleliteralemphasis{\sphinxupquote{int}})

\end{itemize}

\sphinxlineitem{Тип результата}
\sphinxAtStartPar
\sphinxstyleemphasis{Generator}{[}\sphinxstyleemphasis{List}{[}str{]}, None, None{]}

\end{description}\end{quote}

\end{fulllineitems}\end{savenotes}

\index{count\_lines\_in\_file() (в модуле scan\_module.read\_files)@\spxentry{count\_lines\_in\_file()}\spxextra{в модуле scan\_module.read\_files}}

\begin{savenotes}\begin{fulllineitems}
\phantomsection\label{\detokenize{_autosummary/scan_module.read_files:scan_module.read_files.count_lines_in_file}}
\pysigstartsignatures
\pysiglinewithargsret
{\sphinxcode{\sphinxupquote{scan\_module.read\_files.}}\sphinxbfcode{\sphinxupquote{count\_lines\_in\_file}}}
{\sphinxparam{\DUrole{n}{filename}}}
{}
\pysigstopsignatures
\sphinxAtStartPar
Подсчитывает количество непустых строк в файле
\begin{quote}\begin{description}
\sphinxlineitem{Параметры}
\sphinxAtStartPar
\sphinxstyleliteralstrong{\sphinxupquote{filename}} (\sphinxstyleliteralemphasis{\sphinxupquote{str}})

\sphinxlineitem{Тип результата}
\sphinxAtStartPar
int

\end{description}\end{quote}

\end{fulllineitems}\end{savenotes}

\index{get\_text\_from\_file() (в модуле scan\_module.read\_files)@\spxentry{get\_text\_from\_file()}\spxextra{в модуле scan\_module.read\_files}}

\begin{savenotes}\begin{fulllineitems}
\phantomsection\label{\detokenize{_autosummary/scan_module.read_files:scan_module.read_files.get_text_from_file}}
\pysigstartsignatures
\pysiglinewithargsret
{\sphinxcode{\sphinxupquote{scan\_module.read\_files.}}\sphinxbfcode{\sphinxupquote{get\_text\_from\_file}}}
{\sphinxparam{\DUrole{n}{filename}}}
{}
\pysigstopsignatures
\sphinxAtStartPar
Считывает весь текст из файла как единую строку.
ВНИМАНИЕ: Используйте только для небольших файлов!
\begin{quote}\begin{description}
\sphinxlineitem{Параметры}
\sphinxAtStartPar
\sphinxstyleliteralstrong{\sphinxupquote{filename}} (\sphinxstyleliteralemphasis{\sphinxupquote{str}})

\sphinxlineitem{Тип результата}
\sphinxAtStartPar
str

\end{description}\end{quote}

\end{fulllineitems}\end{savenotes}

\index{get\_text\_from\_file\_stream() (в модуле scan\_module.read\_files)@\spxentry{get\_text\_from\_file\_stream()}\spxextra{в модуле scan\_module.read\_files}}

\begin{savenotes}\begin{fulllineitems}
\phantomsection\label{\detokenize{_autosummary/scan_module.read_files:scan_module.read_files.get_text_from_file_stream}}
\pysigstartsignatures
\pysiglinewithargsret
{\sphinxcode{\sphinxupquote{scan\_module.read\_files.}}\sphinxbfcode{\sphinxupquote{get\_text\_from\_file\_stream}}}
{\sphinxparam{\DUrole{n}{filename}}\sphinxparamcomma \sphinxparam{\DUrole{n}{chunk\_size}\DUrole{o}{=}\DUrole{default_value}{8192}}}
{}
\pysigstopsignatures
\sphinxAtStartPar
Генератор для потокового чтения больших текстовых файлов.
Возвращает чанки текста для эффективной обработки.
\begin{quote}\begin{description}
\sphinxlineitem{Параметры}\begin{itemize}
\item {} 
\sphinxAtStartPar
\sphinxstyleliteralstrong{\sphinxupquote{filename}} (\sphinxstyleliteralemphasis{\sphinxupquote{str}})

\item {} 
\sphinxAtStartPar
\sphinxstyleliteralstrong{\sphinxupquote{chunk\_size}} (\sphinxstyleliteralemphasis{\sphinxupquote{int}})

\end{itemize}

\sphinxlineitem{Тип результата}
\sphinxAtStartPar
\sphinxstyleemphasis{Generator}{[}str, None, None{]}

\end{description}\end{quote}

\end{fulllineitems}\end{savenotes}

\index{count\_characters\_in\_file() (в модуле scan\_module.read\_files)@\spxentry{count\_characters\_in\_file()}\spxextra{в модуле scan\_module.read\_files}}

\begin{savenotes}\begin{fulllineitems}
\phantomsection\label{\detokenize{_autosummary/scan_module.read_files:scan_module.read_files.count_characters_in_file}}
\pysigstartsignatures
\pysiglinewithargsret
{\sphinxcode{\sphinxupquote{scan\_module.read\_files.}}\sphinxbfcode{\sphinxupquote{count\_characters\_in\_file}}}
{\sphinxparam{\DUrole{n}{filename}}}
{}
\pysigstopsignatures
\sphinxAtStartPar
Подсчитывает количество символов в файле
\begin{quote}\begin{description}
\sphinxlineitem{Параметры}
\sphinxAtStartPar
\sphinxstyleliteralstrong{\sphinxupquote{filename}} (\sphinxstyleliteralemphasis{\sphinxupquote{str}})

\sphinxlineitem{Тип результата}
\sphinxAtStartPar
int

\end{description}\end{quote}

\end{fulllineitems}\end{savenotes}


\sphinxstepscope


\subsubsection{scan\_module.read\_layout}
\label{\detokenize{_autosummary/scan_module.read_layout:module-scan_module.read_layout}}\label{\detokenize{_autosummary/scan_module.read_layout:scan-module-read-layout}}\label{\detokenize{_autosummary/scan_module.read_layout::doc}}\index{module@\spxentry{module}!scan\_module.read\_layout@\spxentry{scan\_module.read\_layout}}\index{scan\_module.read\_layout@\spxentry{scan\_module.read\_layout}!module@\spxentry{module}}\subsubsection*{Functions}


\begin{savenotes}\sphinxattablestart
\sphinxthistablewithglobalstyle
\sphinxthistablewithnovlinesstyle
\centering
\begin{tabulary}{\linewidth}[t]{\X{1}{2}\X{1}{2}}
\sphinxtoprule
\sphinxtableatstartofbodyhook
\sphinxAtStartPar
{\hyperref[\detokenize{scan_module:scan_module.read_layout.read_kl}]{\sphinxcrossref{\sphinxcode{\sphinxupquote{read\_kl}}}}} (\autopageref*{\detokenize{scan_module:scan_module.read_layout.read_kl}})(filename)
&
\sphinxAtStartPar
Универсальная функция для чтения раскладок из различных форматов Поддерживает: JSON, CSV, TXT (key:value), XML
\\
\sphinxhline
\sphinxAtStartPar
{\hyperref[\detokenize{scan_module:scan_module.read_layout.save_layout_to_file}]{\sphinxcrossref{\sphinxcode{\sphinxupquote{save\_layout\_to\_file}}}}} (\autopageref*{\detokenize{scan_module:scan_module.read_layout.save_layout_to_file}})(layout, filename{[}, ...{]})
&
\sphinxAtStartPar
Сохраняет раскладку в файл в указанном формате
\\
\sphinxhline
\sphinxAtStartPar
{\hyperref[\detokenize{scan_module:scan_module.read_layout.validate_layout}]{\sphinxcrossref{\sphinxcode{\sphinxupquote{validate\_layout}}}}} (\autopageref*{\detokenize{scan_module:scan_module.read_layout.validate_layout}})(layout)
&
\sphinxAtStartPar
Валидирует раскладку на корректность
\\
\sphinxbottomrule
\end{tabulary}
\sphinxtableafterendhook\par
\sphinxattableend\end{savenotes}
\index{read\_kl() (в модуле scan\_module.read\_layout)@\spxentry{read\_kl()}\spxextra{в модуле scan\_module.read\_layout}}

\begin{savenotes}\begin{fulllineitems}
\phantomsection\label{\detokenize{_autosummary/scan_module.read_layout:scan_module.read_layout.read_kl}}
\pysigstartsignatures
\pysiglinewithargsret
{\sphinxcode{\sphinxupquote{scan\_module.read\_layout.}}\sphinxbfcode{\sphinxupquote{read\_kl}}}
{\sphinxparam{\DUrole{n}{filename}}}
{}
\pysigstopsignatures
\sphinxAtStartPar
Универсальная функция для чтения раскладок из различных форматов
Поддерживает: JSON, CSV, TXT (key:value), XML
\begin{quote}\begin{description}
\sphinxlineitem{Параметры}
\sphinxAtStartPar
\sphinxstyleliteralstrong{\sphinxupquote{filename}} (\sphinxstyleliteralemphasis{\sphinxupquote{str}})

\sphinxlineitem{Тип результата}
\sphinxAtStartPar
dict | None

\end{description}\end{quote}

\end{fulllineitems}\end{savenotes}

\index{\_read\_json\_layout() (в модуле scan\_module.read\_layout)@\spxentry{\_read\_json\_layout()}\spxextra{в модуле scan\_module.read\_layout}}

\begin{savenotes}\begin{fulllineitems}
\phantomsection\label{\detokenize{_autosummary/scan_module.read_layout:scan_module.read_layout._read_json_layout}}
\pysigstartsignatures
\pysiglinewithargsret
{\sphinxcode{\sphinxupquote{scan\_module.read\_layout.}}\sphinxbfcode{\sphinxupquote{\_read\_json\_layout}}}
{\sphinxparam{\DUrole{n}{filename}}}
{}
\pysigstopsignatures
\sphinxAtStartPar
Читает раскладку из JSON файла
Ожидаемый формат: \{«a»: 1, «b»: 2, …\} или \{«layout»: \{«a»: 1, «b»: 2\}\}
\begin{quote}\begin{description}
\sphinxlineitem{Параметры}
\sphinxAtStartPar
\sphinxstyleliteralstrong{\sphinxupquote{filename}} (\sphinxstyleliteralemphasis{\sphinxupquote{str}})

\sphinxlineitem{Тип результата}
\sphinxAtStartPar
dict

\end{description}\end{quote}

\end{fulllineitems}\end{savenotes}

\index{\_read\_csv\_layout() (в модуле scan\_module.read\_layout)@\spxentry{\_read\_csv\_layout()}\spxextra{в модуле scan\_module.read\_layout}}

\begin{savenotes}\begin{fulllineitems}
\phantomsection\label{\detokenize{_autosummary/scan_module.read_layout:scan_module.read_layout._read_csv_layout}}
\pysigstartsignatures
\pysiglinewithargsret
{\sphinxcode{\sphinxupquote{scan\_module.read\_layout.}}\sphinxbfcode{\sphinxupquote{\_read\_csv\_layout}}}
{\sphinxparam{\DUrole{n}{filename}}}
{}
\pysigstopsignatures
\sphinxAtStartPar
Читает раскладку из CSV файла
Ожидаемые форматы:
\sphinxhyphen{} letter,error
\sphinxhyphen{} key,value
\sphinxhyphen{} symbol,weight
\begin{quote}\begin{description}
\sphinxlineitem{Параметры}
\sphinxAtStartPar
\sphinxstyleliteralstrong{\sphinxupquote{filename}} (\sphinxstyleliteralemphasis{\sphinxupquote{str}})

\sphinxlineitem{Тип результата}
\sphinxAtStartPar
dict

\end{description}\end{quote}

\end{fulllineitems}\end{savenotes}

\index{\_read\_text\_layout() (в модуле scan\_module.read\_layout)@\spxentry{\_read\_text\_layout()}\spxextra{в модуле scan\_module.read\_layout}}

\begin{savenotes}\begin{fulllineitems}
\phantomsection\label{\detokenize{_autosummary/scan_module.read_layout:scan_module.read_layout._read_text_layout}}
\pysigstartsignatures
\pysiglinewithargsret
{\sphinxcode{\sphinxupquote{scan\_module.read\_layout.}}\sphinxbfcode{\sphinxupquote{\_read\_text\_layout}}}
{\sphinxparam{\DUrole{n}{filename}}}
{}
\pysigstopsignatures
\sphinxAtStartPar
Читает раскладку из текстового файла
Поддерживаемые форматы:
\sphinxhyphen{} key:value
\sphinxhyphen{} key=value
\sphinxhyphen{} key value
\sphinxhyphen{} key       value
\begin{quote}\begin{description}
\sphinxlineitem{Параметры}
\sphinxAtStartPar
\sphinxstyleliteralstrong{\sphinxupquote{filename}} (\sphinxstyleliteralemphasis{\sphinxupquote{str}})

\sphinxlineitem{Тип результата}
\sphinxAtStartPar
dict

\end{description}\end{quote}

\end{fulllineitems}\end{savenotes}

\index{\_read\_xml\_layout() (в модуле scan\_module.read\_layout)@\spxentry{\_read\_xml\_layout()}\spxextra{в модуле scan\_module.read\_layout}}

\begin{savenotes}\begin{fulllineitems}
\phantomsection\label{\detokenize{_autosummary/scan_module.read_layout:scan_module.read_layout._read_xml_layout}}
\pysigstartsignatures
\pysiglinewithargsret
{\sphinxcode{\sphinxupquote{scan\_module.read\_layout.}}\sphinxbfcode{\sphinxupquote{\_read\_xml\_layout}}}
{\sphinxparam{\DUrole{n}{filename}}}
{}
\pysigstopsignatures
\sphinxAtStartPar
Читает раскладку из XML файла
Ожидаемый формат:
<layout>
\begin{quote}

\sphinxAtStartPar
<key symbol=»a» error=»1»/>
<key symbol=»b» error=»2»/>
\end{quote}

\sphinxAtStartPar
</layout>
\begin{quote}\begin{description}
\sphinxlineitem{Параметры}
\sphinxAtStartPar
\sphinxstyleliteralstrong{\sphinxupquote{filename}} (\sphinxstyleliteralemphasis{\sphinxupquote{str}})

\sphinxlineitem{Тип результата}
\sphinxAtStartPar
dict

\end{description}\end{quote}

\end{fulllineitems}\end{savenotes}

\index{\_auto\_detect\_and\_read() (в модуле scan\_module.read\_layout)@\spxentry{\_auto\_detect\_and\_read()}\spxextra{в модуле scan\_module.read\_layout}}

\begin{savenotes}\begin{fulllineitems}
\phantomsection\label{\detokenize{_autosummary/scan_module.read_layout:scan_module.read_layout._auto_detect_and_read}}
\pysigstartsignatures
\pysiglinewithargsret
{\sphinxcode{\sphinxupquote{scan\_module.read\_layout.}}\sphinxbfcode{\sphinxupquote{\_auto\_detect\_and\_read}}}
{\sphinxparam{\DUrole{n}{filename}}}
{}
\pysigstopsignatures
\sphinxAtStartPar
Автоматически определяет формат файла и читает раскладку
\begin{quote}\begin{description}
\sphinxlineitem{Параметры}
\sphinxAtStartPar
\sphinxstyleliteralstrong{\sphinxupquote{filename}} (\sphinxstyleliteralemphasis{\sphinxupquote{str}})

\sphinxlineitem{Тип результата}
\sphinxAtStartPar
dict

\end{description}\end{quote}

\end{fulllineitems}\end{savenotes}

\index{\_extract\_layout\_from\_dict() (в модуле scan\_module.read\_layout)@\spxentry{\_extract\_layout\_from\_dict()}\spxextra{в модуле scan\_module.read\_layout}}

\begin{savenotes}\begin{fulllineitems}
\phantomsection\label{\detokenize{_autosummary/scan_module.read_layout:scan_module.read_layout._extract_layout_from_dict}}
\pysigstartsignatures
\pysiglinewithargsret
{\sphinxcode{\sphinxupquote{scan\_module.read\_layout.}}\sphinxbfcode{\sphinxupquote{\_extract\_layout\_from\_dict}}}
{\sphinxparam{\DUrole{n}{data}}}
{}
\pysigstopsignatures
\sphinxAtStartPar
Извлекает раскладку из словаря различной структуры
\begin{quote}\begin{description}
\sphinxlineitem{Параметры}
\sphinxAtStartPar
\sphinxstyleliteralstrong{\sphinxupquote{data}} (\sphinxstyleliteralemphasis{\sphinxupquote{Any}})

\sphinxlineitem{Тип результата}
\sphinxAtStartPar
dict

\end{description}\end{quote}

\end{fulllineitems}\end{savenotes}

\index{\_is\_numeric() (в модуле scan\_module.read\_layout)@\spxentry{\_is\_numeric()}\spxextra{в модуле scan\_module.read\_layout}}

\begin{savenotes}\begin{fulllineitems}
\phantomsection\label{\detokenize{_autosummary/scan_module.read_layout:scan_module.read_layout._is_numeric}}
\pysigstartsignatures
\pysiglinewithargsret
{\sphinxcode{\sphinxupquote{scan\_module.read\_layout.}}\sphinxbfcode{\sphinxupquote{\_is\_numeric}}}
{\sphinxparam{\DUrole{n}{value}}}
{}
\pysigstopsignatures
\sphinxAtStartPar
Проверяет, является ли строка числом
\begin{quote}\begin{description}
\sphinxlineitem{Параметры}
\sphinxAtStartPar
\sphinxstyleliteralstrong{\sphinxupquote{value}} (\sphinxstyleliteralemphasis{\sphinxupquote{str}})

\sphinxlineitem{Тип результата}
\sphinxAtStartPar
bool

\end{description}\end{quote}

\end{fulllineitems}\end{savenotes}

\index{save\_layout\_to\_file() (в модуле scan\_module.read\_layout)@\spxentry{save\_layout\_to\_file()}\spxextra{в модуле scan\_module.read\_layout}}

\begin{savenotes}\begin{fulllineitems}
\phantomsection\label{\detokenize{_autosummary/scan_module.read_layout:scan_module.read_layout.save_layout_to_file}}
\pysigstartsignatures
\pysiglinewithargsret
{\sphinxcode{\sphinxupquote{scan\_module.read\_layout.}}\sphinxbfcode{\sphinxupquote{save\_layout\_to\_file}}}
{\sphinxparam{\DUrole{n}{layout}}\sphinxparamcomma \sphinxparam{\DUrole{n}{filename}}\sphinxparamcomma \sphinxparam{\DUrole{n}{format\_type}\DUrole{o}{=}\DUrole{default_value}{'json'}}}
{}
\pysigstopsignatures
\sphinxAtStartPar
Сохраняет раскладку в файл в указанном формате
\begin{quote}\begin{description}
\sphinxlineitem{Параметры}\begin{itemize}
\item {} 
\sphinxAtStartPar
\sphinxstyleliteralstrong{\sphinxupquote{layout}} (\sphinxstyleliteralemphasis{\sphinxupquote{dict}}) – Словарь раскладки

\item {} 
\sphinxAtStartPar
\sphinxstyleliteralstrong{\sphinxupquote{filename}} (\sphinxstyleliteralemphasis{\sphinxupquote{str}}) – Путь к файлу для сохранения

\item {} 
\sphinxAtStartPar
\sphinxstyleliteralstrong{\sphinxupquote{format\_type}} (\sphinxstyleliteralemphasis{\sphinxupquote{str}}) – Формат файла („json“, „csv“, „txt“, „xml“)

\end{itemize}

\sphinxlineitem{Результат}
\sphinxAtStartPar
True если сохранение прошло успешно

\sphinxlineitem{Тип результата}
\sphinxAtStartPar
bool

\end{description}\end{quote}

\end{fulllineitems}\end{savenotes}

\index{validate\_layout() (в модуле scan\_module.read\_layout)@\spxentry{validate\_layout()}\spxextra{в модуле scan\_module.read\_layout}}

\begin{savenotes}\begin{fulllineitems}
\phantomsection\label{\detokenize{_autosummary/scan_module.read_layout:scan_module.read_layout.validate_layout}}
\pysigstartsignatures
\pysiglinewithargsret
{\sphinxcode{\sphinxupquote{scan\_module.read\_layout.}}\sphinxbfcode{\sphinxupquote{validate\_layout}}}
{\sphinxparam{\DUrole{n}{layout}}}
{}
\pysigstopsignatures
\sphinxAtStartPar
Валидирует раскладку на корректность
\begin{quote}\begin{description}
\sphinxlineitem{Параметры}
\sphinxAtStartPar
\sphinxstyleliteralstrong{\sphinxupquote{layout}} (\sphinxstyleliteralemphasis{\sphinxupquote{dict}}) – Словарь раскладки для проверки

\sphinxlineitem{Результат}
\sphinxAtStartPar
(is\_valid, list\_of\_errors)

\sphinxlineitem{Тип результата}
\sphinxAtStartPar
tuple

\end{description}\end{quote}

\end{fulllineitems}\end{savenotes}


\sphinxstepscope


\subsection{tests\_module}
\label{\detokenize{_autosummary/tests_module:module-tests_module}}\label{\detokenize{_autosummary/tests_module:tests-module}}\label{\detokenize{_autosummary/tests_module::doc}}\index{module@\spxentry{module}!tests\_module@\spxentry{tests\_module}}\index{tests\_module@\spxentry{tests\_module}!module@\spxentry{module}}\subsubsection*{Modules}


\begin{savenotes}\sphinxattablestart
\sphinxthistablewithglobalstyle
\sphinxthistablewithnovlinesstyle
\centering
\begin{tabulary}{\linewidth}[t]{\X{1}{2}\X{1}{2}}
\sphinxtoprule
\sphinxtableatstartofbodyhook
\sphinxAtStartPar
{\hyperref[\detokenize{tests_module:module-tests_module.test_imports}]{\sphinxcrossref{\sphinxcode{\sphinxupquote{test\_imports}}}}} (\autopageref*{\detokenize{tests_module:module-tests_module.test_imports}})
&
\sphinxAtStartPar
Тестовый скрипт для проверки импортов
\\
\sphinxbottomrule
\end{tabulary}
\sphinxtableafterendhook\par
\sphinxattableend\end{savenotes}

\sphinxstepscope


\subsubsection{tests\_module.test\_imports}
\label{\detokenize{_autosummary/tests_module.test_imports:module-tests_module.test_imports}}\label{\detokenize{_autosummary/tests_module.test_imports:tests-module-test-imports}}\label{\detokenize{_autosummary/tests_module.test_imports::doc}}\index{module@\spxentry{module}!tests\_module.test\_imports@\spxentry{tests\_module.test\_imports}}\index{tests\_module.test\_imports@\spxentry{tests\_module.test\_imports}!module@\spxentry{module}}
\sphinxAtStartPar
Тестовый скрипт для проверки импортов


\chapter{Индексы и таблицы}
\label{\detokenize{index:id4}}\begin{itemize}
\item {} 
\sphinxAtStartPar
\DUrole{xref}{\DUrole{std}{\DUrole{std-ref}{genindex}}}

\item {} 
\sphinxAtStartPar
\DUrole{xref}{\DUrole{std}{\DUrole{std-ref}{modindex}}}

\item {} 
\sphinxAtStartPar
\DUrole{xref}{\DUrole{std}{\DUrole{std-ref}{search}}}

\end{itemize}


\renewcommand{\indexname}{Содержание модулей Python}
\begin{sphinxtheindex}
\let\bigletter\sphinxstyleindexlettergroup
\bigletter{d}
\item\relax\sphinxstyleindexentry{data\_module}\sphinxstyleindexpageref{data_module:\detokenize{module-data_module}}
\item\relax\sphinxstyleindexentry{data\_module.make\_export\_file}\sphinxstyleindexpageref{data_module:\detokenize{module-data_module.make_export_file}}
\item\relax\sphinxstyleindexentry{data\_module.make\_export\_plot}\sphinxstyleindexpageref{data_module:\detokenize{module-data_module.make_export_plot}}
\item\relax\sphinxstyleindexentry{database\_module}\sphinxstyleindexpageref{database_module:\detokenize{module-database_module}}
\item\relax\sphinxstyleindexentry{database\_module.database}\sphinxstyleindexpageref{database_module:\detokenize{module-database_module.database}}
\indexspace
\bigletter{p}
\item\relax\sphinxstyleindexentry{processing\_module}\sphinxstyleindexpageref{processing_module:\detokenize{module-processing_module}}
\item\relax\sphinxstyleindexentry{processing\_module.calculate\_data}\sphinxstyleindexpageref{processing_module:\detokenize{module-processing_module.calculate_data}}
\indexspace
\bigletter{s}
\item\relax\sphinxstyleindexentry{scan\_module}\sphinxstyleindexpageref{scan_module:\detokenize{module-scan_module}}
\item\relax\sphinxstyleindexentry{scan\_module.read\_files}\sphinxstyleindexpageref{scan_module:\detokenize{module-scan_module.read_files}}
\item\relax\sphinxstyleindexentry{scan\_module.read\_layout}\sphinxstyleindexpageref{scan_module:\detokenize{module-scan_module.read_layout}}
\indexspace
\bigletter{t}
\item\relax\sphinxstyleindexentry{tests\_module}\sphinxstyleindexpageref{_autosummary/tests_module:\detokenize{module-tests_module}}
\item\relax\sphinxstyleindexentry{tests\_module.test\_imports}\sphinxstyleindexpageref{tests_module:\detokenize{module-tests_module.test_imports}}
\end{sphinxtheindex}

\renewcommand{\indexname}{Алфавитный указатель}
\printindex
\end{document}